% Copyright 2022  李文威 (Wen-Wei Li).
% Permission is granted to copy, distribute and/or modify this
% document under the terms of the Creative Commons
% Attribution 4.0 International (CC BY 4.0)
% http://creativecommons.org/licenses/by/4.0/

% 《代数学方法》卷一自订封面页, 由主档引入.

\setCJKfamilyfont{coverfont}{WeibeiSC-Bold}	% 设置书名字体
\setCJKfamilyfont{cover-author-font}{STXingkaiSC-Bold}	% 设置作者字体
\colorlet{octa}{cyan!50!gray}	% Color for the octahedron

\begin{titlepage}\begin{tikzpicture}[remember picture, overlay, pencildraw/.style={
		color=octa, thick,
		decorate,
		decoration={random steps, segment length=1pt, amplitude=0.7pt}
	}]

	\node[anchor=center] (title) at ([xshift=19em, yshift=-10em] current page.north west)
		{ \fontsize{45}{45}\CJKfamily{coverfont}工程数学讲义};

	\node[anchor=center] (volume) at ([yshift=-7em] title.center) {\fontsize{30}{30}\CJKfamily{coverfont}课程组专用版权~2.0};

	\node[anchor=west] (author) at ([yshift=-5em, xshift=4pt] volume.south west) {\fontsize{18}{18}\CJKfamily{cover-author-font}李小飞};
	\node[anchor=west] at ([xshift=2.4em] author.east) {\fontsize{18}{18}\CJKfamily{cover-author-font}编著};

	\draw[line width=2pt, color=black!70!gray] ([xshift=-2.5em, yshift=1em] author.north west) -- ++(25em,0);
	\shade[top color=gray, bottom color=black,] ([xshift=-1em, yshift=1em] title.north west) rectangle ++(-1em,-20em);

	% The anchors for the pictures at lower-right corner.
	\coordinate (pic) at ([xshift=-23em, yshift=22em] current page.south east);
	\coordinate (pic2) at ([xshift=3em] pic);
	
	% The pentagon axiom
	\begin{scope}[shift = (pic2), scale=1.5, opacity=0.55]
			\node (P0) at (90:2.3cm) {$x+iy,\, (x,y  \in \mathbb{R})$};
			\node (P1) at (90+72:2cm) {$r(\cos \theta + i \sin \theta )$} ;
			\node (P2) at (90+2*72:2cm) {\makebox[5ex][r]{$|z|e^{i \theta}$}};
			\node (P3) at (90+3*72:2cm) {\makebox[5ex][l]{$ w=u(x,y)+iv(x,y)$}};
			\node (P4) at (90+4*72:2cm) {$P(\theta,\varphi)$};
			\path[commutative diagrams/.cd, every arrow, every label]
				(P0) edge node[swap] {$ \hat{R}(\theta)$} (P1)
				(P1) edge node[swap] {$e^{i \pi} =-1$} (P2)
				(P2) edge node[swap, inner sep=1em] {} (P3)
				(P4) edge node {} (P3)
				(P0) edge node {$S(0,0) \sim N(\infty,\infty)$} (P4);
	\end{scope}
	
	% The octahedron
	\begin{scope}[shift=(pic), rotate=-30, scale=2]
		\coordinate (A1) at (pic);
		\coordinate (A2) at (6em, 2em);
		\coordinate (A3) at (10em, 0);
		\coordinate (A4) at (4em, -2em);
		\coordinate (B1) at (5em, 5em);
		\coordinate (B2) at (5em, -5em);
	\end{scope}

	\begin{scope}[loosely dashed, opacity=0.8]
		\draw[pencildraw] (A1) --node[sloped, below] {\small\sffamily $\frac{\partial u}{\partial x}$} (A2) --node[sloped, below] {\small\sffamily $\frac{\partial u}{\partial y}$ } (A3);
		\draw[pencildraw] (B1) --node[sloped, below] {\small\sffamily $\frac{\partial v}{\partial x}$ } (A2) --node[sloped, below] {\small\sffamily $\frac{\partial v}{\partial y}$ } (B2);
	\end{scope}
	\draw[pencildraw, fill=octa!10,opacity=0.6] (A1) --node[sloped, below] {\small\sffamily $\frac{\partial u}{\partial x}$ } (A4) --node[sloped, below] {\small\sffamily $\frac{\partial^2}{\partial x\partial y}$ } (B1);
	\draw[pencildraw, fill=octa!20,opacity=0.6] (A1) --node[sloped, below] {\small\sffamily $\frac{\partial u}{\partial y}$ } (A4) --node[sloped, below] {\small\sffamily $\frac{\partial^2}{\partial u\partial v}$ } (B2);
	\draw[pencildraw, fill=octa!30,opacity=0.6] (A3) --node[sloped, below] {\small\sffamily $\frac{\partial v}{\partial x}$ } (A4) --node[sloped, below] {\small\sffamily $\frac{\partial^2}{\partial x \partial y}$ } (B1);
	\draw[pencildraw, fill=octa!50,opacity=0.6] (A3) --node[sloped, below] {\small\sffamily $\frac{\partial v}{\partial y}$ } (A4) --node[sloped, below] {\small\sffamily $\frac{\partial^2}{\partial v \partial u}$ } (B2);
	\draw[pencildraw] (B1) --node[sloped, below] {\small\sffamily $\frac{\partial^2}{\partial u\partial v}$ } (A1) --node[sloped, below] {\small\sffamily $\frac{\partial^2}{\partial y\partial x}$ } (B2) --node[sloped, below] {\small\sffamily $\frac{\partial^2}{\partial u\partial v}$ } (A3) -- node[sloped, below] {\small\sffamily $\frac{\partial^2}{\partial y\partial x}$ } cycle;
\end{tikzpicture}

%\clearpage	% 进入内页
%\begin{center}
%	\Large{\sffamily\bfseries\thmheiti 网络版 \\ 2023 年 2 月修订} \\ \vspace{2em}
%	\Large{\sffamily\bfseries\thmheiti 编译日期: \today} \\ \vspace{1em}
%	版面: B5 (176×250mm) \\ \vspace{1em}
%	本书已由高等教育出版社发行 \\
%	(2019 年 1 月第 1 版, 2023 年 2 印) \\
%%	\texttt{ISBN: 978-7-04-050725-6}
%\end{center}
%\vfill

%\begin{flushleft} \small
%	李文威 \\
%	个人主页: \href{https://www.wwli.asia}{www.wwli.asia} \\
%	(含勘误表等信息)
%\end{flushleft}
%\vspace{1.5em}
%\begin{tabular*}{\textwidth}{ccc}
%	\includegraphics{ccby.png}
%	& \begin{minipage}[b]{0.6\textwidth}
%		\small\sffamily
%		本作品采用知识共享 署名 4.0 国际 许可协议进行许可. 访问 \url{http://creativecommons.org/licenses/by/4.0/} 查看该许可协议.
%	\end{minipage}
%\end{tabular*}

%\cleardoublepage
\end{titlepage}
