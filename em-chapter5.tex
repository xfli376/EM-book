\chapter{特殊函数与多项式}
特殊函数是指一些具有特定性质的函数。许多特殊函数来源于某个或某类微分方程的解,通常作为微分方程的解或解函数的积分出现,由于在数学分析、泛函分析、物理研究、具体工程问题等领域得到广泛应用而具有举足轻重的地位。

主要的特殊函数及多项式包括厄密多项式,勒让德多项式、球谐函数、伽马函数、贝塞耳函数、椭圆函数、超几何函数、合流超几何函数、马丢函数、拉梅函数、$\zeta$函数及$Li$多重对数函数等。

\section{厄密多项式}
厄密多项式在数学与物理上都有重要应用。在概率论中,埃奇沃斯级数的表达式中含厄密多项式,在
组合数学中,阿佩尔微分方程
\begin{equation}
	\left(e^{-x^{2} / 2} u^{\prime}\right)^{\prime}+\lambda e^{-x^{2} / 2} u=0
\end{equation}
的解函数就是厄密多项式。取边界条件为$u$在无穷远处有界,则方程的固有值应取自然数$\lambda \in \mathbb{N}$。而属于$\lambda = n$的固有函数正是$n$阶厄密多项式: 
\begin{equation}
	u_{n}(x)=(-1)^{n} e^{x^{2} / 2} \frac{d^{n}}{d x^{n}} e^{-x^{3} / 2}
\end{equation}
在物理学中,厄密方程 
\begin{equation}
	H'' -2 x H' +2\lambda H=0 
\end{equation}
在边界条件为$H$在无穷远处为零时,其固有值也取自然数$n$,对应的固有解函数就是厄密多项式
\begin{equation}
	H_{n}(x)=(-1)^{n} e^{x^{2}} \frac{d^{n}}{d x^{n}} e^{-x^{2}}
\end{equation}
这两种厄密多项式略有不同,两者的关系为
\begin{equation}
	H_{n}(x) = 2^{n/2} u_n(\sqrt{2}x)
\end{equation}
下面我们给出厄密多项式的获得过程及其重要性质。
	 
\subsection{厄密方程}
\begin{example}
	求解n阶厄密方程
	\begin{equation}\label{eq:hemfc}
		H'' -2 x H' +2n H=0 
	\end{equation} 
\end{example}
\emph{解:}
	设方程有级数解
	\begin{equation*}
		H=\sum_{k=0}^{\infty} c_k x ^k
	\end{equation*}     
	求导
	$$\begin{aligned}
		H'=\sum_{k=1}^{\infty} k c_k x ^{k-1}, \qquad  
		2 x H' =\sum_{k=0}^{\infty} 2 k c_k x ^{k}
	\end{aligned}$$ 
	再求导
	$$\begin{aligned}
		H''&=\sum_{k=2}^{\infty} k(k-1) c_k x ^{k-2} \\
		   &=\sum_{k-2=0}^{\infty} (k-2+2)(k-2+1) c_{k-2+2} x ^{k-2} \\
		   &=\sum_{k=0}^{\infty} (k+2)(k+1) c_{k+2} x ^{k}
	\end{aligned}$$  
   代回厄密方程,得
   \begin{equation*}
	   \sum_{k=0}^{\infty} (k+2)(k+1) c_{k+2} x ^{k} -\sum_{k=0}^{\infty} 2 k c_k x ^{k} +2n \sum_{k=0}^{\infty} c_k x ^k=0 
   \end{equation*}
   这是有关$x$的多项式, 系数项应为零
   \[(k+2)(k+1) c_{k+2} - 2 k c_k  + 2n c_k =0 \]
   整理得系数递推式
   \begin{equation*}
	   c_{k+2} = \frac{ 2(k-n)}{(k+2)(k+1) } c_k, ~~  \left( k=0,1,2,3, ...  \right)
   \end{equation*} 
	对于n阶厄密方程来说, $n$是确定数, 而$k$是级数的指标。当$k$增加到$k=n$时,系数$c_{k+2} =0 $ 。因此, 无论$n$是奇还是偶数, 级数的最高次都是$k=n$,令最高次项系数为:
	\begin{equation*}
		c_n =2^n
	\end{equation*}
	系数降幂递推式,\begin{equation*}
		c_{k-2} = -\frac{k(k-1) } { 2(n-(k-2))}  c_k
	\end{equation*}   
	取$k=n$
	\begin{equation*}
		\begin{aligned}
			c_{n-2} &=-\frac{n(n-1) } { 2(n-(n-2))}  c_n  = -\frac{n(n-1) } { 2\times2}  2^n 
		\end{aligned}
	\end{equation*} 
	\begin{equation*}
		\begin{aligned}
			c_{n-4} &=-\frac{(n-2)(n-3) } { 2(n-(n-4))}  c_{n-2}  \\
			&=(-1)^2 \frac{(n-2)(n-3) } { 2(n-(n-4))}\frac{n(n-1) } { 2\times2}  2^n   \\
			&= (-1)^2 \frac{n(n-1)(n-2) (n-3) } { (2\times 2)\times (2\times 4)}  2^n
		\end{aligned}
	\end{equation*}
	\begin{equation*}
		\begin{aligned}
			c_{n-6} &= (-1)^3 \frac{n(n-1)(n-2) (n-3)(n-4) (n-5) } { (2\times 2)\times (2\times 4)\times (2\times 6)}  2^n \\
		&= (-1)^3 \frac{n!/(n-6)!} {2^3 (6)!!}  2^n \\
		c_{n-2\times 3} &= (-1)^3 \frac{n!/(n-2\times 3)!} {2^3 2^3 3!}  2^n \\
		&= (-1)^3 \frac{n!} { 3! (n-2\times 3)!}  2^{n-2\times 3}
		\end{aligned}
	\end{equation*} 
用$m$取代$3$,得一般式
	\begin{equation*}
 c_{n-2m} =(-1)^m \frac{n! } {  m ! (n-2m)!}  2^{n-2m} 
	\end{equation*}
	无论奇偶, n阶厄密多项式总可以表示为
	\begin{equation}\label{eq:njhemdxs}
		\boxed{H_n(x) =\sum_{m=0}^{M}  (-1)^m \frac{n! } {  m ! (n-2m)!}  2^{n-2m} x^{n-2m} ,  ~~~ M=[n/2] }
	\end{equation}   
	式中$M=[n/2]$表示向下取整。\\
~~\\
至此,$n$阶厄密方程有两个基本解函数:一个是$n$阶厄密多项式,一个是无穷级数。当$n$为偶数时,奇数阶项构成无穷级数,当$n$为奇数时,偶数阶项构成无穷级数。在实际物理体系中,这个无穷级数解通常不满足边界条件,应删除。即:n阶厄密方程[\ref{eq:hemfc}]的解为$n$阶厄密多项式[\ref{eq:njhemdxs}]。

\subsection{厄密多项式的性质}

\begin{proposition}
	厄密多项式存在生成函数
	\begin{equation}\label{eq:Hermi-0}
     w(x,t)=e^{2xt-t^2}
	\end{equation}
\end{proposition}
\begin{proof}
 这是一个二元函数, 对它做关于变量$t$的$Taylor$展开
 \begin{equation*}
	w(x,t) =\sum_{n=0}^{\infty} \frac{1}{n!}  c_n(x) t^n
\end{equation*}
如果展开系数函数$c_n(x)$是厄密多项式, 则必满足厄密方程
\begin{equation*}
	\left[  \frac{d^2}{dx^2} -2x\frac{d}{dx} +2n  \right] c_n(x)=0
\end{equation*}
1)、针对等式  
$$ w(x,t) = \sum_{n=0}^{\infty} \frac{1}{n!}  c_n(x) t^n = e^{2xt-t^2} $$ 
对$x$求导
$$
\begin{aligned}
	\dfrac{\partial }{\partial x} \sum_{n=0}^{\infty} \frac{1}{n!}  c_n (x) t^n =  \dfrac{\partial }{\partial x} e^{2xt-t^2}
\end{aligned}  
$$
$$
\begin{aligned}
	\sum_{n=0}^{\infty} \frac{1}{n!}  c'_n (x) t^n  = 2t e^{2xt-t^2} = 2t ~w(x,t)
\end{aligned}  
$$
代入$w(x,t)$的展开式
$$
\begin{aligned}
	\sum_{n=0}^{\infty} \frac{1}{n!}  c'_n (x) t^n 
	&= 2t \sum_{n=0}^{\infty} \frac{1}{n!}  c_n(x) t^n \\
	&= \sum_{n=0}^{\infty} \frac{1}{n!} 2 c_n(x) t^{n+1} \\ 
	&= \sum_{n=1}^{\infty} \frac{1}{n!} 2n c_{n-1}(x) t^{n} \\ 
\end{aligned}  
$$
整理,得
$$ \sum_{n=0}^{\infty} \frac{1}{n!}  c'_n (x) t^n  - \sum_{n=1}^{\infty} \frac{1}{n!} 2n c_{n-1}(x) t^{n} =0 $$
多项式的各项系数应为零,得递推式-I
\begin{equation}\label{eq:Hermi-1}
	c'_n (x)=2n c_{n-1} (x)	
\end{equation}
求导
$$
c~''_n(x)= 2n c'_{n-1} (x) 
$$
显然有
$$c'_{n-1}(x) = 2(n-1) c_{n-2} (x) $$
代回上式, 得递推式-II
\begin{equation}\label{eq:Hermi-2}
	c~''_n(x)= 4n(n-1) c_{n-2} (x)
\end{equation}
~\\
2)针对等式 
$$ w(x,t) = e^{2xt-t^2} $$ 
对$t$求导
$$
\begin{aligned}
	\dfrac{\partial w(x,t)}{\partial t} &= \dfrac{\partial }{\partial t} e^{2xt-t^2} \\
	&= 2(x-t) e^{2xt-t^2}  \\ 
	&= 2(x-t) ~w(x,t) 
\end{aligned}
$$
移项,得   
$$  \dfrac{\partial w(x,t)}{\partial t} +2(t-x) ~w(x,t) =0$$  
代入$w(x,t)$的展开式 
$$
\begin{aligned}
	 \dfrac{\partial}{\partial t} \sum_{n=0}^{\infty} \frac{1}{n!}  c_n(x) t^n +2(t-x) \sum_{n=0}^{\infty} \frac{1}{n!}  c_n(x) t^n = 0 
\end{aligned}
$$ 
$$
\begin{aligned}
	 &\sum_{n=0}^{\infty} \frac{1}{n!} n c_n(x) t^{n-1} + \sum_{n=0}^{\infty} \frac{2}{n!}  c_n(x) t^{n+1} + \sum_{n=0}^{\infty} \frac{-2x}{n!}  c_n(x) t^n = 0 \\
	 &\sum_{n=0}^{\infty} \frac{1}{(n-1)!} c_n(x) t^{n-1} + \sum_{n=0}^{\infty} \frac{2(n+1)}{(n+1)!}  c_n(x) t^{n+1} + \sum_{n=0}^{\infty} \frac{-2x}{n!}  c_n(x) t^n = 0 
\end{aligned}
$$ 
做脚标变换
$$
\sum\limits_{n=0}^{\infty} \dfrac{1}{n!} c_{n+1}(x) t^{n} +\sum\limits_{n=1}^{\infty}\dfrac{2n}{n!} c_{n-1}(x)t^n +\sum\limits_{n=0}^{\infty} \dfrac{-2x}{n!} c_n(x)t^n =0   
$$ 
左端是关于$t$的多项式, 各项系数应为零,有
$$ c_{n+1}(x) -2xc_n(x) +2nc_{n-1} (x) =0 $$ 
向下递推,有
\begin{equation}\label{eq:Hermi-3}
	c_{n}(x) -2xc_{n-1}(x) +2(n-1)c_{n-2} (x) =0	
\end{equation}
整理
$$  c_{n}(x) - \dfrac{x}{n} (2n c_{n-1} (x))  +\dfrac{1}{2n} (4n(n-1) c_{n-2} (x)) =0 $$ 
代入[\ref{eq:Hermi-1}]及[\ref{eq:Hermi-2}],得
$$  c_{n}(x) - \dfrac{x}{n}c~'_{n}(x) +\dfrac{1}{2n}c~''_{n} (x) =0 $$ 
整理
\begin{equation*}
	 \left[  \dfrac{d^2}{dx^2} -2x\frac{d}{dx} +2n  \right] c_n(x)=0 
\end{equation*}
表明 $c_n(x)$满足厄密方程, 因此有  
$$ c_n(x)= H_n (x)  $$ 
\textcolor{red}{证毕!}
\end{proof}

~~\\ 

\begin{proposition}
厄密多项式有递推关系
\begin{equation}
	\begin{aligned}
		& H'_n(x)=2nH_{n-1}(x) \\ 
		& H_{n}(x) -2xH_{n-1}(x) +2(n-1)H_{n-2} (x) =0  \\ 
	\end{aligned}
\end{equation}
\end{proposition}
\begin{proof}
把关系
$$ c_n(x)= H_n (x)  $$  
代入式 [\ref{eq:Hermi-1}]
\begin{equation*}
	c'_n (x)=2n c_{n-1} (x)	
\end{equation*}
得
$$
H'_n(x)=2nH_{n-1}(x)  
$$ 
同理,代入式 [\ref{eq:Hermi-3}]
\begin{equation*}
	c_{n}(x) -2xc_{n-1}(x) +2(n-1)c_{n-2} (x) =0	
\end{equation*}
得
$$
H_{n}(x) -2xH_{n-1}(x) +2(n-1)H_{n-2} (x) =0  
$$ 
\textcolor{red}{证毕!}
\end{proof}
~~\\ 

\begin{example}
	试求前五个厄密多项式
	\end{example}
   \emph{解:}
	有厄密多项表达式
		\begin{equation*}
			H_n(x) =\sum_{m=0}^{M}  (-1)^m \frac{n! } {  m ! (n-2m)!}  2^{n-2m} x^{n-2m} ,  ~~~ M=[n/2]
		\end{equation*} 
	(1) 取 $n=0$, 有 $M=[n/2] =0$, 有 $m=0$, 代入上式, 得 
	\[H_0 (x) = 1\]  
	(2) 取 $n=1$, 有 $M=[n/2] =0$, 有 $m=0$, 代入上式, 得 
	\[H_1 (x) = 2x\] 
	(3) 递推关系 $H_{n}(x) -2xH_{n-1}(x) +2(n-1)H_{n-2} (x) =0$ 中, 取 $n=2$, 得
	$$H_{2}(x) -2xH_{1}(x) +2(2-1)H_{0} (x) =0 $$ 
	代入 $H_0 (x) = 1, H_1 (x) = 2x$, 得 
	\[H_2 (x) = 4x^2 -2 \] 
	(4) 依次递推, 可得其他厄密多项式,
\begin{figure}[h]
   \centering
   \includegraphics[width=0.6\textwidth]{figs/h0to5.png}
   \caption{厄密多项式图例}
   %\label{fig:}
\end{figure}
前七个厄密多项式具体表达为:
$$
\begin{aligned}
   & H_0(x)=1 \\
   & H_1(x)=2 x \\
   & H_2(x)=4 x^2-2 \\
   & H_3(x)=8 x^3-12 x \\
   & H_4(x)=16 x^4-48 x^2+12 \\
   & H_5(x)=32 x^5-160 x^3+120 x \\
   & H_6(x)=64 x^6-480 x^4+720 x^2-120
   \end{aligned}
   $$


	~~\\ 

\begin{proposition}
	厄密多项式有如下微分表达式
		\begin{equation}
			H_n(x) =(-1) ^n e^{x^2}  \frac{d~^n }{d~x^n}  e^{-x^2}
		\end{equation}
\end{proposition}
	\begin{proof}
	对厄密多项式的母函数作关于$t$的$Taylor$展开
	\begin{equation*}
		w(x,t)=e^{2xt-t^2} = \sum_{n=0}^{\infty} \frac{1}{n!} \left[  \frac{\partial ~^n w  }{\partial t^n}  \right] _{t=0} t^n 
	\end{equation*}
	展开系数就是厄密多项式
	\[ 
	\begin{aligned}
		H_n(x) & = \left[  \frac{\partial^n w  }{\partial t^n}  \right] _{t=0} 
	\end{aligned} \]
	代入母函数的解析式
	\[ 
	\begin{aligned}
		H_n(x) & = \left[  \frac{\partial^n   }{\partial t^n} e^{2xt-t^2} \right] _{t=0} \\
		& = \left[  \frac{\partial^n   }{\partial t^n} e^{x^2 -(x-t)^2} \right] _{t=0} \\
		& = e^{x^2} \left[  \frac{\partial^n   }{\partial t^n} e^{-(x-t)^2} \right] _{t=0} 
	\end{aligned} \]
	做变量代换$u=x-t$
	\[ 
	\begin{aligned}
		H_n(x)
		&= (-1) ^n e^{x^2}   \left[  \frac{d^n }{du^n}  e^{-u^2}   \right] _{u=x}  \\
		&= (-1) ^n e^{x^2}  \frac{d~^n }{d~x^n}  e^{-x^2}
	\end{aligned} \]
	证毕!
	\end{proof}
~~\\ 

\begin{proposition}-4:
	试证明厄密多项式具有带权正交归一性
	\[ \frac{1}{2^n n! \sqrt{\pi} }\int\limits_{-\infty}^{+\infty} e^{-x^{2}} H_m(x) H_n(x)dx = \left\{ 
		\begin{aligned}
			& 0, \quad (m\ne n) \\ 
			& 1, \quad (m= n)
		\end{aligned}\right.\]
\end{proposition}
\begin{proof}
	(I)构造方程
	$$\Psi'' +(2n+1-x^2) \Psi =0 $$ 
	当 $ |x| \to \infty$,方程可近似为
		 \begin{equation}\label{eq:osc1}
			 \left(\frac{\mathbf{d} ^2}{\mathbf{d} x^2} - x^2 \right) \Psi=0 
		 \end{equation} 
		 考虑平方指数函数解
		 $$ \exp(\frac{x ^2}{2}), \quad \exp(-\frac{x ^2}{2})  $$ 
		 对它们求导
		 \begin{equation*}
			 \frac{d^2 }{d x ^2} \exp(\frac{x ^2}{2}) =(x ^2 +1)  \exp(\frac{x ^2}{2}) 
		 \end{equation*}    
		 \begin{equation*}
			 \frac{d^2 }{d x ^2} \exp( - \frac{x ^2}{2}) =(x ^2 -1)  \exp( - \frac{x ^2}{2}) 
		 \end{equation*}  
		 当 $ x \to \infty$, 它们可近似为:\\
		 \begin{equation*}
			 (x ^2 )  \exp( \frac{x ^2}{2}) ~~, ~~ (x ^2 )  \exp( - \frac{x ^2}{2}) 
		 \end{equation*} 
		 代回方程[\ref{eq:osc1}]
		 \begin{equation*}
			 \left(\frac{\mathbf{d} ^2}{\mathbf{d} x^2} - x^2 \right) \Psi \approx (x ^2 )  \exp( \frac{x ^2}{2}) - x ^2   \exp( \frac{x ^2}{2}) =0 
		 \end{equation*}
		 即上述平方指数函数近似地满足方程,因此,极限状态下方程的解应与如下函数相关联
		 \begin{equation*}
			 C_1  \exp( \frac{x ^2}{2}) + C_2   \exp( - \frac{x ^2}{2})  
		 \end{equation*}     
		 删除发散项(第一项),得极限状态解函数形式 \\
		 \begin{equation*}
			 \Psi_\infty (x)  \sim C_2 \exp( - \frac{x ^2}{2})  
		 \end{equation*}  
		 (2) 非极限状态的解函数可考虑对极限状态解函数作常数变异 
		 \begin{equation*}
			 \Psi(x) = H(x) e^{-x^2/2 }  
		 \end{equation*}   
		 求导
		 \begin{equation*}
			 \Psi'(x) = H'(x) e^{-x^2/2 } -  H(x) x e^{-x^2/2 } 
		 \end{equation*} 
		 再求导 
		 \begin{equation*}
			 \Psi''(x) = \left[  \left( x^2 -1 \right) H -2x H' +H''  \right] e^{-x^2/2}
		 \end{equation*}  
		 代回方程[\ref{eq:osc1}], 得 
		 $$ \left[  \left( x^2 -1 \right) H -2x H' +H''  \right] e^{-x^2/2} + \left( 2n+1 - x^2 \right) H(x) e^{-x^2/2 }=0 $$  \\ 
		 整理
		 \[\left[H'' -2 x H' + 2n H \right]e^{-x^2/2 } =0\]
		 由系数项应为零, 得
		 \[H'' -2 x H' +2n H =0\]
		 这正是n阶厄密方程 \\
		 ~\\
		 因此,方程[\ref{eq:osc1}]的属于$n$的解函数为 $$\Psi_n(x)=H_n(x) e^{-x^{2}/2}$$
	考虑两个解函数的内积,有
	\begin{equation}\label{eq:hemdqzj}
		\int\limits_{-\infty}^{+\infty} \Psi_n(x) \Psi_m(x) d x =\int\limits_{-\infty}^{+\infty} e^{-x^{2}} H_m(x) H_n(x)dx
	\end{equation}
	两解函数都满足方程
	$$\Psi _n'' +(2n+1-x^2) \Psi _n =0 \quad (1)$$
	$$\Psi _m'' +(2m+1-x^2) \Psi _m =0 \quad (2)$$
	$(1)\times \Psi _m - (2)\times \Psi _n$, 得
	$$\Psi _n''\Psi _m - \Psi _m''\Psi _n = 2(m-n)\Psi_n \Psi_m $$
	左端积分
	\[ 
	\begin{aligned}
		\int\limits_{-\infty}^{+\infty}  [\Psi _n''\Psi _m - \Psi _m''\Psi _n]dx 
		&= [\Psi_m \Psi''_n -\Psi_n \Psi''_m]_{-\infty} ^{+\infty} - \int_{-\infty} ^{+\infty} [\Psi'_m \Psi'_n -\Psi'_m \Psi'_m] dx \\
		&= 0  
	\end{aligned}
	\]
	因此, 右端积分也为零
	\[2(m-n)\int\limits_{-\infty}^{+\infty} \Psi_n \Psi_m dx =0 \]
	当 $m\ne n$时, 只有 
	\[\int\limits_{-\infty}^{+\infty} \Psi_n \Psi_m dx =0 \]
	代回式[\ref{eq:hemdqzj}], 得 
	\[\int\limits_{-\infty}^{+\infty} e^{-x^{2}} H_m(x) H_n(x)dx\]
	当$m \ne n$时,两厄密多项式带权正交。\\
    ~\\
	(II)由递推公式:
	$$H_{n} -2xH_{n-1} +2(n-1)H_{n-2} =0  $$ 
	$$ \implies H^2_{n}-2xH_n H_{n-1}+2(n-1) H_n H_{n-2} =0  $$
	$$H_{n+1} -2xH_{n} +2nH_{n-1} =0  $$ 
	$$\implies H_{n+1} H_{n-1}-2xH_{n} H_{n-1}+2nH^2_{n-1} =0  $$ 
	两式相减
	$$H^2 _n(x) -H_{n+1} H_{n-1}=2n H^2 _{n-1}(x) - 2(n-1) H_n H_{n-2}$$	  
	两端同乘权重函数 
	$$ e^{-x^2} H^2 _n(x) dx -  e^{-x^2} H_{n+1} H_{n-1} dx = 2n  e^{-x^{2}} H^2 _{n-1}(x) dx  - e^{-x^{2}} 2(n-1) H_n H_{n-2} dx $$ 	
	两端积分
	$$ \int\limits_{-\infty}^{+\infty} e^{-x^2} H^2 _n(x) dx - \int\limits_{-\infty}^{+\infty} e^{-x^2} H_{n+1} H_{n-1} dx = 2n \int\limits_{-\infty}^{+\infty} e^{-x^{2}} H^2 _{n-1}(x) dx  - \int\limits_{-\infty}^{+\infty} e^{-x^{2}} 2(n-1) H_n H_{n-2} dx $$ 	
	由带权正交性可知,左边第二项和右边第二项都为零,得递推式
	\[
	\begin{aligned}
		\int\limits_{-\infty}^{+\infty} e^{-x^2} H^2 _n(x) dx &=2n \int\limits_{-\infty}^{+\infty} e^{-x^{2}} H^2 _{n-1}(x) dx\\ 
		&= 2n \times 2(n-1) \int\limits_{-\infty}^{+\infty} e^{-x^{2}} H^2 _{n-2}(x) dx  \\
		&= 2n \times 2(n-1) ... (2(n-n)) \int\limits_{-\infty}^{+\infty} e^{-x^{2}} H^2 _{0}(x) dx 
	\end{aligned}
	\]
	整理,并代入$H _{0}(x)=1$
	\[
	\begin{aligned}
		\int\limits_{-\infty}^{+\infty} e^{-x^2} H^2 _n(x) dx
		&= 2^n n! \int\limits_{-\infty}^{+\infty} e^{-x^{2}} H^2 _{0}(x) dx  \\	
		&= 2^n n! \int\limits_{-\infty}^{+\infty} e^{-x^{2}} dx  \\	
		&= 2^n n! \sqrt{\pi}  
	\end{aligned}
	\]
   \textcolor{red}{证毕!}
\end{proof}

厄密方程的固有解函数(厄密多项式)构成正交归一完全集,因此,一般的函数可以做厄密多项式展开。

\begin{example}
	求函数
	\[ f(x) =2x^3 +3x \] 
	的厄密展开式
\end{example}
   \emph{解:}设$f(x)$的厄密展开式为
	\[ f(x) = \sum_{n=0}^{+\infty} a_n H_n(x) \] 
	已知前四厄密多项式如下
	\[\begin{aligned} H_0 (x) &= 1,\quad H_1 (x) = 2x \\ H_2 (x) &= 4x^2 -2, \quad  H_3 (x) = 8x^3 -12x \end{aligned}\]
	考虑幂阶关系, 展开式中的最高阶应为$n=3$
	\[
	\begin{aligned}
		f(x) & = \sum_{n=0}^{3} a_n H_n(x) \\ 
		&= a_0 H_0(x) + a_1 H_1(x) + a_2 H_2(x) +  a_3 H_3(x) \\
		&= a_0 + a_1 (2x) + a_2 (4x^2 -2)  + a_3 (8x^3 -12x) \\
		&= (a_0 -2 a_2) + (2 a_1 - 12a_3 )x +  4 a_2 x^2 + 8a_3 x^3 
	\end{aligned}
	\]
	代入 $f(x)$, 并整理, 得
	\[ (a_0 -2 a_2) + (2 a_1 - 12a_3 -3)x +  4 a_2 x^2 + (8a_3 -2) x^3 = 0 \]
	多项式等于零的充要条件是各项系数都为零, 解得
	\[\left\{
	\begin{aligned}
		a_0 &= 0, \quad 
		a_1 = 3\\
		a_2 &= 0, \quad 
		a_3 = \frac{1}{4}
	\end{aligned} \right.
	\]
	因此,有 
	\[ f(x) =   \frac{1}{4} H_3(x)+ 3 H_1(x)\] 
	当然,也可用拼凑法
   \[
  \begin{aligned}
   f(x) &= 2x^3 +3x \\
   &= \frac{1}{4}(8x^3 -12x)+ 3x +3x \\
   &= \frac{1}{4}(8x^3 -12x)+ 3(2x) \\
   &= \frac{1}{4} H_3(x) + 3H_1(x)
  \end{aligned}
	\]

\section{球谐函数}
拉普拉斯方程只能在四种已知坐标系中分离变量: 笛卡尔坐标系、圆柱坐标系、球面坐标系和椭圆坐标系,球谐函数是拉普拉斯方程在球面坐标系中的解函数。

对于拉普拉斯方程 
$$ \nabla ^2 u = 0 $$	
代入球面坐标系拉普拉斯算子
$$ \displaystyle  \nabla ^{2} =\frac{1}{r^2} \frac{\partial }{\partial r} (r^2\frac{\partial }{\partial r} )+
	\frac{1}{r^2 \sin \theta  } \frac{\partial }{\partial \theta } (\sin \theta \frac{\partial }{\partial \theta } )
	+\frac{1}{r^2 \sin^2 \theta  } \frac{\partial^2}{\partial\varphi ^2}$$
并考虑到
\begin{equation*}
	L^2 =  -\left[ \frac{1}{ \sin \theta  } \frac{\partial }{\partial \theta } (\sin \theta \frac{\partial }{\partial \theta } )
	+\frac{1}{ \sin^2 \theta  } \frac{\partial^2}{\partial\varphi ^2} \right]
\end{equation*}	
拉普拉斯方程变为 
\begin{equation*}
	\left[\frac{1}{r^2}  \frac{\partial }{\partial r} (r^2\frac{\partial }{\partial r} ) - \frac{1}{r^2} L^2  \right] u
	=0
\end{equation*}	
可做径向/角向分离变量,令 
\begin{equation*}
	u=R (r) Y(\theta,\varphi)
\end{equation*}	
代回方程,得
\begin{equation*}
	\frac{ L^2 Y}{Y}= \frac{1}{R}   \frac{\partial }{\partial r} (r^2\frac{\partial R }{\partial r} )=\lambda
\end{equation*}
得两个方程	\\
(1)径向方程:
\begin{equation}\label{eq:r}
	\frac{d}{d r} (r^2\frac{d R }{d r} ) -\lambda R =0
\end{equation}	
(2)角向方程:
\begin{equation*}
	L^2 Y=\lambda Y
\end{equation*}	
代入$L^2$的具体形式
	\begin{equation*}
		\left[ \frac{1}{ \sin \theta  } \frac{\partial }{\partial \theta } (\sin \theta \frac{\partial }{\partial \theta } )
		+\frac{1}{ \sin^2 \theta  } \frac{\partial^2}{\partial\varphi ^2}  +\lambda \right] Y=0 
	\end{equation*}
    可进一步分离变量, 令:
	\begin{equation*}
		Y(\theta,\varphi)= \Theta(\theta) \Phi(\varphi)
	\end{equation*}	
	代回,得
	\begin{equation*}
		\Phi \frac{1}{\sin \theta} \frac{d}{d \theta}\left(\sin \theta \frac{d \Theta}{d \theta}\right)+\Theta \frac{1}{\sin ^{2} \theta} \frac{d^{2} \Phi}{d \varphi^{2}}+\lambda \Theta \Phi=0
	\end{equation*}	
	整理得恒等式,可令其等于任意常数$\lambda'$
	\begin{equation*}
		\frac{\sin ^{2} \theta}{\Theta \sin \theta} \frac{d}{d \theta}\left(\sin \theta \frac{d \Theta}{d \theta}\right)+\sin ^{2} \theta \lambda=-\frac{1}{\Phi} \frac{d^{2} \Phi}{d \varphi^{2}}=\lambda'
	\end{equation*}
	得两个方程 \\
	(2')纬度方程:
	\begin{equation}\label{eq:theta}
		\frac{1}{\sin \theta} \frac{d}{d \theta}\left(\sin \theta \frac{d \Theta}{d \theta}\right)+\left[\lambda-\frac{\lambda'}{\sin ^{2} \theta}\right] \Theta=0,(0<\theta \le \pi)
	\end{equation}		
	(2'')经度方程:
	\begin{equation}\label{eq:varphi}
		\frac{d^{2} \Phi}{d \varphi^{2}}+\lambda' \Phi=0,(0<\varphi\le2 \pi)
	\end{equation}	
	最终,球坐标系下的拉普拉斯方程被分离成三个方程:径向方程 [\ref{eq:r}], 纬度方程[\ref{eq:theta}], 经度方程[\ref{eq:varphi}]。 这是三个一维的微分方程, 理论上是可以精确求解的。

	\subsection{解经度方程}
	\begin{example}
	求解经度方程
	\begin{equation*}
		\frac{d^{2} \Phi}{d \varphi^{2}}+\lambda' \Phi=0,(0<\varphi\le2 \pi)
	\end{equation*}
	\end{example}
	\emph{解:}
	这是二阶常微分方程, 存在两个待定系数, 它们可由周期性边界条件确定, 因此是如下边值问题
		\[\begin{cases}
			\dfrac{d^{2} \Phi}{d \varphi^{2}}+\lambda' \Phi=0,\quad 0 < \varphi < 2 \pi \\ 
			\Phi(0)=\Phi(2 \pi), \quad \Phi^{\prime}(0)=\Phi^{\prime}(2 \pi)
		\end{cases}\]
		有解\\	
		固有值和固有函数
		\[\begin{cases}
			\lambda'=m^2, ~~~ (m=0,\pm 1,\pm 2,\cdots) \\ 
			\Phi_m (\varphi)=A_m e^{im\varphi}
		\end{cases}\]	
		求归一化系数 
		\begin{equation*}
		\begin{split}
			\int_{0}^{2\pi}  |\Phi_m (\varphi)|^2 d\varphi &= 1 \\
			\int_{0}^{2\pi}  A_m e^{im\varphi} A_m e^{-im\varphi} d\varphi &= 1 \\
			|A_m|^2 \int_{0}^{2\pi} 1 d\varphi &= 1 \\
			|A_m|^2 2\pi &= 1 \\
			A_m&=\frac{1}{\sqrt{2\pi}} 
		\end{split}
		\end{equation*}
		代回, 得归一化固有函数	
		\begin{equation}
		\boxed{\Phi_m (\varphi)=\frac{1}{\sqrt{2\pi}} e^{im\varphi}, \qquad (m=0,\pm 1,\pm 2,\cdots)}  
		\end{equation}	
	结束! 
	
	~~\\ 
	
	\subsection{解纬度方程}
	
	\begin{example}
		求解纬度方程
		\begin{equation*}
			\frac{1}{\sin \theta} \frac{d}{d \theta}\left(\sin \theta \frac{d \Theta}{d \theta}\right)+\left[\lambda-\frac{\lambda'}{\sin ^{2} \theta}\right] \Theta=0,\qquad (0<\theta \le \pi)
		\end{equation*}		
	\end{example}
	\emph{解:}
		把经度方程中解得的 $\lambda'=m^2$代回纬度方程, 得
		\begin{equation}
			\frac{1}{\sin \theta} \frac{d}{d \theta}\left(\sin \theta \frac{d \Theta}{d \theta}\right)+\left[\lambda-\frac{m^{2}}{\sin ^{2} \theta}\right] \Theta=0
		\end{equation}	
		这是一个方程束,对于一个特定方程来说,$m$是常数, 称为连带勒让德(Legendre)方程。\\
		做微分展开
		\begin{equation*}
			\frac{d^{2} \Theta}{d \theta^{2}}+\frac{\cos \theta}{\sin \theta} \frac{d \Theta}{d \theta}+\left[\lambda-\frac{m^{2}}{\sin ^{2} \theta}\right] \Theta=0
		\end{equation*}		
		令$x=\cos \theta$, 有 $$y(x)= y(\cos \theta) =\Theta (\theta)$$ 
		对$x=\cos \theta$ 求导
		\begin{equation*}
			\frac{d x}{d  \theta} =-\sin \theta  
		\end{equation*}	
		链式求导	
		\begin{equation*}
			\frac{d \Theta}{d \theta} =\frac{d y}{d x}\frac{d x}{d \theta} =-\sin \theta \frac{d y}{d x}
		\end{equation*}	
		二阶导	
		\begin{equation*}
			\frac{ d^2 \Theta }{d \theta ^2} =\sin ^2 \theta \frac{d^2 y}{d x^2} -\cos \theta \frac{d y}{d x}
		\end{equation*}		
		代回, 得
		\begin{equation*}
			\sin  \theta\frac{d^{2} y}{d x^{2}}-2 x \frac{d y}{d x}+\left[\lambda-\frac{m^{2}}{\sin^2 \theta}\right] y=0, \quad (|x|\le 1)
		\end{equation*}	
		注意到$\cos\theta =x,~ \sin^2  \theta =1-x^2 $ \\
		得连带勒让德方程的简洁形式
		\begin{equation}
			\boxed{\left(1-x^{2}\right) \frac{d^{2} y}{d x^{2}}-2 x \frac{d y}{d x}+\left[\lambda-\frac{m^{2}}{1-x^{2}}\right] y=0, \quad (|x|\le 1)} 
		\end{equation}
		解纬度方程转化为解连带勒让德方程! 
		
	~~\\ 
	
	
	\subsection{解勒让德方程}
	首先考虑最简单的情况,取$m=0$, 得勒让德方程
		\begin{equation}
			\boxed{\left(1-x^{2}\right) \frac{d^{2} y}{d x^{2}}-2 x \frac{d y}{d x}+\lambda y=0} 
		\end{equation}
	\emph{解:}	
		这是个二阶变系数常微分方程,令方程有级数解,
		\[ y=\sum_{k=0}^{\infty} a_k x ^k \]
		对级数求导
	$$\begin{cases}
			y' = \sum\limits_{k=1}^{\infty} k a_k x^{k-1} \\
			xy'= \sum\limits_{k=0}^{\infty} k a_k x^{k}
			\\
			y'' = \sum\limits_{k=2}^{\infty} k (k-1) a_k x^{k-2} =  \sum\limits_{k=0}^{\infty} (k+2) (k+1) a_{k+2} x^k \\
			x^{2}y'' =  \sum\limits_{k=0}^{\infty} k (k-1) a_k x^{k}
		\end{cases}$$
		代回方程,注意指标对齐,得:
		\[ \sum\limits_{k=0}^{\infty} (k+2) (k+1) a_{k+2} x^k - \sum\limits_{k=0}^{\infty} k (k-1) a_k x^{k} -2\sum\limits_{k=0}^{\infty} k a_k x^{k} + \lambda \sum_{k=0}^{\infty} a_k x ^k \]
		整理
		\begin{equation*}
			\sum_{k=0}^{\infty}\left\{(k+1)(k+2) a_{k+2}+[\lambda-k(k+1)] a_{k}\right\} x^{k}=0
		\end{equation*}	
		多项式为零,要求各项系数为零, 有
		\begin{equation*}
			(k+1)(k+2) a_{k+2}+[\lambda-k(k+1)] a_{k}=0
		\end{equation*}	
		得递推式
		\begin{equation} \label{eq:dts}
			a_{k+2}=-\frac{k(k+1)-\lambda}{(k+1)(k+2) }a_{k}
		\end{equation}
		这是隔项递推,若 $a_0$ 已知,则所有的偶数次幂的系数 $a_{2m}$ 可得, 同理若 $a_1$ 已知,则所有的奇数次幂的系数 $a_{2m+1}$ 可得。
		偶数项写在一起
		\begin{equation*}
			y_{1}(x)= a_{0} + a_{2} x^2 + a_{2} x^4 +...  
		\end{equation*}	
		奇数项写在一起
		\begin{equation*}
			y_{2}(x)= a_{1} x+ a_{3} x^3 + a_{5} x^5 +... 
		\end{equation*}
		方程的级数解为\[ y(x)=  y_{1}(x) +  y_{2}(x) \]
		现在考虑级数解的收敛性问题, 由式[\ref{eq:dts}], 得
		\[ \frac{a_{k+2}}{a_{k}} = -\frac{k(k+1)-\lambda}{(k+1)(k+2) }\]
		当 $k \to \infty$时,有
		\[ \lim_{k \to \infty}|\frac{a_{k+2}}{a_{k}}| \approx 1-\frac{2}{k}\]
		对于偶数项
		\[ \lim_{k \to \infty} |\frac{a_{k+2}}{a_{k}}| \approx 1-\frac{2}{2m} = 1-\frac{1}{m} \]
		对于函数$\ln(1+x)+\ln(1-x) = \ln (1-x^2)$在$x=\pm 1$的泰勒展开式,相邻项展开系数有
		\[ \lim_{k \to \infty} \frac{a_{k+1}}{a_{k}} = 1-\frac{1}{k} \]
		我们知道$\ln (1-x^2)$在$x=\pm 1$处发散, 因此,$$ \lim_{\left \vert x \right \vert \to 1} y_{1}(x) \to \infty $$
		同理,$$ \lim_{\left \vert x \right \vert \to 1} y_{2}(x) \to \infty $$
		即:级数解 $ y(x)=  y_{1}(x) +  y_{2}(x) $ 在 $x=\pm 1$处发散。
		~~\\
		考察式[\ref{eq:dts}], 如果设
		\[\lambda = l(l+1), (l=0,1,2,\cdots )\]
		则有新的递推式
		\begin{equation}\label{eq:dts2}
			a_{k+2}=-\frac{k(k+1)-l(l+1)}{(k+1)(k+2) }a_{k}
		\end{equation}
		考虑到$l$对应角向(方)算符本征方程
		\begin{equation*}
			L^2 Y=\lambda Y
		\end{equation*}	
		的本征值$\lambda$, 对于一个确定的本征值$\lambda$来说, $l$有确定值。新的递推式[\ref{eq:dts2}]表明:对于确定的$l$来说, 当$k$增长到 $k=l$, 有$a_{k+2} =0$, 级数截断成$l$次多项式。
		\begin{itemize}
			\item $k = 2m, y_1 \text{截断}(y_2\text{仍当无穷级数})$
			\item $k = 2m+1, y_2 \text{截断}(y_1\text{仍当无穷级数})$
		\end{itemize}
		即, 只有通过设定$\lambda = l(l+1)$, 才可以得到一个在$[-1, 1]$收敛的多项式, 它是符合条件的物理解。\\
	~~\\
		为了明确这个多项式解的具体形式,把递推式[\ref{eq:dts2}]逆向
		\begin{equation*}
			a_{k}=-\frac{(k+1)(k+2) }{(l-k)(l+k+1)}a_{k+2}
		\end{equation*}
		递推两次,
		\begin{equation*}
			a_{k-2}=-\frac{(k-1)(k)}{(l-(k-2))(l+k-1)}a_{k}
		\end{equation*}	
		\begin{equation*}
			a_{k-4}=-\frac{(k-3)(k-2)}{(l-(k-4))(l+k-3)}a_{k-2}
		\end{equation*}
		代入$a_{k-2}$,得
		\begin{equation*}
			a_{k-4}=(-1)^2\frac{(k-3)(k-2)(k-1)(k)}{(l-(k-4))(l-(k-2))(l+k-3)(l+k-1)}a_{k}
		\end{equation*}		
		注意到最高项为 $k=l$,为了直观起见,令最高项为第$n$项, 即 $n=k=l$, 代入,得
		\begin{equation*}
			a_{n-4}=(-1)^2\frac{(n-3)(n-2)(n-1)(n)}{(2\times 2)(2\times 1)(2n-3)(2n-1)}a_{n}
		\end{equation*}	
		依次递推, 则每一项的系数都可以用最高项的系数$a_{n}$描述,
		现令最高项系数为 \[a_n=\frac{(2n)!}{2^n (n!)^2} = \frac{2n (2n-1)(2n-2)(2n-3)(2n-4)!}{2^n \times [n (n-1) (n-2)!]\times [n (n-1)  (n-2) (n-3) (n-4)!] } \]
		代回上式并约分,得:
		\begin{equation*}
			a_{n-4}=(-1)^2\frac{(2n-4)!}{2! 2^n (n-2)! (n-4)!}
		\end{equation*} 
		为了方便递推, 改写成
		\begin{equation*}
			a_{n-2\times 2}=(-1)^2\frac{(2 n-2\times 2) !}{2!2^{n}  (n-2) !(n-2\times 2) !}
		\end{equation*}	
		再递推一次
		\begin{equation*}
			a_{n-2\times 3}=(-1)^3\frac{(2 n-2\times 3) !}{3!2^{n}  (n-3) !(n-2\times 3) !}
		\end{equation*}	
		写出一般式
		\begin{equation*}
			a_{n-2 \times m}=(-1)^{m} \frac{(2 n-2 m) !}{m ! 2^{n} (n-m) !(n-2 m) !}
		\end{equation*}	
		得勒让德方程的解
		\begin{equation*}
			y(x)=\sum_{m=0}^{[n/2]}a_{n-2m} x^{n-2m} =\sum_{m=0}^{[n / 2]}(-1)^{m} \frac{(2 n-2 m) !}{2^{n} m !(n-m) !(n-2 m) !} x^{n-2 m} =P_{n}(x)
		\end{equation*}	
		称为勒让德多项式。 注意当$m=0$时, 对应最高项, 当$m=[n/2]$时对应最低项,有
		\[
		n-2m = n-2\times[n/2] =
		\begin{cases}
			0 \quad (\text{n 为偶数})\\
			1 \quad (\text{n 为奇数})
		\end{cases}
		\]
		由于$n$只是暂时借用, 真正决定最高项的是$l$,勒让德多项式应记为
		\begin{equation}\label{eq:Pl}
			\boxed{P_{l}(x)=\sum_{m=0}^{[l / 2]}(-1)^{m} \frac{(2 l-2 m) !}{2^{l} m !(l-m) !(l-2 m) !} x^{l-2 m}} 
		\end{equation}
	
	~~\\ 
	
	 \begin{example}
	 求前五个勒让德多项式
	 \end{example}
	\emph{解:}
		在勒让德多项式表达式[\ref{eq:Pl}] 中, \\ 
	(1) 取 $l=0$, 则$m=0$, 代入上式, 得 $P_{0}(x) =1$ \\  
	(2) 取 $l=1$, 则$m=0$, 代入上式, 得 $P_{1}(x) =x$\\
	(3) 取 $l=2$, 则$m=0,1$, 代入上式,得 
	$ P_{2}(x) = \dfrac{1}{2}\left(3 x^{2}-1\right) $\\
	(4) 取 $l=3$, 则$m=0,1$, 代入上式,得 
	$ P_{3}(x) = \dfrac{1}{2}\left(5 x^{3}-3 x\right)  $ \\
	(5) 取 $l=4$, 则$m=0,1,2$, 代入上式,得 
	$$ P_{4}(x) = \dfrac{1}{8}\left(35 x^{4}-30 x^{2}+3\right)  $$
	列表如下,取 $x=\cos\theta$,则得右列
	$$\renewcommand\arraystretch{1.6}
	\begin{array}{|l|l|}
		\hline P_{0}(x)=1 & P_{0}(\cos\theta)=1\\
		\hline P_{1}(x)=x & P_{1}(\cos \theta)=\cos \theta \\
		\hline P_{2}(x)=\dfrac{1}{2}\left(3 x^{2}-1\right) & P_{2}(\cos \theta)=\dfrac{1}{4}[3 \cos 2 \theta+1] \\ 
		\hline P_{3}(x)=\dfrac{1}{2}\left(5 x^{3}-3 x\right) & P_{3}(\cos \theta)=\dfrac{1}{8}[5 \cos 3 \theta+3 \cos \theta] \\
		\hline P_{4}(x)=\dfrac{1}{8}\left(35 x^{4}-30 x^{2}+3\right) & P_{4}(\cos \theta)=\dfrac{1}{64}[35 \cos 4 \theta+20 \cos 2 \theta+9] \\
		\hline P_{5}(x)=\dfrac{1}{8}\left(63 x^{5}-70 x^{3}+15 x\right) & \cdots  \\
		\hline P_{6}(x)=\dfrac{1}{16}\left(231 x^{6}-315 x^{4}+105 x^2 -5\right) & \cdots  \\
		\hline
	\end{array}$$
	
	~~\\ 
	图[\ref{fig:p012345}]给出前六个勒让德多项式的图形,很明显,$P_{l}(x)$具有$l$的奇偶性。
	
	\begin{figure}[h]
		\centering
		\includegraphics[width=0.98\textwidth]{figs/p012345.png}
		\caption{勒让德多项式图}
		\label{fig:p012345}
	\end{figure}
	
	
	~~\\ 
	\begin{example}
		解连带勒让德方程
		\begin{equation*}
			\left(1-x^{2}\right) \frac{d^{2} y}{d x^{2}}-2 x \frac{d y}{d x}+\left[l(l+1)-\frac{m^{2}}{1-x^{2}}\right] y=0, \quad (|x|\le 1) 
		\end{equation*}
	\end{example}
	\emph{解:}
	勒让德多项式满足勒让德方程
	\[\left(1-x^{2}\right) P'' _l  (x) -2 x P' _l (x)+l(l+1)P_l(x)=0\]
	求导
	\[\left[-2x P'' _l  (x) + \left(1-x^{2}\right) P''' _l  (x)\right] - \left[2 P' _l (x) +2 x P'' _l (x) \right] + l(l+1)P'_l(x)=0 \]
	整理成如下形式
	\[\left(1-x^{2}\right) P^{(3)} _l (x) -2(1+1) x P'' _l (x)+\left[l(l+1)-1(1+1)\right]P'_l(x)=0	\]
	再求导
	\[ \begin{aligned}
		0=&\left[-2x P^{(3)}_l (x) + \left(1-x^{2}\right) P^{(4)} _l (x)\right] \\
		&- \left[2(1+1) P'' _l (x) +2(1+1) x P^{(3)} _l (x)\right] + \left[l(l+1)-1(1+1) \right] P''_l(x)\\ 
	\end{aligned}
	 \]
	整理,得
	\[\left(1-x^{2}\right) P^{(4)} _l  (x) -2(2+1) x P^{(3)} _l (x)+(l(l+1)-2(2+1)P''_l(x)=0\]
	经$m$次求导,得
	\begin{equation}\label{eq:lege}
			\left(1-x^{2}\right) P^{(m+2)} _l  (x) -2(m+1) x P^{(m+1)} _l (x)+\left[l(l+1)-m(m+1)\right]P^{(m)} _l(x)=0
	\end{equation}
	令 $$ \boxed{P^{m} _l(x)=(1-x^2)^{m/2}P^{(m)} _l(x), \quad 0\le m\le l, l=1,2,3,\cdots} $$
	%并对其求导
	%\[
	%\begin{aligned}
	%	\frac{d}{d x} P^m_{l}(x) &= -mx P^{(m)} _l(x) + (1-x^2)^{\frac{m}{2}}P^{(m+1)} _l(x) \\ 
	%	\frac{d^2}{d x^2} P^m_{l}(x) &= -m P^{(m)} _l(x) -mx P^{(m+1)} _l(x) -mx P^{(m+1)} _l(x) + (1-x^2)^{\frac{m}{2}}P^{(m+2)} _l(x) \\ 
	%\end{aligned}\]
	代回方程[\ref{eq:lege}],得
	\begin{equation*}
		\left(1-x^{2}\right) \frac{d^{2}}{d x^{2}} P^m_{l}(x) -2 x \frac{d}{d x} P^m_{l}(x)+\left[l(l+1)-\frac{m^{2}}{1-x^{2}}\right] P^m_{l}(x)=0
	\end{equation*}		
	$P^m _{l}(x)$满足连带勒让德方程,即$P^m _{l}(x)$是连带勒让德方程的解, 称为连带勒让德多项式, 其中$m$称为连带勒让德多项式的阶, $l$称为连带勒让德多项式的自由度。
	
	~~\\ 
	前几个连带勒让德多项式,具体如下表。很明显,$P^m_{l}(x)$具有$l+m$的奇偶性 
	$$\renewcommand\arraystretch{1.6}
	\begin{array}{|l|l|}
		\hline P^0_{0}(x)=1 & P^2_{2}(x)=3\left(1-x^2\right)\\
		\hline P^0_{1}(x)=x & P^0_{3}(x)=\dfrac{1}{2}x(5x^2-3) \\
		\hline P^1_{1}(x)=-(1-x^2)^{1/2} & P^1_{3}(x)=\dfrac{3}{2}(1-5x^2)(1-x^2)^{1/2} \\
		\hline P^0_{2}(x)= \dfrac{1}{2}(3x^2-1) & P^2_{3}(x)=15 x(1-x^2)\\ 
		\hline P^1_{2}(x)=- 3x\left(1-x\right)^{1/2}  &P^3_{3}(x)= -15 x(1-x^2)^{3/2} \\
		\hline
	\end{array}$$
	~~\\ 
	
	
	\subsection{勒让德多项式的性质}
	
	\begin{proposition}-1:
		勒让德多项式有如下微分形式:
		\begin{equation*}
			P_{n}(x)=\frac{1}{2^{n} n !} \frac{d^{n}}{d x^{n}}\left(x^{2}-1\right)^{n}, \quad(n=0,1,2,3, \cdots \cdots)
		\end{equation*}
	\end{proposition}
	\begin{proof}
		由二项式定理,有:
		$$\begin{aligned}
			\left(x^{2}-1\right)^{n}&=\sum_{m=0}^{n} C_{n}^{m}(-1)^{m}\left(x^{2}\right)^{n-m}\\
			&=\sum_{m=0}^{n} \frac{(-1)^{m} n !}{m !(n-m) !} x^{2 n-2 m}
		\end{aligned}$$
		求n次导
		\begin{equation*}
			\frac{d^{n}}{d x^{n}}\left(x^{2}-1\right)^{n}=\sum_{m=0}^{n} \frac{(-1)^{m} n !}{m !(n-m) !} \frac{d^{n}}{d x^{n}}\left(x^{2 n-2 m}\right)
		\end{equation*}	
		当$2n-2m<n$时, 上式右边的导数为零,即非零项应要求
		\[
			\begin{aligned}
				2n-2m \ge n \qquad \to \qquad 
				2m \le n 
			\end{aligned}
		\]
		即非零的最高次幂为 $m=[n/2]$
		\begin{align*}
			\frac{d^{n}}{d x^{n}}\left(x^{2}-1\right)^{n}&=\sum_{m=0}^{[n/2]} \frac{(-1)^{m} n !}{m !(n-m) !} \frac{d^{n}}{d x^{n}}\left(x^{2 n-2 m}\right)\\
			&=\sum_{m=0}^{[n / 2]}(-1)^{m} \frac{(2 n-2 m) ! n!}{ m !(n-m) !(n-2 m) !} x^{n-2 m}
		\end{align*}	
		两端同除以 $2^{n} n !$
		\begin{align*}
			\frac{1}{2^{n} n !} \frac{d^{n}}{d x^{n}}\left(x^{2}-1\right)^{n} &= \sum_{m=0}^{[n / 2]}(-1)^{m} \frac{(2 n-2 m) !}{2^{n} m !(n-m) !(n-2 m) !} x^{n-2 m}\\
			&=P_n(x)\\
		\end{align*}
	\textcolor{red}{证毕!}
	\end{proof}
	
	~~\\ 
	
	\begin{proposition}-2:勒让德多项式有母函数
		\begin{equation*}
			w(x, z)=\left(1-2 z x+z^{2}\right)^{-1 / 2}
		\end{equation*}	
	\end{proposition}
	\begin{proof}
		即要证函数展开式与勒让德多项式有如下关系
		\begin{equation*}
			\left(1-2 z x+z^{2}\right)^{-1 / 2}=\sum_{n=0}^{\infty} P_n(x) z^n
		\end{equation*}	
		二项式定理
		\begin{equation*}
			(1+v)^{p}=\sum_{k=0}^{\infty} \frac{p(p-1) \cdots(p-k+1)}{k !} v^{k}
		\end{equation*}	
		取$p=-1/2$, 得:
		\begin{align*}
			(1+v)^{-1/2}&=\sum_{k=0}^{\infty} (-1)^{k}\frac{\frac{1}{2}\frac{3}{2}  \cdots \frac{2k-1}{2}}{k !} v^{k}\\
			&=\sum_{k=0}^{\infty} (-1)^{k}\frac{\frac{1}{2}\frac{2}{2}\frac{3}{2} \frac{4}{2}  \cdots \frac{2k-1}{2}\frac{2k}{2}} {(k !)^2} v^{k}\\
			&=\sum_{k=0}^{\infty}(-1)^{k} \frac{(2 k) !}{2^{2 k}(k !)^{2}} v^{k}
		\end{align*}	
		取$$v=-2zx+z^2=-z(2x-z)$$ 
		\begin{equation*}
			v^{k}=(-1)^{k} z^{k}(2 x-z)^{k}=(-1)^{k} z^{k} \sum_{m=0}^{k} C_{k}^{m}(2 x)^{k-m}(-z)^{m}
		\end{equation*}	
		代回
		\begin{equation*}
		\begin{aligned}
		\left(1-2 z x+z^{2}\right)^{-{1}/{2}} &= \sum_{k=0}^{\infty}(-1)^{k} \frac{(2 k) !}{2^{2 k}(k !)^{2}} (-1)^{k} z^{k} \sum_{m=0}^{k} C_{k}^{m}(2 x)^{k-m}(-z)^{m}\\
		&=\sum_{k=0}^{\infty} \frac{(2 k) !}{2^{2 k}(k !)^{2}} \sum_{m=0}^{k}(-1)^{m} C_{k}^{m}(2 x)^{k-m} z^{k+m}
		\end{aligned}
		\end{equation*}	
		令$k+m=n$,
		\begin{equation*}
		\begin{split}
			\left(1-2 z x+z^{2}\right)^{-{1}/{2}} &= \sum_{k=0}^{\infty} \frac{(2 k) !}{2^{2 k}(k !)^{2}} \sum_{m=0}^{k}(-1)^{m} C_{k}^{m}(2 x)^{k-m} z^{n} \\
			&= \sum_{n=0}^{\infty}\sum_{k+m=n, m\le k} \frac{(2 k) !}{2^{2 k}(k !)^{2}}  (-1)^{m} C_{k}^{m}(2 x)^{k-m} z^{n}
		\end{split}
		\end{equation*}
		去除指标$k$, 因为$k+m=n$, 有 $k=n-m$, 有$k-m = n-2m $,代入上式
		\begin{equation*}
			\begin{split}
			\left(1-2 z x+z^{2}\right)^{-{1}/{2}}&=\sum_{n=0}^{\infty}\sum_{m=0}^{n} \frac{(2 (n-m)) !}{2^{2 (n-m)}((n-m) !)^{2}}(-1)^{m} C_{n-m}^{m}(2 x)^{n-2m} z^{n}\\
			&=\sum_{n=0}^{\infty}\sum_{m=0}^{n}(-1)^{m} \frac{(2 n-2m) !}{2^{2 (n-m)}((n-m) !)^{2}} \frac{(n-m) !}{m!(n-2 m) !}2^{n-2 m}x^{n-2 m} z^{n}\\
			&=\sum_{n=0}^{\infty}\sum_{m=0}^{n}(-1)^{m} \frac{(2 n-2 m) !}{2^{n} m !(n-m) !(n-2 m) !} x^{n-2 m} z^{n}
		\end{split}
		\end{equation*}	
		因此,
		$$
		\left(1-2 z x+z^{2}\right)^{-{1}/{2}}=\sum_{n=0}^{\infty}P_{n}(x) z^{n}  
		$$ 
	\textcolor{red}{证毕!}
	\end{proof}
	
	~~\\ 
	\begin{hint}
		母函数的物理意义: \\
		在母函数中
		$$
		\left(1-2 z x+z^{2}\right)^{-{1}/{2}}=\sum_{n=0}^{\infty}P_{n}(x) z^{n}  
		$$
		取$z=\dfrac{r}{R}, x = \cos \theta, n=l $, 得
		$$
		\left(1-2 \frac{r}{R} \cos \theta+\frac{r^2}{R^2}\right)^{-{1}/{2}}=\sum_{l=0}^{\infty}P_{l}(\cos \theta) (\frac{r}{R})^{l}  
		$$
		两端同除以$\dfrac{1}{R}$, 有
		\[\begin{aligned}
			\frac{1}{\sqrt{r^2+ R^2 -2rR\cos \theta} } =\frac{1}{R} \sum_{l=0}^{\infty}P_{l}(\cos \theta) (\frac{r}{R})^{l}  \\ 
			\frac{1}{\sqrt{\vec{r}^2+ \vec{R}^2 -2\vec{r}\cdot\vec{R} }} =\frac{1}{R} \sum_{l=0}^{\infty}P_{l}(\cos \theta) (\frac{r}{R})^{l}  \\
			\frac{1}{\left\vert\vec{r}-\vec{R} \right\vert} =\frac{1}{R} \sum_{l=0}^{\infty}P_{l}(\cos \theta) (\frac{r}{R})^{l} 
		\end{aligned} \]
		考虑泰勒展开, 有
		\[\begin{aligned}
			\frac{1}{\left\vert\vec{r}-\vec{R} \right\vert} = \frac{1}{\left\vert r- R \right\vert} = \frac{1}{R} \sum_{l=0}^{\infty}a_l(\frac{r}{R})^{l} 
		\end{aligned} \]
		得
		\[ a_l = P_l(\cos \theta), \qquad (r^l\text{部分:见径向解} )\]
		勒让德多项式是一个物理量对球心的泰勒展开下的各阶系数多项式。
	\end{hint}
	
	\begin{proposition} 勒让德多项式递推关系
		\begin{equation*}
			(n+1) P_{n+1}(x)-(2 n+1) x P_{n}(x)+n P_{n-1}(x)=0
		\end{equation*}	
	\end{proposition}
		\emph{解:}
			存在母函数
			\begin{equation*}
				w(x, z)=(1-2zx+z^2)^{-1/2}=\sum_{n=0}^{\infty} P_{n}(x) z^{n}
			\end{equation*}	
			对形式级数求关于z的偏导:
			\begin{equation*}
				\frac{\partial w}{\partial z}=\sum_{n=1}^{\infty} n P_{n}(x) z^{n-1}=\sum_{n=0}^{\infty}(n+1) P_{n+1} z^{n}
			\end{equation*}	
			对函数求关于z的偏导:
			\begin{equation*}
				\frac{\partial w}{\partial z}=	(x-z)(1-2zx+z^2)^{-3/2}
			\end{equation*}		
			\begin{equation*}
				(1-2zx+z^2)\frac{\partial w}{\partial z}=(x-z)(1-2zx+z^2)^{-1/2}
			\end{equation*}		
			\begin{equation*}
				(1-2zx+z^2)\frac{\partial w}{\partial z}-(x-z)w=0
			\end{equation*}	
			代入级数及级数偏导	
			\begin{equation*}
				(1-2zx+z^2)\sum_{n=0}^{\infty}(n+1) P_{n+1} z^{n}-(x-z)\sum_{n=0}^{\infty} P_{n}(x) z^{n}=0
			\end{equation*}		
			整理
			\begin{equation*}
				\sum_{n=1}^{\infty} [(n+1)P_{n+1} -(2n+1)x P_n + nP_{n-1} ] z^{n}=0
			\end{equation*}		
			由多项式系数项为零\\ \textcolor{red}{得证!}
	
		~~\\ 
	
	\begin{proposition}勒让德多项式的正交归一性
		\[ \frac{2n+1}{2} \int\limits_{-1}^{+1}  P_m(x) P_n(x)dx = \left\{ 
			\begin{aligned}
				& 0, \quad (m\ne n) \\ 
				& 1 , \quad (m= n)
			\end{aligned}\right.\]
		\end{proposition}
	\begin{proof}
		(I)正交性:勒让德多项式满足勒让德方程($n=l$)
		\begin{equation*}
			\left(1-x^{2}\right) P'' _n  (x) -2 x P' _n (x)+n(n+1)P_n(x)=0
		\end{equation*}		
		等价形式
		\begin{equation*}
			[\left(1-x^{2}\right) P' _n  (x)]' +n(n+1)P_n(x)=0    \cdots  (1)
		\end{equation*}		
		同理:
		\begin{equation*}
			[\left(1-x^{2}\right) P' _m  (x)]' + m (m+1)P_m(x)=0    \cdots  (2)
		\end{equation*}		
		(1)$\times P_m$-(2)$\times P_n$
		\[ [n(n+1) -m (m+1)] P_m P_n = P_n[\left(1-x^{2}\right) P' _m  (x)]' - P_m[\left(1-x^{2}\right) P' _n  (x)]' \]
		积分 
		\begin{equation*}
				[n(n+1) -m (m+1)]\int_{-1}^{1} P_mP_n dx =\int_{-1}^{1} \left\{P_n [\left(1-x^{2}\right) P' _m] '-P_m [\left(1-x^{2}\right) P' _n ]'\right\}dx
		\end{equation*}
		右端分部积分
		\begin{equation*}
		\begin{split}
			= &\left.\left\{P_n [\left(1-x^{2}\right) P' _m]-P_m [\left(1-x^{2}\right) P' _n ]\right\} \right| _{-1} ^{1}  
			-\int_{-1}^{1}  [\left(1-x^{2}\right) P' _nP' _m -\left(1-x^{2}\right) P' _mP' _n  ]  dx \\
			=&0
		\end{split}
		\end{equation*}		
		因此,式子的左端也为零
		\begin{equation*}
			[n(n+1) -m (m+1)]\int_{-1}^{1} P_mP_n dx =0
		\end{equation*}	
		由于$ n\ne m  $ 
		\begin{equation*}
			\int_{-1}^{1} P_mP_n dx =0 ,\cdots (n\ne m)
		\end{equation*}
		(II)归一性:由递推公式
		\begin{equation*}
			nP_{n} -(2n-1)x P_{n-1} + (n-1)P_{n-2}  =0
		\end{equation*}		
		\begin{equation*}
			nP ^2 _{n} =(2n-1)x P_n P_{n-1} - (n-1)P_nP_{n-2} 
		\end{equation*}		
		\begin{equation*}
			\int_{-1}^{1}  P ^2 _{n} dx = \frac{(2n-1)}{n} \int_{-1}^{1}  x P_n P_{n-1} dx , \cdots (1)
		\end{equation*}	
		递推式可写成
		\begin{equation*}
			(n+1)P_{n+1} -(2n+1)x P_{n} + nP_{n-1}  =0  
		\end{equation*}		
		\begin{equation*}	
			x P_{n}=\frac{n+1}{2n+1}P_{n+1} + \frac{n}{2n+1}P_{n-1} , \cdots (2)
		\end{equation*}	
		把(2)代入(1)式,得积分递推式
		\begin{align*}
			\int_{-1}^{1}  P ^2 _{n} dx &=  \frac{2n-1}{2n+1}\int_{-1}^{1}   P^2_{n-1} dx \\
			&=  \frac{2n-1}{2n+1} \cdot \frac{2(n-1)-1}{2(n-1)+1} \int_{-1}^{1}   P^2_{n-2} dx \\
		\end{align*}		
		反复递推
		\begin{equation*}
			\int_{-1}^{1}  P ^2 _{n} dx =  \frac{1}{2n+1}\int_{-1}^{1}   P^2_{0} dx = \frac{2}{2n+1}
		\end{equation*}	
	\textcolor{red}{证毕!}
	\end{proof}
	~~\\ 
	
	\begin{example}
	计算积分
		\begin{equation*}
			\int_{-1}^{+1} x^2 P _{n}(x) dx 
		\end{equation*}	
	\end{example}
	 
	\emph{解:}
		由于
		\begin{equation*}
			P_0(x)=1, \quad	P_1(x)=x, \quad P_2(x)= \dfrac{1}{2}(3x^2-1)  
		\end{equation*}	
		$x^2$的勒让德多项式只有三项
		\[
			\begin{aligned}
				&x^2 = a_0P_0(x) + a_1P_1(x) + a_2P_2(x) \\
				&a_0 + a_1 x + a_2 \dfrac{1}{2}(3x^2-1) - x^2 =0 \\
				& ( \dfrac{3a_2}{2} -1)x^2 + a_1 x + (a_0 -\dfrac{a_2}{2} ) =0
			\end{aligned} 
		\]
		由多项式各项系数为零,得 
		\begin{equation*}
			a_0 =\dfrac{1}{3}, \quad	a_1 =0, \quad a_2 =\dfrac{2}{3}
		\end{equation*}	
		$x^2$的展开式
		$$ x^2 =\dfrac{2}{3}P_2+\dfrac{1}{3}P_0$$
		原式为:
		\begin{equation*}
		\begin{aligned}
			\int_{-1}^{+1} x^2 P _{n} dx &= \int_{-1}^{+1} (\dfrac{2}{3}P_2+\dfrac{1}{3}P_0)  P _{n} dx \\	  
			&= \int_{-1}^{+1} \dfrac{2}{3}P_2 P _{n} dx + \int_{-1}^{+1} \dfrac{1}{3}P_0 P _{n} dx
		\end{aligned}  	
		\end{equation*}	
		分情况讨论:\\
		(1)  $n=0$, 
		\begin{equation*}
			\int_{-1}^{+1} x^2 P _{n} dx  =   \int_{-1}^{+1} \dfrac{1}{3}P_0  P _{0} dx=\dfrac{1}{3} \frac{2}{2n+1}  = \dfrac{2}{3}	
		\end{equation*}	
		(2)  $n=2$, 
		\begin{equation*}
			\int_{-1}^{+1} x^2 P _{n} dx  =   \int_{-1}^{+1} \dfrac{2}{3}P_2  P _{2} dx= \dfrac{2}{3}\frac{2}{2n+1}  = \dfrac{4}{15}	
		\end{equation*}	
		(3) $ n\neq 0,2$
		\begin{equation*}
			\int_{-1}^{+1} x^2 P _{n} dx  =  0	
		\end{equation*}
	
	~~\\ 
	
	
	\subsection{连带勒让德多项式的性质}
	~\\
	连带勒让德多项式具有以下性质\\
	(1) 正交归一性:
	$$
	\int_{-1}^1 P_l^m(x) P_{l^{\prime}}^m(x) d x=\frac{2}{2 l+1} \frac{(l+m) !}{(l-m) !} \delta_{l l^{\prime}}
	$$
	$$
	\int_{-1}^1 P_l^m(x) P_l^{m^{\prime}}(x) \frac{d x}{1-x^2}=\frac{(l+m) !}{m(l-m) !} \delta _{mm'}
	$$	
	(2) 递推式: 
	\begin{equation*}
		(l+1-m)P^m _{l+1} -(2l+1)x P^m _l + (l+m) P^m _{l-1} =0 
	\end{equation*}	
	~~\\ 
	
	\subsection{球谐函数}
	~~\\ 
	角向方程:
		\begin{equation*}
			\left[ \frac{1}{ \sin \theta  } \frac{\partial }{\partial \theta } (\sin \theta \frac{\partial }{\partial \theta } )
			+\frac{1}{ \sin^2 \theta  } \frac{\partial^2}{\partial\varphi ^2} +l(l+1)\right] Y=0
		\end{equation*}	
	解为球谐函数:
		\begin{equation*}
			Y_{lm}(\theta,\varphi)= \Theta_{lm}(\theta) \Phi_m (\varphi), \quad ( 0 \le \theta \le \pi, 0 \le \varphi\le 2\pi)
		\end{equation*}	
	其中
		\begin{equation*} \left\{
			\begin{aligned}
				\Phi_m (\varphi) &= e^{im\varphi}, \qquad (m=0,\pm 1, \pm 2, \cdots \pm l) \\ 
				\Theta_{lm}(\theta)&= P^m  _{l}(\cos \theta), \qquad (l=0,1,2,3, \cdots (n-1)) 
			\end{aligned}\right.
		\end{equation*}
	前几个球谐函数的空间图像给出在图[\ref{fig:ylm}]中。
	
		\begin{figure}[h]
			\centering
			\includegraphics[width=0.98\textwidth]{figs/ylm.png}
			\caption{球谐函数的形态}
			\label{fig:ylm}
		\end{figure}
	
	~~\\ 
	\begin{example}
		求归一化的球谐函数
	\end{example}
	 \emph{解:}
	 设归一化的球谐函数为
	 \begin{equation*}
		Y_{lm} (\theta,\varphi)= A_{lm}  P_l ^m (\cos \theta)  \Phi_m (\varphi)
	\end{equation*}
	代入归一化公式
		\begin{equation*}
		\begin{split}
			 \iint  |Y_{lm}| ^2  d \sigma & =1  \\
			 \iint  A^2_{lm}  |P_l ^m (cos \theta)|^2  |\Phi (\varphi)|^2 d \sigma  & =1  \\
		 A^2_{lm}\int_{-1}^{+1} |P_l ^m (x)|^2 d x \int_{0}^{2\pi} |\Phi _m (\varphi)|^2 d \varphi  &= 1 \\
		A^2_{lm} 2\pi  \int_{0}^{\pi}    |P_l ^m (cos \theta)|^2  \sin \theta d\theta &=1 \\
		\end{split}		
		\end{equation*}	
	因此,有
	$$
		A^2_{lm}  2\pi  \frac{(l+m)!}{(l-m)!}  \frac{2}{2l+1}  =1   
	$$ 
		解得\[ A_{lm} = \sqrt{\frac{(2l+1)(l-m)!}{4\pi (l+m)!}}\]
	因此, 归一化的球谐函数为
	\begin{equation*}
		Y_{l}^m (\theta,\varphi)= \sqrt{\frac{(2l+1)(l-m)!}{4\pi (l+m)!}}  P_l ^m (cos \theta)  \Phi_m (\varphi)
	\end{equation*}
	归一化的经度函数
	\begin{equation*}
		\Phi_m (\varphi) = \frac{1}{\sqrt{2\pi}} e^{im\varphi}
	\end{equation*}
	归一化的纬度函数
	\begin{equation*}
		\Theta_{lm}(\theta) = B_{lm} P_l ^m (cos \theta)= \sqrt{\frac{(2l+1)(l-m)!}{2 (l+m)!}}  (1-x^2)^{\frac{m}{2}}P^{(m)} _l(x)
	\end{equation*}
	
	
	谐波是指频率为基波频率整数倍的波,比如琴弦的一维谐波,比如鼓面的二维谐波. 球谐函数描述的是单位球面上的三维谐波。每一个基波都有自己的谐波, 球谐函数$Y_{l} ^{m}$ 中, $l$ 描述一个基波有多少个谐波, 称为球谐函数的自由度, $m$ 为谐波的阶, 阶高的称为高阶谐波。具有如下三角形排布
	\renewcommand\arraystretch{1.6}
	\begin{table}[h]
		\centering
		\caption{球谐函数的三角形排布}
		%\label{tab:label}
		\begin{tabular}{cccccc}
				&$Y_{1} ^{0}$& $Y_{1} ^{1}$ & $\hspace{1.5em}$ & $\hspace{1.5em}$ & $\hspace{1.5em}$\\
				&$Y_{2} ^{0}$& $Y_{2} ^{1}$ & $Y_{2} ^{2}$ & $\hspace{1.5em}$ & $\hspace{1.5em}$\\
				&$Y_{3} ^{0}$& $Y_{3} ^{1}$ & $Y_{3} ^{2}$ & $Y_{3} ^{3}$ & $\hspace{1.5em}$\\
				&$Y_{4} ^{0}$& $Y_{4} ^{1}$ & $Y_{4} ^{2}$ & $Y_{4} ^{3}$ & $Y_{4} ^{4}$	
		\end{tabular}
		\end{table}	
	~~\\
	前几个复球谐函数
	$$
	\begin{aligned}
	Y_0^0 & =\frac{1}{2} \sqrt{\frac{1}{\pi}} \\
	Y_1^{-1} & =\frac{1}{2} \sqrt{\frac{3}{2 \pi}} \frac{(x-i y)}{r} \\
	Y_1^0 & =\frac{1}{2} \sqrt{\frac{3}{\pi}} \frac{z}{r} \\
	Y_1^1 & =-\frac{1}{2} \sqrt{\frac{3}{2 \pi}} \frac{(x+i y)}{r} \\
	Y_2^{-2} & =\frac{1}{4} \sqrt{\frac{15}{2 \pi}} \frac{(x-i y)^2}{r^2} \\
	Y_2^{-1} & =\frac{1}{2} \sqrt{\frac{15}{2 \pi}} \frac{z(x-i y)}{r^2} \\
	\end{aligned}
	$$
	$$
	\begin{aligned}
	Y_2^0 & =\frac{1}{4} \sqrt{\frac{5}{\pi}} \frac{\left(2 z^2-x^2-y^2\right)}{r^2} \\
	Y_2^1 & =-\frac{1}{2} \sqrt{\frac{15}{2 \pi}} \frac{z(x+i y)}{r^2} \\
	Y_2^2 & =\frac{1}{4} \sqrt{\frac{15}{2 \pi}} \frac{(x+i y)^2}{r^2}
	\end{aligned}
	$$
	取出经度函数的实部和虚部,球谐函数化为
		\[Y_{lm}=A_{lm} P^m _l(\cos\theta) \Phi_m (\varphi)=A_{lm}\begin{cases}
			 P^m _l(\cos\theta) \cos m \varphi\\
			 P^m _l(\cos\theta) \sin m \varphi\\
		\end{cases}\]
	当然,实球谐函数可写成
	$$
	Y_{\ell m}= \begin{cases}\dfrac{i}{\sqrt{2}}\left(Y_{\ell}^m-(-1)^m Y_{\ell}^{-m}\right), & \text { if } m<0 \\ Y_{\ell}^0, & \text { if } m=0 \\ \dfrac{i}{\sqrt{2}}\left(Y_{\ell}^{-m}+(-1)^m Y_{\ell}^m\right), & \text { if } m>0\end{cases}
	$$
	前几个实球谐函数为:
	$$
	\begin{aligned}
	Y_{1,-1} & =\sqrt{\frac{3}{4 \pi}} \frac{y}{r} \\
	Y_{1,0} & =\sqrt{\frac{3}{4 \pi}} \frac{z}{r} \\
	Y_{1,1} & =\sqrt{\frac{3}{4 \pi}} \frac{x}{r} \\
	Y_{2,-2} & =\frac{1}{2} \sqrt{\frac{15}{\pi}} \frac{x y}{r^2} \\
	Y_{2,-1} & =\frac{1}{2} \sqrt{\frac{15}{\pi}} \frac{y z}{r^2} \\
	Y_{2,0} & =\frac{1}{4} \sqrt{\frac{5}{\pi}} \frac{\left(2 z^2-x^2-y^2\right)}{r^2} \\
	Y_{2,1} & =\frac{1}{2} \sqrt{\frac{15}{\pi}} \frac{x z}{r^2} \\
	Y_{2,2} & =\frac{1}{4} \sqrt{\frac{15}{\pi}} \frac{\left(x^2-y^2\right)}{r^2}
	\end{aligned}
	$$
	~~\\ 
	小结:\\
	(1) 经度方程 
	\begin{equation*}
		\frac{d^{2} \Phi}{d \varphi^{2}}+\lambda' \Phi=0,\quad (0<\varphi\le2 \pi)
	\end{equation*}
	固有值与固有函数:
	   $$ \left\{
	   \begin{aligned}
		&\lambda' = m^2,\quad m = 0, \pm 1, \pm 2, \cdots , \pm l \\
		&\Phi_m (\varphi)=\frac{1}{\sqrt{2\pi}} e^{im\varphi}  
	   \end{aligned} \right. $$ 
	(2) 纬度方程:
	\begin{equation*}
		\frac{1}{\sin \theta} \frac{d}{d \theta}\left(\sin \theta \frac{d \Theta}{d \theta}\right)+\left[\lambda-\frac{m^{2}}{\sin ^{2} \theta}\right] \Theta=0, \quad (0< \theta \le \pi)
	\end{equation*}	
	固有值与固有函数: 
	   $$ \left\{
	   \begin{aligned}
		&\lambda = l (l+1), \quad l =0,1,2,\cdots \\
		&\Theta (\theta) = \sqrt{\frac{(2l+1)(l-m)!}{2 (l+m)!}}  P^m _{l}(\cos\theta)
	   \end{aligned} \right. $$ 
	(3) 角向方程:
	\begin{equation*}
		L^2 Y=\lambda Y
	\end{equation*}	
	固有值与固有函数(球谐函数): 
	$$ \left\{
		\begin{aligned}
		 &\lambda = l (l+1), \quad l =0,1,2,\cdots \\
		 &Y_{l}^m (\theta,\varphi)= \sqrt{\frac{(2l+1)(l-m)!}{4\pi (l+m)!}}  P_l ^m (cos \theta)  \Phi_m (\varphi)
		\end{aligned} \right. $$ 
	
	
	\subsection{解径向方程} 
	~\\
	把固在值$ \lambda =l(l+1)$代入径向方程
	\begin{equation*}
		\frac{d}{d r} (r^2\frac{d R }{d r} ) =\lambda R
	\end{equation*}	
	得
	\begin{equation*}
		\frac{d}{d r} (r^2\frac{d R }{d r} ) =l(l+1) R
	\end{equation*}
	整理	
	\begin{equation*}
		\frac{d^2 R}{d r^2} + \frac{2}{r}\frac{d R }{d r} - \frac{l(l+1)}{r^2} R=0
	\end{equation*}	
	欧拉型常微分方程,做变量代换
	\[ r = e^x\]
	方程化为 
	\[ R'' + R' -l(l+1)R =0\]
	通解为:
	\begin{equation*}
			R(x)=C_1 e ^{-(l+1)x}+C_2 e^ {lx} 
	\end{equation*}	
	代入 $x = \ln r$, 得
	\begin{equation*}
		R(r)=C_1 r ^{-(l+1)} + C_2 r ^ l 
    \end{equation*}	
   ~\\ 
	\begin{hint}
        最终,我们获得了球坐标系拉普拉斯方程的基本解
		\begin{equation}
			\begin{aligned}
				u_{ml} &= R(r)Y_{l} ^m (\theta,\varphi) \\
				  &= [C_{ml} r ^ l  + D_{ml} r ^{-(l+1)}] P_l ^m (cos \theta)  \Phi_m (\varphi)
			\end{aligned}
		\end{equation}
		叠加解
		\begin{equation}
				u=\sum_{l=0}^{\infty}\sum_{m=-l}^{m=l} u_{ml} 
		\end{equation}
		叠加解写成实函数形式,
		\begin{equation}
			\begin{aligned}
				u=\sum_{m, l} & \left(A_{m l} r^{l}+B_{m l} r^{(-l-1)}\right) P_{l}^{m}(\cos \theta)\operatorname{cos} m \varphi\\
					& + \sum_{m,l} \left(C_{m l} r^{l}+D_{m l} r^{(-l-1)}\right) P_{l}^{m}(\cos \theta)\operatorname{sin} m\varphi
			\end{aligned}
		\end{equation}
	\end{hint}

	\section{拉盖尔多项式}

	微分方程
	\begin{equation}
		x y''  + (m + 1 -x) y' +n y =0
	\end{equation}
	为连带拉盖尔方程,只有当n为非负时才有非平凡解。
	当$m =0$时称为拉盖尔方程。
	\begin{equation}
		x y''  + (1 -x) y' +n y =0
	\end{equation}
	由于$x_0=0$是方程的正则奇点,在$x_0=0$及其领域上有有限级数解
	\begin{equation}
		y(x)=a_{0}\left[1+\frac{-n}{(1!)^{2}} x+\frac{-n(1-n)}{(2!)^{2}} x^{2}+\cdots+\frac{-n(1-n) \cdots(k-1-n)}{(k!)^{2}} x^{k}+\cdots\right]
	\end{equation}
	当n为整数时,解退化为n次多项式
	\begin{equation}
		L_{n}(x)=\sum_{i=0}^{n}\binom{n}{i} \frac{(-1)^{i}}{i!} x^{i}
	\end{equation}
	称为拉盖尔多项式。下面给出获得过程及其性质。

	\subsection{解拉盖尔方程} 
		\begin{example}
		试求解拉盖尔方程
			\begin{equation*}
				x y''  + (1 -x) y' +n y =0
			\end{equation*}	 
		\end{example}
		\emph{解:}
		这是个变系数常微分方程, 考虑级数解, 令  
		\begin{equation*}
			y=\sum_{k=0}^{\infty} c_k x^k
		\end{equation*}	
		对级数求导
		$$
			\begin{aligned}
				y' &= \sum\limits_{k=1}^{\infty} k c_k x^{k-1} \\ 
					& =\sum\limits_{k=0}^{\infty} (k+1) c_{k+1} x^{k}\\
				xy' &= \sum\limits_{k=0}^{\infty} k c_k x^{k} \\
				y'' &= \sum\limits_{k=2}^{\infty} k (k-1) c_k x^{k-2} \\
				xy''	&=  \sum\limits_{k=1}^{\infty} k (k-1) c_k x^{k-1} 
			\end{aligned}
		$$
		\[		xy'' = \sum\limits_{k=0}^{\infty} (k+1) k c_{k+1} x^{k}\]
		代回方程 
		\[\sum\limits_{k=0}^{\infty} (k+1) k c_{k+1} x^{k} + \sum\limits_{k=0}^{\infty} (k+1) c_{k+1} x^{k} - \sum\limits_{k=0}^{\infty} k c_k x^{k} + \sum_{k=0}^{\infty} n c_k x^k =0\]
		整理,得
		\begin{equation*}
			\sum_{k=0}^{\infty} [(n-k)c_k +(k+1)^2 c_{k+1}  ] x^k =0
		\end{equation*}	
		多项式为零, 要求各阶系数为零
		\begin{equation*}
			c_{k+1}=-\frac{n-k}{(k+1)^2} c_k
		\end{equation*}
		这是递推式, 反复递推
		\begin{equation*}
			\begin{aligned}
				c_{1} &=-\frac{n-0}{(0+1)^2} c_0 \\ 
				c_{2} &=-\frac{n-1}{(1+1)^2} c_1 \\ 
				&= (-1)^2\frac{n(n-1)}{(2!)^2} c_0
			\end{aligned}
		\end{equation*}
		\begin{equation*}
			c_{k}=(-1)^k \frac{n(n-1)\cdots (n-(k-1))}{(k!)^2} c_0
		\end{equation*}	
		显然, 当$(k-1) = n$时, 分子为零, 即这不是一个无穷级数而是一个多项式, 称为拉盖尔多项式。\\ 
		当$k=n$时,最高阶项系数为
		\begin{equation*}
			c_{n}=(-1)^n \frac{1}{n!} c_0, 
		\end{equation*}	
		若取
		\begin{equation*}
			c_{0}=n!
		\end{equation*}	
		代回, 得拉盖尔多项式系数表达式
		\begin{equation*}
			\begin{aligned}
				c_{k}&= (-1)^k \frac{1}{(k!)^2}\frac{n! }{(n-k)!} n! \\
				&=(-1)^k \frac{(n!) ^2}{(k!)^2 (n-k)!},  \qquad (k=0,1,2,\cdots, n) \\ 
			\end{aligned}
		\end{equation*}	
		代回级数表达式,得拉盖尔多项式
		\begin{equation*}
		\begin{split}
			L_n(x) &=\sum_{k=0}^{n} c_{k} x^k \\
			&= \sum_{k=0}^{n} (-1)^k \frac{(n!)^2 }{(k!)^2 (n-k)!}x^k \\
			&= \sum_{k=0}^{n} (-1)^k \frac{n! }{(k!) (n-k)!} \frac{n!}{k!}x^k   \\
			&= \sum_{k=0}^{n} (-1)^k C^k _n \frac{n!}{k!}x^k
		\end{split}		
		\end{equation*}		
		\textcolor{red}{解毕!}
	
		~~\\ 
	
	\subsection{拉盖尔多项式及其性质} 
		\begin{example}
		试求前四个拉盖尔多项式
		\end{example}	
		\emph{解:}
		根据公式
		\begin{equation*}
				L_n(x)= \sum_{k=0}^{n} (-1)^k C^k _n \frac{n!}{k!}x^k
		\end{equation*}
		需要计算两个表格
			\renewcommand\arraystretch{1.4}
			\begin{table}[h]
				\centering
				\caption{$\dfrac{n!}{k!}$, ~($0 \le k\le n, n =1,2,3,4$)}
				%\label{tab:label}
				\begin{tabular}{cccccc}
						&1 & 1 & $\hspace{1.5em}$ & $\hspace{1.5em}$ & $\hspace{1.5em}$\\
						&2 & 2 & 1 & $\hspace{1.5em}$ & $\hspace{1.5em}$\\
						&6 & 6 & 3 & 1 & $\hspace{1.5em}$\\
						&24 & 24 & 12 & 4 & 1 \\	
				\end{tabular}
				\end{table}	 
				\renewcommand\arraystretch{1.4}
				\begin{table}[h]
					\centering
					\caption{$(-1)^k C^k _n$, ~($0 \le k\le n, n =1,2,3,4$)}
					%\label{tab:label}
					\begin{tabular}{cccccc}
							&1 & -1 & $\hspace{1.5em}$ & $\hspace{1.5em}$ & $\hspace{1.5em}$\\
							&1 & -2 & 1 & $\hspace{1.5em}$ & $\hspace{1.5em}$\\
							&1 & -3 & 3 & -1 & $\hspace{1.5em}$\\
							&1 & -4 & 6 & -4 & 1 \\	
					\end{tabular}
					\end{table}	\\
		两个表格对应位置的数字相乘,构成系数, 得拉盖尔多项式
		$$
	\begin{aligned}
	& L_0(x)=1 \\
	& L_1(x)=-x+1 \\
	& L_2(x)=x^2-4 x+2 \\
	& L_3(x)=-x^3+9 x^2-18 x+6 \\
	& L_4(x)=x^4 -16x^3 +72 x^2-96 x+24
	\end{aligned}
	$$
	前五个拉盖尔多项式的图形显示在图[\ref{fig:l01234}]中。 \begin{figure}[htbp]
		\centering
		\includegraphics[width=0.6\textwidth]{figs/l01234.png}
		\caption{拉盖尔多项式的图形}
		\label{fig:l01234}
	\end{figure}
	
		~~\\ 
	
	
		\begin{proposition}拉盖尔多项式微分形式
			\begin{equation*}
				L_n(x) =e^x \frac{d ^n}{d x^n} (x^n e^{-x})
			\end{equation*}	
		\end{proposition}
		\begin{proof}
			由高阶导数莱布尼兹公式 
			\begin{equation*}
				(u\cdot v) ^{(n)} =\sum_{k=0}^{n} C^k _n u^{(k) }v^{(n-k)} 
			\end{equation*}	
			得:
			\begin{equation*}
			\begin{split}
				(e^{-x}\cdot x^n) ^{(n)}&= \sum_{k=0}^{n} C^k _n [ e ^{-x}]^{(k)}   [(x^n)^{(n-k)}]\\
				&= \sum_{k=0}^{n} C^k _n [(-1)^k e ^{-x}]   [ \frac{n!}{k!} x^k]
			\end{split}		
			\end{equation*}	
		两端同除以$ e^x  $ 
			\begin{equation*}
				\begin{split}
					e^x \frac{d^n }{d x^n} (e^{-x}\cdot x^n) &=e^x \sum_{k=0}^{n} C^k _n [(-1)^k e ^{-x}]   [ \frac{n!}{k!}  x^k]\\
					&= \sum_{k=0}^{n} (-1)^k C^k _n \frac{n!}{k!}x^k  \\
					&=L_n (x)		
				\end{split}		
				\end{equation*}	
			\textcolor{red}{证毕!}
		\end{proof}
		~~\\ 
	
		\begin{proposition}拉盖尔多项式生成函数
			\begin{equation*}
				w(t,x) =\frac{e^{-xt/(1-t) } } {1-t}
			\end{equation*}	
		\end{proposition}
		\begin{proof}
			对函数在$t=0$做泰勒展开
			\begin{equation*}
			\begin{split}
				w(t,x) = \sum_{n=0}^{\infty} \frac{d^n w}{d t^n} |_{t=0} \frac{t^n}{n!}  = \sum_{n=0}^{\infty}  e^x \frac{d^n }{d x^n} (e^{-x}\cdot x^n) \frac{t^n}{n!}= \sum_{n=0}^{\infty} \frac{1}{n!} L_n t^n  
			\end{split}		
			\end{equation*}	
		\textcolor{red}{证毕!}
		\end{proof}

		\begin{proposition}拉盖尔多项式的递推式
			\begin{equation*}
			\begin{split}
				L_{n+1} &= (2n+1-x) L_n  -n^2 L_{n-1}  \\
				L_1 &=(1-x) L_0 
			\end{split}		
			\end{equation*}	
		\end{proposition}
		\begin{proof}
		对母函数求导
			\begin{equation*}
				\begin{split}
					w(t,x) &=\frac{e^{-xt/(1-t) } } {1-t}\\
					\frac{\partial w}{\partial t}  &= [\frac{1}{(1-t)^2} -\frac{x}{(1-t)^3}]  e^{-xt/(1-t) }     \\
					(1-t)^2	\frac{\partial w}{\partial t}  &= [1 -t-x] w   , \cdots \cdots  (1)
				\end{split}		
				\end{equation*}
		对展开式求导	
				\begin{equation*}
					\begin{split}
						w(t,x) &= \sum_{n=0}^{\infty}  L_n  \frac{t^n}{n!} \\
						\frac{\partial w}{\partial t}  &=   \sum_{n=1}^{\infty}  L_n  \frac{t^{n-1}}{(n-1)!}   \\
						&=  \sum_{n=0}^{\infty}  L_{(n+1)}  \frac{t^n}{(n)!} \\
						&=  \sum_{n=2}^{\infty}  L_{(n-1)}  \frac{t^{n-2}}{(n-2)!} \\
					\end{split}		
					\end{equation*}	
					代入(1)式的左边,有:
					\begin{equation*}
						(1-t)^2	\frac{\partial w}{\partial t}  =   \sum_{n=0}^{\infty}  L_{n+1}  \frac{t^n}{(n)!}  -2  \sum_{n=1}^{\infty}  L_{n}  \frac{t^n}{(n-1)!} +  \sum_{n=2}^{\infty}  L_{n-1}  \frac{n(n-1)}{(n)!}  t^n
					\end{equation*}	
					(1)式的右边,有:
					\begin{equation*}
						\begin{split}
							[1 -t-x] w &= (1-x)w -t w \\
							& =   \sum_{n=0}^{\infty} (1-x) L_n  \frac{t^n}{n!} -  \sum_{n=0}^{\infty}  L_n  \frac{t^{n+1}}{n!} \\
							&=  \sum_{n=0}^{\infty} (1-x) L_n  \frac{t^n}{n!} -  \sum_{n=1}^{\infty}  L_{n-1}  \frac{t^n}{(n-1)!}
						\end{split}		
					\end{equation*}	
					(1)式的左边=右边,系数应相等
					\[ L_{n+1} - 2 n L_{n} + n(n-1) L_{n-1} = (1-x) L_n  - n L_{n-1}   \]
					整理得递推式!
					\begin{equation*}
							L_{n+1} = (2n+1-x) L_n  -n^2 L_{n-1}  
					\end{equation*}	
		\textcolor{red}{证毕!}
		\end{proof}
		~~\\ 
	
		\begin{proposition}拉盖尔多项式带权($e^{-x}$)正交归一性 
			\begin{equation*}
				(n!)^2 \int_{0}^{\infty}  e^{-x} L_n L_m dx = \left\{
			\begin{split}
				0, \qquad (n=m)  \\
				1, \quad (n \ne m)  \\
			\end{split}	\right.	
			\end{equation*}	
		\end{proposition}
		\begin{proof}
		(1)正交性: 拉盖尔多项式满足拉盖尔方程:
		\begin{equation*}
			\begin{split}
				x L_n''  + [1 -x] L_n' +n L_n &=0  \\
				[xe^{-x}  L_n'] ' +n e^{-x} L_n &=0  \\
				[xe^{-x}  L_m'] ' +m e^{-x} L_m &=0  \\ 
				L_m[xe^{-x}  L_n'] ' +n e^{-x} L_m L_n &=0  \quad \cdots (a)\\
				L_n  [xe^{-x}  L_m'] ' +m e^{-x} L_n L_m& =0  \quad \cdots (b)\\ 
			\end{split}		
		\end{equation*}	
		(a)-(b),并两端积分
			\begin{equation*}
				\begin{split}
					(m-n) \int_{0}^{\infty}  e^{-x} L_n L_m dx &=  \int_{0}^{\infty} [L_n  [xe^{-x}  L' _m] ' - L_m[xe^{-x}  L' _n] '] dx \\
					&=  -\int_{0}^{\infty} L'_n  [xe^{-x}  L' _m] dx + L'_m[xe^{-x}  L' _n]  dx \\
					&=  \int_{0}^{\infty} [xe^{-x}  L' _m L' _n - xe^{-x}  L' _n L' _m]  dx =0
				\end{split}		
			\end{equation*}	
		(2)归一性:由递推式
			\begin{equation*}
				\begin{split}
					L_{n+1} &= (2n+1-x) L_n  -n^2 L_{n-1}  \\
					L_{n} &= (2n-1-x) L_{n-1}  -(n-1)^2 L_{n-2}  \\
					L^2 _{n} &= (2n-1-x) L_n  L_{n-1}  -(n-1)^2 L_n L_{n-2}  \\
					L_{n-1}	L_{n+1} &= (2n+1-x)	L_{n-1}	 L_n  -n^2 	L^2 _{n-1}  \\
					\int_{0}^{\infty}  e^{-x}  L^2 _{n}  dx &=  n^2   \int_{0}^{\infty}  e^{-x}  L^2 _{n-1} dx  \\
					&=  (n!)^2   \int_{0}^{\infty}  e^{-x}  L^2 _{0} dx  \\
					&=  (n!)^2 
				\end{split}		
			\end{equation*}		
		\textcolor{red}{证毕!}
		\end{proof}
		
		~~\\ 
		\subsection{连带拉盖尔多项式及其性质} 
		~\\
		连带拉盖尔方程
				\begin{equation*}
					x L''  + (m+1 -x) L' +n L =0
				\end{equation*}	 
		基于拉盖尔多项式($m=0$)定义广义拉盖尔多项式
				\begin{equation*}
					L^0 _n (x)= \frac{1} {n!} L_n (x) = \frac{1} {n!} \sum_{k=0}^{n} (-1)^k C^k _n \frac{n!}{k!}x^k 
				\end{equation*}	
			为了方便,广义拉盖尔多项式的上标记号通常略去。前几个广义拉盖尔多项式为 \\
			$$\begin{aligned}
				& L_0(x)=1 \\
				& L_1(x)=-x+1 \\
				& 2! L_2(x)=x^2-4 x+2 \\
				& 3! L_3(x)=-x^3+9 x^2-18 x+6 \\
				& 4! L_4(x)=x^4 -16x^3 +72 x^2-96 x+24
				\end{aligned}
			$$
			连带拉盖尔多项式就是$m\ne 0$的广义拉盖尔多项式
				\begin{equation*}
					L^m _n (x)= \frac{1} {n!}  \sum_{k=0}^{n} (-1)^k C^k _n \frac{(n+m)!}{(m+k)!}x^k = (-1)^m \frac{\mathrm{d}^m}{\mathrm{d}x^m} L_{n+m}(x)
				\end{equation*}	
			对连带拉盖尔多项式求导, 
			\begin{equation*}
				\begin{aligned}
					(L^m _n (x))'&= \frac{1} {n!}  \sum_{k=1}^{n} (-1)^k C^k _n \frac{(n+m)!}{(m+k)!} k x^{k-1} \\ 
					x (L^m _n (x))'&= \frac{1} {n!}  \sum_{k=0}^{n} (-1)^k C^k _n \frac{(n+m)!}{(m+k)!} k x^{k} 
				\end{aligned}
			\end{equation*}	
		   再求导,
			\begin{equation*}
				\begin{aligned}
					(L^m _n (x))''&= \frac{1} {n!}  \sum_{k=2}^{n} (-1)^k C^k _n \frac{(n+m)!}{(m+k)!} k(k-1) x^{k-2} \\ 
					x (L^m _n (x))''&= \frac{1} {n!}  \sum_{k=1}^{n} (-1)^k C^k _n \frac{(n+m)!}{(m+k)!} k(k-1) x^{k-1} \\ 
					&= \frac{1} {n!}  \sum_{k=0}^{n} (-1)^{k+1} C^{k+1} _n \frac{(n+m)!}{(m+k+1)!} k(k+1) x^k 
				\end{aligned}
			\end{equation*}	
			代回方程, 发现满足连带拉盖尔方程。即连带拉盖尔方程的解为连带拉盖尔多项式。
	
		~~\\ 
			当然,也可根据正则奇点邻域的级数解法,令 
			$$
			L(x)=\sum_{n=0}^{\infty} c_n x^{n+s}
			$$
			代回连带拉盖尔方程, 并整理得
			$$
		\begin{aligned}
		& \sum_{n=0}^{\infty} c_n(s+n)(s+n-1) x^{s+n-1}+(m+1) \sum_{n=0}^{\infty} c_n(s+n) x^{s+n-1} \\
		& ~~ -\sum_{n=0}^{\infty} c_n(s+n) x^{s+n}+\lambda \sum_{n=0}^{\infty} c_n x^{s+n}=0
		\end{aligned}
		$$
		由等式成立要求各项系数为0, 解得连带拉盖尔多项式
		\begin{equation*}
			L^m _n (x)= \frac{1} {n!}  \sum_{k=0}^{n} (-1)^k C^k _n \frac{(n+m)!}{(m+k)!}x^k
		\end{equation*}	
	
		~~\\ 
		连带拉盖尔多项式的性质:\\
		(1)生成函数:($\left\vert w \right\vert < 1, \alpha > -1$)
			$$
	(1-w)^{-\alpha-1} \exp \left(\frac{x w}{w-1}\right)=\sum_{n=0}^{\infty} L_n^\alpha(x) w^n
	$$
		(2)微分形式: 
				\begin{equation*}
					L^m _n(x) =\frac{x^{-m}e^x  }{n!} \frac{d ^n}{d x^n} (x^{m+n} e^{-x})
				\end{equation*}
		(3)递推式:
				\begin{equation*}
					(n+1)	L^m _{n+1} = (2n+1+m -x) L^m _n  - (n+m)  L_{n-1}  
				\end{equation*}	
		(4)正交归一性:
				\begin{equation*}
					\frac{n!}{(n+m)!}\int_{0}^{\infty}  e^{-x} x^m  L^m _n L^ m _k dx =\delta _{nk}
				\end{equation*}				
		(5)推论:
				\begin{equation*}
					\int_{0}^{\infty}  e^{-x} x^{m+1}  [L^m _{n}]^2  dx = \frac{(n+m)!}{n!}  (2n+m+1)
			\end{equation*}				
	
	~~\\ 
	\section{伽马函数}
	伽玛函数($\Gamma$函数)是阶乘在实数与复数上扩展的一类函数。该函数在物理学、光学工程、分析学、概率论、偏微分方程和组合数学中有重要的应用。
	
	伽玛函数的创造是件非常有意思的事。1728年,哥德巴赫在考虑数列插值的问题,他发现数列$$1,4,9,16,\ldots$$具有通项公式$a_n = n^2$,而当把$n$从正整数域延拓到实数域时,这个通项公式是函数$y=x^2$。即数列$a_n = n^2$ 只是函数$y=x^2$在$x$取正整数的值构成的序列。因此,哥德巴赫就想:阶乘数列$\{1,2,6,24,120,720,\ldots\}$的通项公式$a_n = n!$延拓到实数域时应对应某个函数,这样可计算非整数的阶乘,比如2.5!、$\pi !$等。他苦思不得此函数,于是写信请教尼古拉斯·伯努利和丹尼尔·伯努利。欧拉当时正好在丹尼尔·伯努利那里玩,因此他得知了这个问题。经过差不多一年的努力,只有22岁的欧拉找到了这个延拓函数!这就是$\Gamma$函数。
	
	\begin{figure}[htbp]
		\centering
		\includegraphics[width=0.5\textwidth]{figs/gamma1.png}
		\caption{$\Gamma$函数,阶乘的延拓}
		%\label{}
	\end{figure}
	
	\subsection{伽马函数的定义}~\\
	伽马函数具有多种定义,欧拉给出的伽马函数是一个含参数的无穷积分
	\begin{equation}
		\Gamma(z)=\int_{0}^{\infty} t^{z-1} e^{-t} dt , \quad (Re(z) > 0)
	\end{equation}
	这是定义在复数域的亚纯函数,除了零和负整数点以外,它全部解析。定义于实数域的伽马函数为
	\begin{equation*}
		\Gamma(x)=\int_{0}^{\infty} t^{x-1} e^{-t} dt, \quad (x>0)
	\end{equation*}	
	\begin{figure}[h]
		\centering
		\includegraphics[width=0.9\textwidth]{figs/gamma2.png}
		\caption{$\Gamma$函数的图像:左图为实数域,右图为复数域}
		%\label{}
	\end{figure}
	
	\subsection{伽马函数的性质}
	
	\begin{example} 试证明伽马函数存在
		\begin{equation}
		\boxed{\Gamma(1)=1} 
	\end{equation}	
	\end{example}
	\begin{proof}
		\[ \begin{aligned}
				\Gamma(z)&=\int_{0}^{\infty} t^{z-1} e^{-t} dt , \quad (Re(z) > 0) \\
			\Gamma(1)&=\int_{0}^{\infty} t^{1-1} e^{-t} dt \\	
					&=\int_{0}^{\infty}  e^{-t} dt \\
					&= \left.-e^{-t}\right\vert_{0}^{\infty} \\
					&=1	
		\end{aligned}\]
	\end{proof}
	~~\\ 
	
	\begin{proposition}
		$\Gamma$函数存在递推关系
	\begin{equation}
		\boxed{\Gamma(z+1)=z \Gamma(z)}
	\end{equation}
	\end{proposition}
	\begin{proof}
		由定义,有
	\[\begin{aligned}
		\Gamma(z+1)&= \int_{0}^{\infty} t^{(z+1)-1} e^{-t} dt \\
			&= \left.-t^z e^{-t} \right\vert_{0}^{\infty} + z \int_{0}^{\infty} t^{z-1} e^{-t} dt \\
			&= z \int_{0}^{\infty} t^{z-1} e^{-t} dt \\
			&=z \Gamma(z)
	\end{aligned}\]
	其中,用到了极限公式
	$$
	\lim _{t \rightarrow+\infty} \frac{t^n}{e^t}=0
	$$
	\end{proof}
	当从小向大递推时,有公式
	\begin{equation}
		\boxed{\Gamma(z)= \frac{1}{z}\Gamma(z+1)} 
	\end{equation}
	~~\\
	
	\begin{example}
		试证明$\Gamma$函数与阶乘有如下关系
	\begin{equation}
		\boxed{\Gamma(n+1)=n!} 
	\end{equation}
	\end{example}
	\begin{proof}
	在递推式
	\begin{equation*}
		\Gamma(z+1)=z \Gamma(z)
	\end{equation*}
	中,取 $z=n$
	\[\begin{aligned}
		\Gamma(n+1)&=n \Gamma(n) \\
		&=n(n-1) \Gamma(n-1) \\
		&=n(n-1)\cdots 1 \Gamma(1) \\
		&=n! \times 1 \\
		&=n!
	\end{aligned}\]
	\end{proof}
	~~\\
	
	\begin{example}
		试证明半正整数$\Gamma$函数的值与$\pi$有如下关系
	\begin{equation}
		\left\{
		\begin{aligned}
			&\Gamma(\frac{1}{2}) =\sqrt{\pi} \\ 
			&\Gamma(n+\frac{1}{2}) = \frac{(2n-1)!!}{2^n} \sqrt{\pi}
		\end{aligned}\right.	
	\end{equation}
	\end{example}
	\begin{proof}
		(1) 在$\Gamma$函数定义式中,取$z=\dfrac{1}{2}$
		\[
			\begin{aligned}
				 \Gamma(z)&=\int_{0}^{\infty} t^{ z-1} e^{-t} dt \\
				 \Gamma(\frac{1}{2})&=\int_{0}^{\infty} t^{\frac{1}{2}-1} e^{-t} dt\\
				 &=\int_{0}^{\infty} \frac{1}{\sqrt{t}} e^{-t} dt\\
				 &=\int_{0}^{\infty} \frac{1}{\sqrt{t}} e^{-(\sqrt{t})^2} d(\sqrt{t})^2
			\end{aligned}	
			\]
			令$x=\sqrt{t} \quad (\ge 0) $,做变量代换,有
			\[\Gamma(\frac{1}{2})
			=2\int_{0}^{\infty} e^{-x^2} dx\]
			问题转化为求积分
			\[I=\int_{0}^{\infty} e^{-x^2} dx\]
			\[I^2=\int_{0}^{\infty} \int_{0}^{\infty} e^{-x^2} e^{-y^2} dxdy\]
			由于 $x,y \ge 0$, 化为极坐标系积分时,应取第一象限,有 $0 \le \theta \le \pi/2$
			\[
			\begin{aligned}
					I^2
					 &=\int_{0}^{\frac{\pi}{2}}d\theta \int_{0}^{\infty}  e^{-r^2} r dr \\
					 &=\frac{\pi}{2} \int_{0}^{\infty} r e^{-r^2} dr \\
					 &=\frac{\pi}{2} \int_{0}^{\infty} \frac{1}{2} e^{-r^2} dr^2 \\
					 &=\frac{\pi}{2}\left[-\frac{1}{2} e^{-k}\right]_{0}^{\infty} \\
					 &=\frac{\pi}{4}
				\end{aligned}	
				\] 
		代回, 得 
		\[\Gamma(\frac{1}{2})
			=2I = \sqrt{\pi} \]
		 \\
		(2) 在递推公式 $\Gamma(1+z)=z \Gamma(z) $ 中,取 $z = 1/2$\\
		\[
			\begin{aligned}
				\Gamma(1+\frac{1}{2})&= \frac{1}{2} \Gamma(\frac{1}{2}) \\
				&=\frac{1}{2} \sqrt{\pi}\\
			\end{aligned} 
			  \]
		  \[
		\begin{aligned}
			\Gamma(2+\frac{1}{2})&= \Gamma(1+\frac{3}{2}) \\
			&=\frac{3}{2} \Gamma(1+\frac{1}{2}) \\
			&= \frac{3}{2}\frac{1}{2}\sqrt{\pi}
		\end{aligned} 
		  \]
		依次递推,有
		  \[
		\begin{aligned}
			\Gamma(3+\frac{1}{2})& = \frac{5}{2} \frac{3}{2}\frac{1}{2}\sqrt{\pi}\\
			&\cdots \\
			\Gamma(n+\frac{1}{2})& = \frac{(2n-1)!!}{2^n}\sqrt{\pi}\\
		\end{aligned} 
		  \]
		写成阶乘
		  \[
		\begin{aligned}
			\Gamma(n+\frac{1}{2})& = \frac{(2n-1)!!}{2^n}\sqrt{\pi}\\
			 & = \frac{(2n)!}{2^{2n} n!}\sqrt{\pi}
		\end{aligned} 
		  \]
		证毕!
	\end{proof}
	~~\\
	
	\begin{example}
		试证明半正整数的阶乘与$\Gamma$函数有如下关系
		\[ (m-\frac{1}{2})! =\Gamma (m+\frac{1}{2}) \]
		\[(\frac{1}{2})! =  \frac{\sqrt{\pi}}{2}\]
		\end{example}
		\begin{proof}
			把$\Gamma$函数与阶乘的关系式从正整数向正实数延拓!
			\[n!=\Gamma(n+1)\qquad \to \qquad x!=\Gamma(x+1), \quad (x>0)\]
			(1) 令 $x=m-\dfrac{1}{2}, \quad (m=1,2,3, \cdots )$,有
			\[ \begin{aligned}
				(m-\frac{1}{2})! &= \Gamma (m-\frac{1}{2}+1) \\
								 &= \Gamma (m+\frac{1}{2})
			\end{aligned} \]
			取$m=1$,有
			\[ \begin{aligned}
				(\frac{1}{2})! &= \Gamma (1+\frac{1}{2}) \\
							&= \frac{1}{2}\Gamma (\frac{1}{2}) \\
							&= \frac{\sqrt{\pi}}{2}
			\end{aligned} \]
		\end{proof}
	
		\begin{example}
			试证明半负整数$\Gamma$函数的值与$\pi$有如下关系
			\begin{equation}
				\begin{split}
				\Gamma(-\frac{1}{2}) &=-2\sqrt{\pi} \\ 
				\Gamma(-\frac{2k-1}{2}) &=(-1)^k\frac{2^{k}}{(2k-1)!!}\sqrt{\pi}
				\end{split}	
			\end{equation}
			\end{example}
			\begin{proof}
				在递推公式 $$\Gamma(z) = \frac{1}{z}\Gamma(1+z) $$ 
		(1) 取$z=-\dfrac{1}{2}$
		\[
		\begin{aligned}
			 \Gamma(-\frac{1}{2})&= (-\frac{2}{1}) \Gamma(1-\frac{1}{2})\\
			 &=(-\frac{2}{1}) \Gamma(\frac{1}{2})\\
			 &=-2 \sqrt{\pi}
		\end{aligned}	
		\]
		(2) 取$z=-\dfrac{3}{2}$
		\[\begin{aligned}
			\Gamma(-\frac{3}{2})&= -\frac{2}{3} \Gamma(1-\frac{3}{2})\\
			&=-\frac{2}{3} \Gamma(-\frac{1}{2})\\
			&=(-\frac{2}{3})(-\frac{2}{1}) \sqrt{\pi} \\
			&=\frac{4}{3}\sqrt{\pi} \\
	   \end{aligned}\]	
	   (3) 取$z=-\dfrac{5}{2}$
		\[\begin{aligned}
			\Gamma(-\frac{5}{2})&= -\frac{2}{5} \Gamma(1-\frac{5}{2})\\
			&=-\frac{2}{5} \Gamma(-\frac{3}{2})\\
			&=(-\frac{2}{5})(-\frac{2}{3})(-\frac{2}{1}) \sqrt{\pi} \\
	   \end{aligned}\]	
	   (4) 取$z=-\dfrac{2k-1}{2}$
		\[\begin{aligned}
			\Gamma(-\frac{2k-1}{2})
			&=(-\frac{2}{2k-1}) \cdots (-\frac{2}{5})(-\frac{2}{3})(-\frac{2}{1}) \sqrt{\pi} \\
			&= (-1)^k\frac{2^{k}}{(2k-1)!!}\sqrt{\pi} 
	   \end{aligned}\]
		\end{proof}
	
		\begin{proposition}
			$\Gamma$函数在非正整数点的极限值是无穷大
			\begin{equation}
				\lim\limits_{z\to -n }\Gamma(z)=\infty, \qquad (n=0,1,2, \cdots)
			\end{equation}
			\end{proposition}
			\begin{proof}
	把递推公式
		\[x\Gamma(x)= \Gamma(x+1) \]
		变形成
		\[\Gamma(x)=\frac{1}{x} \Gamma(x+1) \]
		求极限
		\[\lim\limits_{x\to 0 }	\Gamma(x)=	\lim\limits_{x\to 0 } \frac{1}{x} \Gamma(x+1) =\infty \]
	取$x\to -1$,有
	\[\lim\limits_{x\to -1 }\Gamma(x)=\lim\limits_{x\to -1 } \frac{1}{x} \Gamma(x+1) =  \lim\limits_{y\to 0 } \frac{1}{y-1} \Gamma(y) =\infty \]
	取$x\to -n$,有
		\begin{equation*}
		\begin{split}
			\lim\limits_{x\to -n }	\Gamma(x)=\lim\limits_{x\to -n } \frac{1}{x} \Gamma(x+1) 
			=\lim\limits_{y\to -n+1 } \frac{1}{y-1} \Gamma(y)  = \lim\limits_{x\to -n+1 }  \Gamma(x)
		\end{split}
		\end{equation*}	
	依次递推,有
	\[ 	\lim\limits_{x\to -n }	\Gamma(x) = \lim\limits_{x\to -n+2 }\Gamma(x) =\cdots  =\lim\limits_{x\to -1 }\Gamma(x) = \infty
	 \]
		\end{proof}
		通常,这个性质也可写成
		\begin{equation} \boxed{\frac{1}{\Gamma(-n)} =0, \qquad (n=0,1,2, \cdots)} \end{equation}
	\begin{hint} 
		$\Gamma(z)$函数存在非正整数奇点($z = 0,-1,-2,\cdots$), 并且它们都是一阶极点! 如此之外,在其他所有的点都连续。
	\end{hint}
	~~\\
	
	\subsection{伽马函数的应用}
	
	\begin{example}
	利用$\Gamma$函数计算积分
		\[ \int_{0}^{\infty} x^2 e^{-x^2} dx \]
	\end{example}
	
	\emph{解:}
	注意$\Gamma$函数的结构
		\[\Gamma(x)=\int_{0}^{\infty} t^{x-1} e^{-t} dt, \qquad (x>0)\]
		变型
		\[\begin{aligned}
			\int_{0}^{\infty} x^2 e^{-x^2} dx = \int_{0}^{\infty} x^2 e^{-(x^2)} \frac{1}{2x}d(x^2)
			= \frac{1}{2}\int_{0}^{\infty} (x^2)^{\frac{1}{2}} e^{-(x^2)} d(x^2)
		\end{aligned} \]
	令 $x^2 =t$,有 
		  \[\begin{aligned}
			\int_{0}^{\infty} x^2 e^{-x^2} dx  &= \frac{1}{2}\int_{0}^{\infty} t^{\frac{1}{2}} e^{-t} d(t) = \frac{1}{2}\int_{0}^{\infty} t^{\frac{3}{2}-1} e^{-t} d(t) \\	
			&= \frac{1}{2}\Gamma(\frac{3}{2})  \\ 
			&= \frac{1}{2} \Gamma(1+\frac{1}{2}) \\
			&= \frac{1}{2} (\frac{1}{2})!	\\
			&= \frac{\sqrt{\pi}}{4}
		\end{aligned} \]
	~~\\
	


\section{贝塞尔函数}
贝塞尔(Bessel)函数是贝塞尔方程的级数解。贝塞尔函数在波动问题和涉及势场的各种问题中有着重要应用,因此贝塞尔函数是除初等函数外,在物理和工程领域最为常用的特殊函数之一。

\subsection{贝塞尔方程}
\begin{example}
	对于半径为$r_0$的侧面绝缘的薄均匀圆盘,边界温度始终保持为0度,当盘的初始温度已知时 ($\Psi(x,y)$),求体系的温度分布函数。
\end{example}
\begin{figure}[h]
	\centering
	\includegraphics[width=0.25\textwidth]{figs/bessel0.png}
	\caption{薄均匀圆盘}
	%\label{}
\end{figure}
\emph{解:}
	这是非稳恒温度场,温度分布函数服从热传导方程:
	\begin{equation}\label{eq:bessel0}
		\begin{cases}
		u_t=a^2 [u_{xx}   +u_{yy}] ~~~~ (0< x^2 +y^2 <r_0 ^2, t>0)\\
		u(x,y,t)|_{x^2+y^2=r_0 ^2}= 0 \\
		u(x,y,t)|_{t=0}= \Psi(x,y)
	\end{cases} 
    \end{equation}
	考虑到圆域边界条件,为了便于分离变量,应改用极坐标系
	$$\begin{cases}
		\displaystyle	u_t=a^2 [ {	\frac{\partial^2 u }{\partial r^2 } +\frac{1}{r } \frac{\partial u }{\partial r } +
		\frac{1}{r^2 } \frac{\partial ^2 u }{\partial \theta ^2
		} }], ~~~~ (0<r<r_0, t>0)\\
		u(r,\theta)|_{r=r_0}=0 	\\
		u(r,\theta,t)|_{t=0} = \Psi(r,\theta) 	
	\end{cases} $$
	令 $u(r,\theta,t) =R(r)\Theta(\theta)T(t)$,  代回原方程
	\begin{equation*}
		R\Theta T'=a^2 [ R'' \Theta T + \dfrac{1}{r} R' \Theta T  + \dfrac{1}{r^2} R \Theta '' T ]
	\end{equation*}
	整理:
	\begin{equation*}
		\frac{T'}{a^2T} =\frac{R''}{R}+\frac{1}{r} \frac{R'}{R} +\frac{1}{r^2} \frac{\Theta ''} {\Theta}  =-\lambda
	\end{equation*}
	转化为两个方程:
	\begin{equation*}
		T'(t)+\lambda a^2T(t)=0  ~~~~~....... ~~~~~~(1)
	\end{equation*}	
	\begin{equation*}
		r^2 \frac{R'' (r)}{R(r)}+r \frac{R'(r)}{R(r)} + \lambda r^2 +\frac{\Theta ''(\theta)} {\Theta (\theta)} =0  ~~~~~~......~~(2)
	\end{equation*}
	方程-1是衰减模型,若固有值$\lambda$已知,则方程可解!\\
	方程-2是固有值问题,可继续分离变量:	
	\begin{equation*}
		r^2 \frac{R'' (r)}{R(r)}+r \frac{R'(r)}{R(r)} + \lambda r^2 =-\frac{\Theta ''(\theta)} {\Theta (\theta)} =\mu  
	\end{equation*}
	得两个固有值方程 \\
	角向:
	$$ \begin{cases}
		\Theta ''(\theta)+\mu \Theta (\theta) =0 \\
		\Theta (\theta) =	\Theta (\theta+2\pi)  \\
		\Theta' (\theta) =	\Theta' (\theta+2\pi)  
	\end{cases} $$	
	径向:
	$$ \begin{cases}
		r^2 R'' (r)+r R'(r) +( \lambda r^2 -\mu)R(r)=0  \\
		R(r_0)=0
	\end{cases} $$	
	角向固有值问题有解,\\
	固有值:
	\begin{equation*}
		\mu=n^2, ~~~~~(n=0,1,2,3...)
	\end{equation*}
	固有函数:
	\begin{equation*}
		\Theta_n(\theta) = \frac{1}{\sqrt{2\pi}} e^{-i n \theta }, ~~~~~(n=0,1,2,3...)
	\end{equation*}
	~\\
	把$\mu=n^2$,代回径向方程,径向固有值问题转化为贝塞尔方程
	\begin{equation} 
		\boxed{ \begin{cases}
		r^2 R'' (r)+r R'(r) +( \lambda r^2 -n^2)R(r)=0  \\
		R(r_0)=0
	\end{cases} }
    \end{equation}	
	设贝塞尔方程的解为$R(r)$, 则温度分布函数为
	$$u(r,\theta,t) =R(r)\Theta(\theta)T(t)$$	
	因此,问题转化为求解贝塞尔方程。

~~\\ 

\begin{example}
	鼓膜的运动可以用如下圆域波动方程描述
	\begin{equation}\label{eq:bessel1}
		\begin{cases}
		u_{tt}=a^2 [u_{xx}   +u_{yy}] ~~~~ (0< x^2 +y^2 <r_0 ^2, t>0)\\
		u(x,y,t)|_{x^2+y^2=r_0 ^2}= 0 \\
		u(x,y,t)|_{t=0}= \Psi(x,y), u'(x,y,t)|_{t=0}= \varphi(x,y)
	\end{cases} 
   \end{equation}
试求解鼓膜的波动函数
\end{example}
\emph{解:}
比较上述圆域波动方程和圆域热传导方程[\ref{eq:bessel0}],会发现只是把后方程的$u_{t}$变成了$u_{tt}$,导致相应的初始条件增加了一项$u'(x,y,t)|_{t=0}= \varphi(x,y)$。因此求解过程没有本质的不同。 \\
把方程写成极坐标形式
$$\begin{cases}
	\displaystyle	u_{tt}=a^2 [ {	\frac{\partial^2 u }{\partial r^2 } +\frac{1}{r } \frac{\partial u }{\partial r } +
	\frac{1}{r^2 } \frac{\partial ^2 u }{\partial \theta ^2
	} }], ~~~~ (0<r<r_0, t>0)\\
	u(r,\theta)|_{r=r_0}=0 	\\
	u(r,\theta,t)|_{t=0} = \Psi(r,\theta) ,\quad u'(r,\theta,t)|_{t=0}= \varphi(r,\theta)	
\end{cases} $$
令 $u(r,\theta,t) =R(r)\Theta(\theta)T(t)$,  代回原方程
	\begin{equation*}
		R\Theta T''=a^2 [ R'' \Theta T + \dfrac{1}{r} R' \Theta T  + \dfrac{1}{r^2} R \Theta '' T ]
	\end{equation*}
	整理:
	\begin{equation*}
		\frac{T''}{a^2T} =\frac{R''}{R}+\frac{1}{r} \frac{R'}{R} +\frac{1}{r^2} \frac{\Theta ''} {\Theta}  =-\lambda
	\end{equation*}
	转化为两个方程:
	\begin{equation*}
		T''(t)+\lambda a^2T(t)=0  ~~~~~....... ~~~~~~(1)
	\end{equation*}	
	\begin{equation*}
		r^2 \frac{R'' (r)}{R(r)}+r \frac{R'(r)}{R(r)} + \lambda r^2 +\frac{\Theta ''(\theta)} {\Theta (\theta)} =0  ~~~~~~......~~(2)
	\end{equation*}
	此处的方程-1是振动模型,若固有值$\lambda$已知,则方程可解\\
	此处的方程-2与上例的方程-2完全相同,经分离变量后,问题也转化为求贝塞尔方程
	\begin{equation*} 
		\begin{cases}
		r^2 R'' (r)+r R'(r) +( \lambda r^2 -n^2)R(r)=0  \\
		R(r_0)=0
	\end{cases} 
    \end{equation*}	

~~\\

\begin{remark}:
贝塞尔方程是圆域波动及圆域势场等相关问题的一个基本数学方程,具有普适性。
\end{remark}
~~\\

为了得到它的最简形式,做变量变换,令
	\begin{equation*}
		x=\sqrt{\lambda} r, ~~~~y(x)= R(r) =R(\frac{x}{\sqrt{\lambda}})
	\end{equation*}
	有: 
	\[
	\begin{aligned}
		&\frac{dy}{dx} = \frac{dR}{dr} \frac{dr}{dx} = \frac{1}{ \sqrt{\lambda}} \frac{dR}{dr}\\
		&\frac{d^2y}{dx^2} = \frac{1}{\lambda} \frac{dR^2}{dr^2}\\
	\end{aligned}	
	\]
	代回原方程,得:
	\begin{equation} \label{eq:bessel}
		\boxed{x^2\frac{d^2y}{dx^2} + x\frac{dy}{dx} +(x^2 -n^2)y=0}
	\end{equation}
	注意到$n$是角向方程的固有值,是整数。因此,称为整数阶贝塞尔方程。\\ 
	与欧拉方程
	\begin{equation*}
		x^2\frac{d^2y}{dx^2} + x\frac{dy}{dx} +n^2y=0
	\end{equation*}
	相比较,贝塞尔方程的零次项也是变系数的。 令$t=\ln x $, 欧拉方程可转化为常系数微分方程
	\[ \frac{d^2y}{dt^2} - n^2y=0 \]	
	同样令$t=\ln x $, 贝塞尔方程转化为 
	\[ \frac{d^2y}{dt^2} +(x^2-n^2)y=0 \]
	这依然是个变系数微分方程。因此,贝塞尔方程没有通常意义的初等函数表达式解。

\subsection{整数阶塞尔函数}
整数阶贝塞尔方程的解称为整数阶塞尔函数。现在我们来解整数阶贝塞尔方程。
\begin{example}
	求解贝塞尔方程
	\begin{equation*}
		x^2\frac{d^2y}{dx^2} + x\frac{dy}{dx} +(x^2 -n^2)y=0
	\end{equation*}		
\end{example}
\emph{解:}
	所谓没有通常意义的初等函数表达式解,即没有通常意义的级数解。现设方程有如下级数解:
	\begin{equation*}
		y=\sum\limits_{k=0}^{\infty} a_k x^{s+k},  \qquad (a_0 \not =0)
	\end{equation*}	
	做计算,注意指(脚)标对齐技巧的使用
	\[\begin{aligned}
		x^2 y &= \sum\limits_{k=0}^{\infty} a_k x^{s+k+2}  \\ 
		&= \sum\limits_{k=2}^{\infty}  a_{k-2} x^{s+k}  \\
		(x^2 -n^2)y &=  \sum\limits_{k=2}^{\infty}  a_{k-2} x^{s+k} - n^2 \sum\limits_{k=0}^{\infty} a_k x^{s+k} \quad \cdots \quad (1)
	\end{aligned}
	\] 
	求导:
	\begin{equation*}
	\begin{aligned}
		\frac{dy}{dx}&=\sum\limits_{k=0}^{\infty} (s+k) a_k x^{s+k-1} \\
		\frac{d^2y}{dx^2}&=\sum\limits_{k=0}^{\infty} (s+k) (s+k-1) a_k x^{s+k-2}
	\end{aligned}
	\end{equation*}
	因此,有
	\[\begin{aligned}
		x\frac{dy}{dx} &= \sum\limits_{k=0}^{\infty} (s+k) a_k x^{s+k} \quad \cdots \quad (2) \\ 
		x^2\frac{d^2y}{dx^2}&= \sum\limits_{k=0}^{\infty} (s+k) (s+k-1) a_k x^{s+k} \quad \cdots \quad (3)
	\end{aligned}
	\] 
	把(1)(2)(3)式代回方程,得:
	\begin{equation*}
		\sum\limits_{k=0}^{\infty} [(s+k) ^2 -n^2]  a_k x^{s+k} + \sum\limits_{k=2}^{\infty}  a_{k-2} x^{s+k} =0
	\end{equation*}	
	第一项(k=0)系数为零:
	\begin{equation*}
		\begin{aligned}
		[(s+0) ^2 -n^2] a_0 &= 0 \\
		s^2 -n^2 &= 0 \\
		\to  s_1=n, &\quad s_2=-n
	\end{aligned}
	\end{equation*}	
	第二项(k=1)系数应为零:
	\begin{equation*}
		[(s+1) ^2 -n^2] a_1=0,\qquad \to a_1=0. 
	\end{equation*}	
	后面各项系数都为零:
	\begin{equation*}
		[(s+k) ^2 -n^2] a_k+ a_{k-2}=0, ~~~ (k=2,3,4,...)
	\end{equation*}	
	得递推关系
	\begin{equation*}
		a_k=-\frac{1}{(s+k) ^2 -n^2 } a_{k-2}
	\end{equation*}	
	由$a_1=0$,可推出所有的奇数项系数为零 \[a_{2m+1}=0\]
	现求偶数项系数,令$k=2m\quad (m=1,2,3,...)$,并取$s=s_1=n$, (注:取$s=s_2=-n$ 也不影响解题过程),得:
	\begin{equation*}
	\begin{aligned}
		a_{2m}&=\frac{-1}{(n+2m) ^2 -n^2 } a_{2m-2} \\
		&=\frac{-1}{2^2 m (n+m) } a_{2(m-1)}
	\end{aligned}
	\end{equation*}	
	上式中,依次取$m=m-1, m-2, \cdots, 1$
	\begin{equation*}
		\begin{aligned}
			a_{2(m-1)} &=\frac{-1}{2^2 (m-1) (n+m-1) } a_{2(m-2)} \\ 
			a_{2(m-2)} &=\frac{-1}{2^2 (m-2) (n+m-2) } a_{2(m-3)} \\ 
			\cdots &\cdots \\
			a_{2} &=\frac{-1}{2^2 1 (n+1) } a_{0} \\
		\end{aligned}
		\end{equation*}	
	归纳,得:
	\begin{equation*}
		a_{2m}=(-1)^m  \frac{1}{2^{2m} m! (n+m) (n+m-1)... (n+1) } a_0 
	\end{equation*}	
	取:$a_0=1/2^n n!$, 得:
	\begin{equation*}
		a_{2m}=(-1)^m  \frac{1}{2^{2m+n} m! (n+m) ! }
	\end{equation*}	
	采用$ \Gamma $表示
	\begin{equation*}
		a_{2m}=(-1)^m  \frac{1}{2^{2m+n} m! \Gamma(n+m+1) }
	\end{equation*}	
	贝塞尔方程有级数特解:
	\begin{equation*}
		y(x) = \sum\limits_{m=0}^{\infty} a_{2m} x^{n+2m} 
	\end{equation*}		
	

~~\\ 
分析收敛性:
	\begin{equation*}
		\lim\limits_{m\to \infty}|\frac{ a_{2m+2}} {a_{2m}}|= \lim\limits_{m\to \infty}\frac{ 1}{4(m+1)(n+m+1)} =0
	\end{equation*}	
	满足级数收敛条件,说明$n$整数阶贝塞尔方程的级数特解是某函数的级数展开式,称为$n$整数阶贝塞尔函数,记为$J_n(x)$:
	\begin{equation}
		\boxed{\begin{aligned}
			J_n(x) &=\sum\limits_{m=0}^{\infty} a_{2m} x^{n+2m}  \\ 
			&= \sum\limits_{m=0}^{\infty}  \frac{(-1)^m }{ m! (n+m) ! } \left(\frac{x}{2}\right)^{n+2m} 
		\end{aligned}}
		\end{equation}
	~~\\ 
	贝塞尔函数可采用$\Gamma $函数来表示
	\begin{equation}
		\boxed{
			J_n(x) = \sum\limits_{m=0}^{\infty}  \frac{(-1)^m }{ m! \Gamma (n+m+1) ! } \left(\frac{x}{2}\right)^{n+2m} }
		\end{equation}
	取$n=0$, 得零阶贝塞尔函数$J_0(x)$
	\begin{equation*}
			J_0(x)  =\sum\limits_{m=0}^{\infty}  \frac{(-1)^m }{ (m!)^2 } \left(\frac{x}{2}\right)^{2m} 
	\end{equation*}

	\begin{figure}[h]
		\centering
		\includegraphics[width=0.5\textwidth]{figs/bessel1.png}
		\caption{贝塞尔函数}
		\label{fig:bessel1}
	\end{figure}


	依次取$n=0,1,2,3\cdots$, 可得$J_0(x),J_1(x),J_2(x),J_3(x)\cdots$,它们的图形如图[\ref{fig:bessel1}]所示。很显然,它们具有哀减振荡特性。在$x=0$处,各函数的取值为
	\begin{equation*}
		J_0(0)=1,\qquad J_1(0)=J_2(0) =J_3(0) = J_4(0) =J_5(0) = 0 
	\end{equation*}	
	注意:若在上述求解过程中取$s=s_2=-n$, 则可得负整数阶贝塞尔函数$J_{-n}(x)$。
	\begin{equation}
		\boxed{
			J_{-n}(x) = \sum\limits_{m=0}^{\infty}  \frac{(-1)^m }{ m! \Gamma (-n+m+1) ! } \left(\frac{x}{2}\right)^{-n+2m} }
		\end{equation}
	当然,存在半奇数阶贝塞尔函数, 比如:
	\begin{equation}
		\boxed{
			J_{\pm\frac{1}{2}}(x) = \sum\limits_{m=0}^{\infty}  \frac{(-1)^m }{ m! \Gamma (\pm \frac{1}{2}+m+1) ! } \left(\frac{x}{2}\right)^{\pm\frac{1}{2}+2m} }
		\end{equation} 
~~\\ 

\begin{example}
	试证明负整数阶贝塞尔函数与正整数阶贝塞尔函数之间存在如下关系
	\begin{equation*}
		J_{-n}(x)=(-1)^n J_n(x)
	\end{equation*}	
\end{example}
 \begin{proof}
	$\Gamma$ 函数形式的正整数阶贝塞尔函数:
	\begin{equation*}
		J_n(x) = \sum\limits_{m=0}^{\infty} (-1)^m  \frac{1}{m! \Gamma(n+m+1) } (\frac{x}{2})^{n+2m} 
	\end{equation*}	
	用$\Gamma$ 函数形式的负数阶塞尔函数
	\begin{equation*}
		J_{(-n)}(x) = \sum\limits_{m=0}^{\infty} (-1)^m  \frac{1}{m! \Gamma(-n+m+1) } (\frac{x}{2})^{-n+2m} 
	\end{equation*}	
	对于 $m<n$ 的项,分母中的$\Gamma$函数为无穷大,因此都为零,去除后为:
	\begin{equation*}
		J_{(-n)}(x) = \sum\limits_{m=n}^{\infty} (-1)^m  \frac{1}{m! \Gamma(-n+m+1) } (\frac{x}{2})^{-n+2m} 
	\end{equation*}	
	令 $m-n=k$, 有 $m=n+k $, 
	\begin{equation*}
	\begin{aligned}
		J_{(-n)}(x) &= (-1)^n\sum\limits_{k=0}^{\infty} (-1)^k  \frac{1}{(n+k)! \Gamma(k+1) } (\frac{x}{2})^{n+2k}  \\
		&= (-1)^n\sum\limits_{k=0}^{\infty} (-1)^k  \frac{1}{(n+k)! k! } (\frac{x}{2})^{n+2k} \\
		&=(-1)^n J_{n} (x)
	\end{aligned}
	\end{equation*}	
	证毕!
 \end{proof}
 ~~\\ 

 \begin{example}
	试证明半奇数阶贝塞尔函数有如下形式
	\begin{equation*}
		J_{1/2} (x) =\sqrt{\frac{2}{\pi x}} \sin x,  \qquad  J_{-1/2} (x) =\sqrt{\frac{2}{\pi x}} \cos x
	\end{equation*}
 \end{example}
  \begin{proof}
	(1)半奇数阶贝塞尔函数
	\begin{equation*}
		J_{1/2}(x) = \sum\limits_{m=0}^{\infty} (-1)^m  \frac{1}{m! \Gamma(1/2+m+1) } (\frac{x}{2})^{1/2+2m} 
	\end{equation*}	 
	计算$\Gamma$函数部分 
	\begin{equation*}
		\begin{split}
			\Gamma(1/2+m+1) &= (\frac{2m+1}{2}) \Gamma(1/2+m) \\
			& = (\frac{2m+1}{2}\frac{2m-1}{2})  \Gamma(1/2+m-1) \\
			&\cdots \cdots \\
			& = \frac{(2m+1)!!}{2^{m+1}} \Gamma(1/2) \\
			& = \frac{(2m+1)!!}{2^{m+1}} \sqrt{\pi} \\
		\end{split}	
	\end{equation*}	
	代回,有:
	\begin{equation*}
		J_{1/2}(x) = \sqrt{\frac{2}{\pi x}} \sum\limits_{m=0}^{\infty} (-1)^m  \frac{x^{2m+1}}{(2m+1)!} 
	\end{equation*}	 
	求和部分正是$ \sin $ 函数的展开式
	\begin{equation*}
		J_{1/2}(x) = \sqrt{\frac{2}{\pi x}} \sin x  
	\end{equation*}	 
	(2)负半奇数阶贝塞尔函数
	\begin{equation*}
		J_{-1/2}(x) = \sum\limits_{m=0}^{\infty} (-1)^m  \frac{1}{m! \Gamma(-1/2+m+1) } (\frac{x}{2})^{-1/2+2m} 
	\end{equation*}	
	计算Gamma函数部分 
		\begin{equation*}
			\begin{split}
				\Gamma(-1/2+m+1) &= \Gamma(1/2+m) \\
				& = \frac{2}{2m+1}\frac{(2m+1)!!}{2^{m+1}} \sqrt{\pi}  \\
				& =  \frac{(2m-1)!!}{2^m} \sqrt{\pi} \\
			\end{split}	
	\end{equation*}	
	代回,有:
	\begin{equation*}
		J_{-1/2}(x) = \sqrt{\frac{2}{\pi x}} \sum\limits_{m=0}^{\infty} (-1)^m  \frac{x^{2m}}{(2m)!} 
	\end{equation*}	 
	求和部分正是$ \cos $ 函数的展开式
	\begin{equation*}
		J_{-1/2}(x) = \sqrt{\frac{2}{\pi x}} \cos x  
	\end{equation*}	 
  \end{proof}
  


\subsection{贝塞尔函数的性质}
\begin{proposition}:1 贝塞尔函数的导数存在递推式	
	\begin{equation*}
		\begin{split}
			\left[x^{n} J_{n}(x)\right]'= &x^{n} J_{n-1}(x),\qquad \cdots (1) \\
			\left[x^{-n} J_{n}(x)\right]'=& -x^{-n} J_{n+1}(x) ,\qquad \cdots (2) 
		\end{split}
	\end{equation*}
\end{proposition}
\begin{proof}$\Gamma$函数形式的贝塞尔函数
	\begin{equation*}
		J_n(x) = \sum\limits_{m=0}^{\infty} (-1)^m  \frac{1}{m! \Gamma(n+m+1) } (\frac{x}{2})^{n+2m} 
	\end{equation*}	 
	(1)两端同乘$x^n$, 再求导:
	\begin{equation*}
		\begin{split}
			[x^n J_n(x)]' &= \frac{d}{dx}\sum\limits_{m=0}^{\infty} (-1)^m  
			\frac{1}{m! \Gamma(n+m+1) } (\frac{x^{2n+2m}}{2^{n+2m}})\\
			&=\sum\limits_{m=0}^{\infty} (-1)^m  
			\frac{2(n+m)}{m! \Gamma(n+m+1) } (\frac{x^{2n-1+2m}}{2^{n+2m}})\\
			&=x^n\sum\limits_{m=0}^{\infty} (-1)^m  
			\frac{1}{m! \Gamma(n-1+m+1) } (\frac{x^{n-1+2m}}{2^{n-1+2m}})\\ 
			&=x^n J_{n-1}(x) 
		\end{split}
		\end{equation*}	 
		(2)两端同乘$x ^{-n} $, 再求导:
		\begin{equation*}
			\begin{split}
				[x ^{-n}  J_n(x)]' &= \frac{d}{dx}\sum\limits_{m=0}^{\infty} (-1)^m  
				\frac{1}{m! \Gamma(n+m+1) } (\frac{x^{2m}}{2^{n+2m}})\\
				&=\sum\limits_{m=1}^{\infty} (-1)^m  
				\frac{2m}{m! \Gamma(n+m+1) } (\frac{x^{2m-1}}{2^{n+2m}})\\
				&=x ^{-n} \sum\limits_{m=1}^{\infty} (-1)^m  
				\frac{m}{m! \Gamma(n+m+1) } (\frac{x^{n-1+2m}}{2^{n-1+2m}})
			\end{split}
			\end{equation*}	 
	令  $ m = k+1  $
	\begin{equation*}
		\begin{split}
			[x ^{-n}  J_n(x)]' 
			&=x ^{-n} \sum\limits_{m=1}^{\infty} (-1)^m  
			\frac{m}{m! \Gamma(n+m+1) } (\frac{x^{n-1+2m}}{2^{n-1+2m}})\\ 
			&=-x ^{-n} \sum\limits_{k=0}^{\infty} (-1)^k  
			\frac{k+1}{(k+1)! \Gamma(n+k+1+1) } (\frac{x^{n+1+2k}}{2^{n+1+2k}})\\
			&=-x ^{-n} \sum\limits_{k=0}^{\infty} (-1)^k  
			\frac{1}{k! \Gamma(n+1+k+1) } (\frac{x^{n+1+2k}}{2^{n+1+2k}})\\
			&=-x ^{-n}  J_{n+1}(x) 
		\end{split}
		\end{equation*}
\end{proof}
 ~~\\ 

 \begin{proposition}:2 贝塞尔函数存在递推式	
	\begin{equation*}
		\begin{split}
			2 n J_{n}(x)=&xJ_{n-1}(x)+x J_{n+1}(x) ,\qquad \cdots (1) \\
			2 J_{n}^{\prime}(x)=&J_{n-1}(x)-J_{n+1}(x) ,\qquad \cdots (2) 
		\end{split}
	\end{equation*}
\end{proposition}
\begin{proof} 由导数递推式
	\begin{equation*}
		\begin{split}
			[x^n J_n(x)]' & = x^n J_{n-1}(x)  \\
			n x^{n-1} J_n(x) + x^n J'_n(x) & = x^n J_{n-1}(x) \qquad \cdots (3) \\
			[x ^{-n}  J_n(x)]' & = -x ^{-n}  J_{n+1}(x)  \\
			-n x^{-n-1} J_n(x) + x ^{-n} J'_n(x) & = -x ^{-n}  J_{n+1}(x) \qquad \cdots (4)  \\
		\end{split}
	\end{equation*}	
	$ (3)\times x ^{-n} - (4)\times x ^{n}  $, 并整理得:
	\begin{equation*}
		2 n J_{n}(x)=x J_{n-1}(x)+x J_{n+1}(x) 
	\end{equation*}	 	
	$ (3)\times x^{-n-1} + (4)\times x^{n-1}  $, 并整理得:
	\begin{equation*}
	2 J_{n}^{\prime}(x)=J_{n-1}(x)-J_{n+1}(x) 
	\end{equation*}	 
\end{proof}
 ~~\\ 

 \begin{example}
 试证明半奇阶贝塞尔函数
 \[J_{\frac{3}{2}} (x) = \sqrt{\frac{2}{\pi x}} \left[ \frac{1}{x} \sin x -\cos x\right]\]
 \end{example}
 \begin{proof}
 在递推式
 \[2 n J_{n} (x)= xJ_{n-1} (x)+x J_{n+1} (x)\]
取$n=\dfrac{1}{2}$
\[J_{\frac{1}{2}} (x)= xJ_{-\frac{1}{2}} (x)+x J_{\frac{3}{2}} (x)\]
代入$J_{-\frac{1}{2}} (x), J_{\frac{1}{2}} (x)$, 得
\[\sqrt{\frac{2}{\pi x}} \sin x= x \sqrt{\frac{2}{\pi x}} \cos x+x J_{\frac{3}{2}} (x)\]
整理得
\[J_{\frac{3}{2}} (x) = \sqrt{\frac{2}{\pi x}} \left[ \frac{1}{x} \sin x -\cos x\right]\]
\end{proof}
 ~~\\ 

 \begin{proposition}:3 贝塞尔函数存在渐近公式
	\begin{equation*}
		J_{n}(x) \approx \sqrt{\frac{2}{\pi x}} \cos \left(x-\frac{n \pi}{2}-\frac{\pi}{4}\right)
	\end{equation*}
\end{proposition}
\begin{proof} 对贝塞尔方程
	\begin{equation*}
		x^2\frac{d^2y}{dx^2} + x\frac{dy}{dx} +(x^2 -n^2)y=0
	\end{equation*}
	做变量代换 \[ y=\frac{u}{\sqrt{x}}\] 
	求导:
	\[ y'=\frac{u'}{x^{\frac{1}{2}}} -\frac{1}{2}\frac{u}{x^{\frac{3}{2}}} \] 
	\[ y''=\frac{u''}{x^{\frac{1}{2}}} -\frac{u'}{x^{\frac{3}{2}}} + \frac{3}{4}\frac{u}{x^{\frac{5}{2}}} \] 
	代入,得关于 $u(x)$的方程:
	\begin{equation*}
		u'' +[1+\frac{\frac{1}{4}-n^2}{x^2}] u=0
	\end{equation*}
	当$x \to \infty $时, 有方程:
	\begin{equation*}
		u'' + u=0 
	\end{equation*}	
	通解为:\[u=A\cos(x+\theta)\]
	确定系数$A$和$\theta$ (不证)
	\[A =\sqrt{\frac{2}{\pi}}, \quad \theta = -[\frac{n \pi}{2}+\frac{\pi}{4}]  \]
	得n阶贝塞尔函数的渐近形式
	\begin{equation*}
		J_{n}(x) \approx \sqrt{\frac{2}{\pi x}} \cos \left(x-\frac{n \pi}{2}-\frac{\pi}{4}\right)
	\end{equation*}
	\begin{figure}[h]
		\centering
		\includegraphics[width=0.5\textwidth]{figs/bessel2.png}
		\caption{贝塞尔函数的渐近形式}
		\label{fig:bessel2}
	\end{figure}
	图[\ref{fig:bessel2}]给出的前几个贝塞尔函数渐近形式的图形, 与标准的贝塞尔函数图形[\ref{fig:bessel1}]相比较,可以看出有明显的差别。
	但是贝塞尔函数是无穷级数,这给计算带来很大的复杂性。贝塞尔函数渐近形式是解释式,计算非常方便。\\
	~~\\ 
	若令$J_{n}(x)=0$,则对应图像与x轴的交叉点,即零点。$J_{n}(x)$有无穷多个零点,它们与$\cos$函数的零点相关。结合$\cos$函数零点公式,可得n阶贝塞尔函数的零点近似公式:
	\begin{equation*}
		\mu_{m}^{n} \approx m \pi+\frac{n \pi}{2}+\frac{3 \pi}{4}, \quad m=0,\pm 1, \pm 2, \cdots 
	\end{equation*}
\end{proof}
~~\\ 

在圆域热传导方程及波动方程中,分离变量所得的径向方程为
	\[\begin{cases}
		r^2 R'' (r)+r R'(r) +( \lambda r^2 -n^2)R(r)=0  \\
		R(r_0)=0
	\end{cases}  \]
	方程的解为$n$阶贝塞尔函数 $$R(r)=J_n(x) = J_n(\sqrt{\lambda}r)$$ 
	零边界条件 $R(r_0)=0$ 可表示为: $$J_n(\sqrt{\lambda}r_0)=0$$ 
	零边界条件下的固有值问题, 正是贝塞尔函数的零点问题, 由\\
	\[ \sqrt{\lambda}r_0 = \mu_{m}^{n}\]
	得:\\
	(1)固有值
	\[\lambda_m ^n =(\frac{\mu_{m}^{n}}{r_0})^2 \quad m=0,\pm 1, \pm 2, \cdots \]
	(2)固有函数
	\[R_m ^n(r) = J_n (\frac{\mu_{m}^{n}}{r_0}r) \]
	把固有值$ \lambda_m ^n  $ 代入方程
	\[T''(t)+\lambda a^2T(t)=0 \] 
	结合初始条件,圆域波动问题得解。\\
	把固有值$ \lambda_m ^n  $ 代入方程
	\[T'(t)+\lambda a^2T(t)=0 \] 
	结合初始条件,圆域热传导问题得解 \\
 ~~\\ 

\begin{proposition}-4:贝塞尔函数有如下零点递推式
	\[ J'_n(\mu_m ^n)= - J_{n+1}(\mu_m ^n)\]
	\end{proposition}
\begin{proof}
	由微分递推公式 
 	\begin{equation*}
	\begin{split}
	 \left[x^{-n} J_{n}(x)\right]'=& -x^{-n} J_{n+1}(x) \\
	 -n x^{-n-1} J_n(x) + x^{-n} J'_n(x)&= -x^{-n} J_{n+1}(x) \\
	 -n J_n(x) + x J'_n(x)&= -x J_{n+1}(x) \\
	\end{split}
\end{equation*}
代入零点$J_n(x) = J_n(\mu_m ^n) =0 $
\begin{equation*}
	\begin{split}
	 -n 0 + x J'_n(\mu_m ^n)&= -x J_{n+1}(\mu_m ^n) \\
	 J'_n(\mu_m ^n)&= - J_{n+1}(\mu_m ^n)
 	\end{split}
	\end{equation*}	
\textcolor{red}{证毕!}
\end{proof}
~~\\

\begin{proposition}-5:贝塞尔函数正交归一性
	\begin{equation*}
		\int_0 ^{r_0} r J_n (\frac{\mu_{m} ^{n}}{r_0}r) J_n (\frac{\mu_{k} ^{n}}{r_0}r) dr =
		\left\{
		\begin{aligned}
			0  \qquad \qquad \qquad (m \ne k)\\
			\frac{1}{2} r_0^2 [J'_n(\mu_m ^n)]^2 \quad (m = k)
		\end{aligned}\right.
	\end{equation*}
	\end{proposition}
\begin{proof}
	(1) 对于径向方程:
	\begin{equation*}
		r^2 R''+r R' +(\lambda r^2 -n^2)R=0
	\end{equation*}
	代入固有值
	\begin{equation*}
		r^2 R''+r R' +((\frac{\mu_{m}^{n}}{r_0})^2 r^2 -n^2)R=0 
	\end{equation*}	
	做等价变换
	\begin{equation*}
		r R''+ R' +((\frac{\mu_{m}^{n}}{r_0})^2 r -\frac{n^2}{r})R=0  
	\end{equation*}	
	\begin{equation*}
		(r R')' +((\frac{\mu_{m}^{n}}{r_0})^2 r -\frac{n^2}{r})R=0  
	\end{equation*}	
	 令:\[J_n (\frac{\mu_{m}^{n}}{r_0}r)=R_1, \qquad J_n (\frac{\mu_{k}^{n}}{r_0}r) =R_2\]
	它们都是方程的解
	\begin{equation*}
		(r R_1')' +((\frac{\mu_{m}^{n}}{r_0})^2 r -\frac{n^2}{r})R_1=0  \cdots (1)
	\end{equation*}	 
	\begin{equation*}
		(r R_2')' +((\frac{\mu_{k}^{n}}{r_0})^2 r -\frac{n^2}{r})R_2=0  \cdots (2) 
	\end{equation*}	
	$(1)\times R_2-(2)\times R_1,$,并积分
	\begin{equation*}	
		\left[\left(\frac{\mu_{m}^{n}}{r_0}\right)^{2}-\left(\frac{\mu_{k}^{n}}{r_0}\right)^{2}\right] \int_0 ^{r_0} r R_{1} R_{2} dr 
		=\int_0 ^{r_0}  [R_{1}\left(r R_{2}^{\prime}\right)^{\prime}-R_{2}\left(r R_{1}^{\prime}\right)^{\prime}] dr
	\end{equation*}
	\begin{equation*}
		\begin{split}
			=& [r R_1 R_2']|_0 ^{r_0} - [r R_2 R_1']|_0 ^{r_0} + \int_0 ^{r_0} r R_2' R_1' dr - \int_0 ^{r_0} rR_1' R_2' dr\\
			=& \int_0 ^{r_0} rR_2'R_1' dr - \int_0 ^{r_0} rR_1'R_2' dr\\
			=& 0
		\end{split}
	\end{equation*}	
	当$ m \ne k $ 时,有
	\begin{equation*}	
		\int_0 ^{r_0} r R_{1} R_{2} dr = 0
	\end{equation*}
	(2) 对于径向方程
	\begin{equation*}
		r^2 R''+r R' +(\lambda r^2 -n^2)R=0 
	\end{equation*}	
	乘以$R'$
	\begin{equation*}
		2r^2 R'R''+2r (R')^2 +(\lambda r^2 -n^2)R'R=0 
	\end{equation*}	
	整理,得
	\begin{equation*}
		\frac{1}{2 \lambda}[r^2 (R')^2 + (\lambda r^2 -n^2)R^2]'= rR^2
	\end{equation*}
	两端积分
	\begin{equation*}
		\begin{split}
			\int_0 ^{r_0} r R^2 dr =& \frac{1}{2\lambda} \int_0 ^{r_0} [r^2 (R')^2 + (\lambda r^2 -n^2)R^2]' dr  \\
			=& \frac{1}{2\lambda} |r^2 (R')^2 + (\lambda r^2 -n^2)R^2 |_0 ^{r_0} \\
			=& \frac{1}{2\lambda} r_0^2 (R'(r_0))^2 \\
			=& \frac{1}{2} r_0^2 [J'_n(\mu_m ^n)]^2 \\
			=& \frac{r_0^2}{2} [J_{n+1}(\mu_m ^n)]^2
		\end{split}
	\end{equation*}
\textcolor{red}{证毕!}
\end{proof}
~~\\ 

\begin{example}
设函数$f(r)$的贝塞尔函数展开式为
	\begin{equation*}
		f(r)=\sum_{m=1}^\infty c_m J_n(\frac{\mu_m ^n}{r_0} r)
	\end{equation*}	
	试求展开系数$c_m$	
\end{example}
 
\emph{解:}
	展开式两端同乘以$r J_n(\frac{\mu_k ^n}{r_0} r)$
	\begin{equation*}
		r J_n(\frac{\mu_k ^n}{r_0} r) f(r)=\sum_{m=1}^\infty c_m r J_n(\frac{\mu_k ^n}{r_0} r) J_n(\frac{\mu_m ^n}{r_0} r)
	\end{equation*}
	两端积分
	\begin{equation*}
		\int_0 ^{r_0}  r J_n(\frac{\mu_k ^n}{r_0} r) f(r) dr=\sum_{m=1}^\infty c_m \int_0 ^{r_0}  r J_n(\frac{\mu_k ^n}{r_0} r) J_n(\frac{\mu_m ^n}{r_0} r) dr
	\end{equation*}
	由正交性, 右边$m \ne k $的积分项都为零
	\begin{equation*}
		\begin{aligned}
			\int_0 ^{r_0}  r J_n(\frac{\mu_m ^n}{r_0} r) f(r) dr &= c_m \int_0 ^{r_0}  r J_n(\frac{\mu_m ^n}{r_0} r) J_n(\frac{\mu_m ^n}{r_0} r) dr \\ 
			&= c_m \frac{r_0^2}{2} [J_{n+1}(\mu_m ^n)]^2 
		\end{aligned}
	\end{equation*}
	整理得
	\begin{equation*}
		c_m=\frac{2} {r^2_0 J_{n+1} ^2 (\mu_m ^n)} \int_0 ^{r_0} r f(r) J_n (\frac{\mu_m ^n}{r_0} r)  dr 
	\end{equation*}	

~~\\ 

\subsection{应用实例}
\begin{example}
	求解热传导方程
	\begin{equation*}
		\begin{cases}
		u_t= u_{xx}   +u_{yy} ~~~~ (0< x^2 +y^2 <R ^2, t>0)\\
		u(x,y,t)|_{x^2+y^2=r_0 ^2}= 0 \\
		u(x,y,t)|_{t=0}= x^2+y^2 +2
	\end{cases} 
    \end{equation*}
\end{example}
\emph{解:}
	这是圆域热传导方程, 改用极坐标系
	$$\begin{cases}
		\displaystyle	u_t= {	\frac{\partial^2 u }{\partial r^2 } +\frac{1}{r } \frac{\partial u }{\partial r } +
		\frac{1}{r^2 } \frac{\partial ^2 u }{\partial \theta ^2
		} }, ~~~~ (0<r<R, 0<\theta<2\pi, t>0)\\
		u(r,\theta)|_{r=R}=0 	\\
		u(r,\theta,t)|_{t=0} =r^2 +2 	
	\end{cases} $$
	令 $u(r,\theta,t) =F(r)\Theta(\theta)T(t)$, 代回, 原方程可分离变量成三个方程 \\
	方程-1: $$ \begin{cases}
		\Theta ''(\theta)+\mu \Theta (\theta) =0 \\
		\Theta (\theta) =	\Theta (\theta+2\pi)  \\
		\Theta' (\theta) =	\Theta' (\theta+2\pi)  
	\end{cases} $$
	方程-2:	$$ \begin{cases}
		r^2 F'' (r)+r F'(r) +( \lambda r^2 -\mu)F(r)=0  \\
		F(R)=0
	\end{cases} $$
	方程-3: \begin{equation*}
		T'(t)+\lambda T(t)=0 
	\end{equation*}	
	方程-1 有解 \\
	固有值:
	\begin{equation*}
		\mu=n^2, ~~~~~(n=0,1,2,3...)
	\end{equation*}
	固有函数:
	\begin{equation*}
		\Theta_n(\theta) = a_n e^{-i n \theta}, ~~~~~(n=0,1,2,3...)
	\end{equation*}
	把固有值$ \mu = n^2 $ 代入方程-2
	$$ \begin{cases}
		r^2 F'' (r)+r F'(r) +( \lambda r^2 -n^2)F(r)=0  \\
		F(R)=0
	\end{cases} $$
	这是贝塞尔方程的零点问题, 有解 \\
	固有值:
	\[\lambda_m ^n =(\frac{\mu_{m}^{n}}{R})^2, \quad (m=0, \pm 1, \pm 2,\cdots) \]
	固有函数:\[F_m ^n(r) = J_n (\frac{\mu_{m}^{n}}{R}r) \]
	把固有值$ \lambda_m ^n  $ 代回方程-3: \begin{equation*}
		T'(t)+(\frac{\mu_{m}^{n}}{R})^2 T(t)=0 
	\end{equation*}	
	这是衰减模型, 有解
	\begin{equation*}
		T_m ^n=Ce^{- (\frac{\mu_{m}^{n}}{R})^2 t}
	\end{equation*}
	方程的基本解为
	$$u_m ^n(r,\theta,t) = J_n (\frac{\mu_{m}^{n}}{R}r) e^{-i n \theta} e^{- (\frac{\mu_{m}^{n}}{R})^2 t}$$
	叠加解为
	$$u(r,\theta,t) = \sum_{n=0}^{\infty} \sum_{m=0}^{\infty} c_m ^n J_n (\frac{\mu_{m}^{n}}{R}r) e^{-i n \theta - (\frac{\mu_{m}^{n}}{R})^2 t}$$
	取$t=0$, 结合初值条件
	$$u(r,\theta,0) = r^2 +2  = \sum_{n=0}^{\infty} \sum_{m=0}^{\infty} c_m ^n J_n (\frac{\mu_{m}^{n}}{R}r) e^{-i n \theta }$$
	两端同乘以 $e^{i k \theta }$, 并对$\theta$ 积分
	\[
	\begin{aligned}
	\int _0 ^{2\pi} e^{i k \theta } (r^2 +2) d\theta & = \sum_{n=0}^{\infty} \sum_{m=0}^{\infty} c_m ^n  J_n (\frac{\mu_{m}^{n}}{R}r) \int_0 ^{2\pi} e^{i k \theta } e^{-i n \theta}  d\theta \\ 
	\int _0 ^{2\pi} e^{i m \theta } \frac{r^2 +2}{2\pi} d\theta & = \sum_{m=0}^{\infty} c_m ^n J_n (\frac{\mu_{m}^{n}}{R}r)
	\end{aligned}\]
	两端同乘以 $ r J_n (\frac{\mu_{k}^{n}}{R}r)$, 并对$r$ 积分
	\[ \begin{aligned}
		\int _0 ^{R} \int _0 ^{2\pi} r J_n (\frac{\mu_{k}^{n}}{R}r) e^{i m \theta } \frac{r^2 +2}{2\pi} d\theta dr & = \sum_{m=0}^{\infty} c_m ^n \int _0 ^{R}   r J_n (\frac{\mu_{k}^{n}}{R}r) J_n (\frac{\mu_{m}^{n}}{R}r) dr \\
		\int _0 ^{R} \int _0 ^{2\pi} r J_n (\frac{\mu_{m}^{n}}{R}r) e^{i m \theta } \frac{r^2 +2}{2\pi} d\theta dr & = c_m ^n \frac{1}{2} R^2 [J'_n(\mu_m ^n)]^2 \\
		\end{aligned}\]
	整理得
	\[c_m ^n = \frac{1}{R^2 [J'_n(\mu_m ^n)]^2} \int _0 ^{R} \int _0 ^{2\pi}  J_n (\frac{\mu_{m}^{n}}{R}r)\frac{r^3 +2r}{\pi} e^{i m \theta }  d\theta dr \]
	方程的最终解为
	\[
	\left\{\begin{aligned}
		u(r,\theta,t) & = \sum_{n=0}^{\infty} \sum_{m=0}^{\infty} c_m ^n J_n (\frac{\mu_{m}^{n}}{R}r) e^{-i n \theta - (\frac{\mu_{m}^{n}}{R})^2 t} \\
		\\
		c_m ^n &= \frac{1}{R^2 [J'_n(\mu_m ^n)]^2} \int _0 ^{R} \int _0 ^{2\pi}  J_n (\frac{\mu_{m}^{n}}{R}r)\frac{r^3 +2r}{\pi} e^{i m \theta }  d\theta dr 
	\end{aligned}\right. \]


\begin{example}
	求解热传导方程
	$$\begin{cases}
		\displaystyle u_t= \frac{\partial^2 u }{\partial r^2 } +\frac{1}{r } \frac{\partial u }{\partial r }, ~~~~ (0<r<1, 0<\theta<2\pi, t>0) \\
		u(r,\theta)|_{r=1}=0 	\\
		u(r,\theta,t)|_{t=0} =1-r^2
	\end{cases} $$
\end{example}
\emph{解:}
	这是极坐标系下的圆域热传导方程。与标准的方程相比较,主方程少了与$\theta$有关的那一项,因此 \\
	令 $u(r,\theta,t) =F(r)T(t)$, 代回, 原方程可分离变量成三个方程 \\
	方程-1:	$$ \begin{cases}
		r^2 F'' (r)+r F'(r) +( \lambda r^2 )F(r)=0  \\
		F(r)|_{r=1}=0
	\end{cases} $$
	方程-2: \begin{equation*}
		T'(t)+\lambda T(t)=0 
	\end{equation*}	
	方程-1是零阶($n=0$)贝塞尔方程的零点问题, 有解 \\
	固有值:
	\[\lambda_m ^0 =(\mu_{m}^{0})^2, \quad (m=0, \pm 1, \pm 2,\cdots) \]
	固有函数:\[F_m ^0(r) = J_n (\mu_{m}^{0}r) \]
	把固有值$ \lambda_m ^0  $ 代回方程-2: \begin{equation*}
		T'(t)+(\mu_{m}^{0})^2 T(t)=0 
	\end{equation*}	
	这是衰减模型, 有解
	\begin{equation*}
		T_m ^n=Ce^{- (\mu_{m}^{0})^2t}
	\end{equation*}
	方程的基本解为
	$$u_m ^n(r,\theta,t) = J_n ((\mu_{m}^{0})^2r) e^{- (\mu_{m}^{0})^2 t}$$
	叠加解为
	$$u(r,\theta,t) = \sum_{m=0}^{\infty} c_m ^0 J_n ((\mu_{m}^{0})^2r) e^{- (\mu_{m}^{0})^2 t}$$
	取$t=0$, 结合初值条件
	$$u(r,\theta,0) = 1- r^2  = \sum_{m=0}^{\infty} c_m ^0 J_n (\mu_{m}^{0}) $$
	两端同乘以 $ r J_n (\mu_{m}^{0}r)$, 并对$r$ 积分
	\[ \begin{aligned}
		\int _0 ^{1} r J_n (\mu_{k}^{0}r)  (1-r^2)  dr & = \sum_{m=0}^{\infty} c_m ^0 \int _0 ^{1}   r J_n (\mu_{k}^{0}r) J_n (\mu_{m}^{0}r) dr \\
		\int _0 ^{1}  r J_n (\mu_{m}^{0}r) (1-r^2)  dr & = c_m ^0 \frac{1}{2} [J'_n(\mu_m ^0)]^2 \\
		\end{aligned}\]
	整理得
	\[c_m ^0 = \frac{2}{[J'_n(\mu_m ^0)]^2} \int _0 ^{1} J_n (\mu_{m}^{0}r) (r-r^3) dr \]
	方程的最终解为
	\[
	\left\{\begin{aligned}
		u(r,\theta,t) &= \sum_{m=0}^{\infty} c_m ^0 J_n ((\mu_{m}^{0})^2r) e^{- (\mu_{m}^{0})^2 t} \\
		\\
		c_m ^0 &= \frac{2}{[J'_n(\mu_m ^0)]^2} \int _0 ^{1} J_n (\mu_{m}^{0}r) (r-r^3) dr  
	\end{aligned}\right. \]

~~\\ 


\section{狄拉克函数}
物理上存在质点、点电荷等物理模型。其体积无穷小,质量密度或电荷密度无穷大,数学上采用狄拉克函数-$\delta(x) $ 描述这些“点”模型密度的空间分布。
\subsection{狄拉克函数定义}
设有一均匀细线,其质量为$1~\text{kg}$长度为$2l~\text{m}$。则其线密度为$$\rho=\frac{1}{2l}~\text{(kg/m)}$$\\
若取线段中点为坐标原点,线密度的数学表达为 
	  \[\rho(x)=\left\{\begin{array}{c}
		\dfrac{1}{2 l}, \qquad (-l \leq x \leq l) \\
		0, \quad (x<-l, x>l)
		\end{array}\right.\]
积分得质量
	 \[\int_{-\infty}^{+\infty} \rho(x) d x= 1\]
显然,有
	 \[\int_{a}^{b} \rho(x) d x =1, \qquad  (a,b) \subset(-l,l) \]
考虑当$l \to 0$时, 线段变为质点模型。把质点的点密度定义为狄拉克函数$\delta(x)$,有
	 \[\delta(x)=\left\{\begin{array}{c}
		 \infty, \quad(x=0) \\
		 0, \quad(x \neq 0)
		 \end{array}\right. \]
狄拉克函数的积分为
	 \[\int_{-\infty}^{+\infty} \delta(x) d x=1 \]
考虑到物理测量,不防改$\infty$为 $1$,把狄拉克函数定义改为
	 \[\delta(x)=\left\{\begin{array}{c}
		 1, \quad(x=0) \\
		 0, \quad(x \neq 0)
		 \end{array}\right. \]
而它的积分不变
	 \[\int_{-\infty}^{+\infty} \delta(x) d x=1 \] 
则狄拉克函数依然描述了物理学的各种“点”模型。\\
考虑到量子化, 狄拉克函数也定义成一个连续实函数的序列 
\[\lim_{n\to\infty} \int_{-\infty}^{+\infty} \delta_n (x) d x=1 \]

\subsection{狄拉克函数性质}
1、$\delta(x)$函数的积分有如下性质
	$$\begin{aligned}
		&\int _{a} ^{b} \delta(x) d x=1 , \quad 0 \in (a,b) \\
		&\int_{-\infty}^{+\infty} \delta(x) \psi (x) d x=\psi  (0) \\
		&\int_{a}^{b} \delta(x) \psi (x) d x=\psi (0)  , \quad 0\in(a,b) \\
		&\int_{-\infty}^{+\infty} \delta(x-x_0) \psi (x) d x=\psi  (x_0)\\
	\end{aligned}$$
\begin{proof} 
性质-1:
\[ 1 = \int_{-\infty}^{a} \delta(x) d x + \int _{a} ^{b} \delta(x) d x + \int _{b} ^{+\infty} \delta(x) d x = \int _{a} ^{b} \delta(x) d x \]
性质-2:
\[\delta(x) \psi (x) = \left\{\begin{aligned}
	\psi (0), \quad(x=0) \\
	0, \quad(x \neq 0)
\end{aligned}\right. \quad = \psi (0) \delta(x) \]
\[ \int_{-\infty}^{+\infty} \delta(x) \psi (x) d x= \int_{-\infty}^{+\infty} \psi  (0)\delta(x)d x = \psi (0) \int_{-\infty}^{+\infty} \delta(x)d x = \psi  (0)\]
性质-3:
  \[ \int_{a}^{b} \delta(x) \psi (x) d x= \int_{a}^{b} \psi  (0)\delta(x)d x = \psi (0) \int_{a}^{b} \delta(x)d x = \psi  (0)\]
性质-4: 令 $x'=x-x_0$
\[ \int_{-\infty}^{+\infty} \delta(x-x_0) \psi (x) d x = \int_{-\infty}^{+\infty} \delta(x') \psi (x') d x'= \psi (x')|_{x'=0} = \psi (x)|_{x=x_0} = \psi (x_0)  \]
因此,性质-4也称挑选性。
\end{proof} 

~~\\
2、$\delta(x)$函数有如下性质
$$\begin{aligned}
		&\delta(-x)=\delta(x) \\
		&\delta'( -x ) = - \delta'( x ) \\
		&\delta(\alpha x ) = \frac{\delta( x ) }{\left|\alpha \right|} \\
		&x\delta( x ) = 0 \\
		&x\delta_{\lambda}( x ) = \lambda \delta_{\lambda}( x ) \\
\end{aligned}$$

\begin{proof} 性质-5: $\delta$函数是偶函数 \\
	~对于$$\int_{-\infty}^{+\infty} \delta(-x) \psi (x) d x$$ 令 $-x =x'$, 有 $-dx = dx'$,代入得 
	\[
		\begin{aligned}
			\int_{-\infty}^{+\infty} \delta(-x) \psi (x) d x &= -\int_{+\infty}^{-\infty} \delta(x') \psi (-x') d x' = \int_{-\infty}^{+\infty} \delta(x') \psi (-x') d x' \\
			&=  \psi (-x') | _{x'=0}\\
			&=  \psi (0) \\
			&=  \int_{-\infty}^{+\infty} \delta(x) \psi (x) d x 
		\end{aligned}
		   \]
		得证! \\ 
	性质-6: $\delta$函数的导数是奇函数
	\[
		\begin{aligned}
		  \int_{-\infty}^{+\infty} \delta'(x) \psi (x) d x &= \delta(x) \psi (x)|_{-\infty}^{+\infty} - \int_{-\infty}^{+\infty} \delta(x) \psi' (x) d x \\
			&=  - \int_{-\infty}^{+\infty} \delta(x) \psi' (x) d x \\
			&=  - \psi' (0)
		\end{aligned}\]
	\[
			\begin{aligned}
			  \int_{-\infty}^{+\infty} \delta'(-x) \psi (x) d x &= \int_{+\infty}^{-\infty} \delta'(x') \psi (-x') d (-x') \\
				&= \int_{-\infty}^{+\infty} \delta'(x') \psi (-x') d (x') \\
				&=  \psi' (-x') _{x'=0}\\
				&=  \psi' (0) 
			\end{aligned}
			   \]
	即
	\[\int_{-\infty}^{+\infty} \delta'(x) \psi (x) d x = - \int_{-\infty}^{+\infty} \delta'(-x) \psi (x) d x\]
	得 \[\delta'(-x)   = \delta'(-x) \] 
	推论:\[ \int_{-\infty}^{+\infty} \delta^{(n)}(-x) \psi (x) d x = (-1)^n \psi^{(n)} (0)\]
	性质-7:放缩
	\[\int_{-\infty}^{+\infty} \delta(\alpha x) d x=
	\frac{1}{\alpha}\int_{-\infty}^{+\infty} \delta(x') d x' =
	 \int_{-\infty}^{+\infty} \frac{1}{\alpha}\delta(x) d x\]
	 即
	 \[\delta(\alpha x)  = \frac{1}{\alpha}\delta(x)\] 
	性质-8:在性质-2中
	\[\int_{-\infty}^{+\infty} \delta(x) \psi (x) d x=\psi  (0)\] 
	取$\psi (x) =x$, 得 
	\[ \int_{-\infty}^{+\infty} x\delta(x) d x = 0\]
	定积分为零,有两种可能:一是被积函数是零函数$x\delta(x) = 0$,二是$x\delta(x)$在积分区域出现正负面积相等的情况。现有$x\delta(x)$除在$x=0$之外各点的值都为零,因此不可能出现第二种情况,得:
	\[ x\delta(x) = 0\]
\end{proof}

~~\\
狄拉克函数在不同坐标系下的形式为:
\[
		\begin{aligned}
		  \delta (\vec{r}-\vec{r}_0) &=  \delta (x-x_0) \delta (y-y_0) \delta (z-z_0)\\
		  &=  \frac{1}{r^2\sin\theta}\delta (r-r_0) \delta (\theta -\theta_0) \delta (\varphi -\varphi_0)
		\end{aligned}
		   \]
极坐标系下为 
\[ \delta (\vec{r}-\vec{r}_0) =  \frac{1}{r}\delta (r-r_0) \delta (\theta -\theta_0) \]

\subsection{狄拉克函数的傅里叶变换}
\begin{example}
	试证明$\delta$函数的傅里叶变换公式为
	\[ \boxed{\delta(x) = \frac{1}{2\pi}\int_{-\infty}^{+\infty} e^{-ikx} d k} \]
\end{example}
\begin{proof}
有性质-4
 \[\int_{-\infty}^{+\infty} \delta(x-x_0) \psi (x) d x = \psi (x_0)\]
取 $ \psi (x) = e^{ikx} $, 得
\[\int_{-\infty}^{+\infty} \delta(x-x_0) e^{ikx} d x = e^{ikx_0}\]
取 $ x_0 =0 $, 得
\[\int_{-\infty}^{+\infty} \delta(x) e^{ikx} d x = 1\]
利用傅里叶变换公式,有:
\[ \delta(x) = \frac{1}{2\pi} \int_{-\infty}^{+\infty} 1 \cdot e^{-ikx} d k = \frac{1}{2\pi} \int_{-\infty}^{+\infty} e^{-ikx} d k \]
\textcolor{red}{证毕!} \\
代入波矢与动量关系$$ k = \frac{p_x}{\hbar}$$
得量子力学形式
\begin{equation}\label{eq:qm-1}
\boxed{\delta(x) = \frac{1}{2\pi\hbar} \int_{-\infty}^{+\infty} e^{-\frac{i}{\hbar}p_xx} d p_x }
\end{equation}
\end{proof}

\begin{example}
	试证明动量表象$\delta$函数的傅里叶变换公式为
	\[\delta(p_x) = \frac{1}{2\pi\hbar} \int_{-\infty}^{+\infty} e^{\frac{i}{\hbar} p_x x} d x \]
\end{example}
\begin{proof}
	在式[\ref{eq:qm-1}]中,取$x=-x$, 
	\[ \delta(-x) = \frac{1}{2\pi\hbar} \int_{-\infty}^{+\infty} e^{\frac{i}{\hbar}p_xx} d p_x \]
	又$\delta(-x)=\delta(x)$
	\[ \delta(x) = \frac{1}{2\pi\hbar} \int_{-\infty}^{+\infty} e^{\frac{i}{\hbar}p_xx} d p_x \]
	全空间积分
	\[ 
	\begin{aligned}
	\int_{-\infty}^{+\infty} \delta(x) dx &= \frac{1}{2\pi\hbar} \iint_{-\infty}^{+\infty} e^{\frac{i}{\hbar}p_xx} d p_x dx \\
	 1 &= \frac{1}{2\pi\hbar} \iint_{-\infty}^{+\infty} e^{\frac{i}{\hbar}p_xx}   dp_x dx\\
	\int_{-\infty}^{+\infty} \delta(p_x) dp_x &= \int_{-\infty}^{+\infty} \left[\frac{1}{2\pi\hbar} \int_{-\infty}^{+\infty} e^{\frac{i}{\hbar}p_xx}  dx\right] dp_x 
	\end{aligned}
	\]	
	得
	\[\delta(p_x) = \frac{1}{2\pi\hbar} \int_{-\infty}^{+\infty} e^{\frac{i}{\hbar}p_x x}  dx\]
\end{proof}

\begin{example}
	已知自由粒子的波函数取平面波形式
	$$
	\Psi _{p_x}(x)= e^{\frac{i}{\hbar}p_x x} 
	$$ 
	试求归一化平面波函数
\end{example}
\emph{解:}
	设归一化的波函数为
	$$
	\psi _{p_x}(x)= C e^{\frac{i}{\hbar}p_x x} 
	$$ 
	在动量表象$\delta$函数的傅里叶变换公式中, 以 $p_x - p'_x$ 取得  $p_x$
	\[ 
		\begin{aligned}
			\delta(p_x-p'_x) &= \frac{1}{2\pi\hbar} \int_{-\infty}^{+\infty} e^{\frac{i}{\hbar}(p_x-p'_x) x}  dx\\
		 &= \frac{1}{2\pi\hbar} \int_{-\infty}^{+\infty} e^{\frac{i}{\hbar}p_x x} e^{\frac{-i}{\hbar}(p'_x) x} dx\\
		 &= \frac{1}{C^2 2\pi\hbar} \int_{-\infty}^{+\infty} \psi^* _{p'_x}(x) \psi_{p_x}(x) dx
		\end{aligned}
		\]
	由于$\psi _{p_x}(x)$是属于固体值$p_x$的固有函数, 当$p_x \ne p'_x$时,两固有函数正交, 因此有
	\[ 
		\begin{aligned}
			\delta(p_x-p_x) &= \frac{1}{C^2 2\pi\hbar} \int_{-\infty}^{+\infty} \psi^* _{p_x}(x) \psi_{p_x}(x) dx\\
			1 & = \frac{1}{C^2 2\pi\hbar} \\
			C  & = \frac{1}{\sqrt{2\pi\hbar} }
		\end{aligned}
		\]
	因此,有
	$$
	\psi _{p_x}(x)= \frac{1}{\sqrt{2\pi\hbar}} e^{\frac{i}{\hbar}p_x x} 
	$$


\begin{example}
	设狄利克雷内核函数为
	\[ D_m(x) =  \frac{1}{2\pi}\frac{\sin (m+\frac{1}{2})x}{ \sin\frac{1}{2}x } \]
	试证明它是狄拉克函数的辅助函数,即有
	\[\lim_{m \to \infty} D_m(x) =  \delta(x) \]
\end{example}
 \begin{proof}
	对三角公式,两端同时积分
	\[
	\begin{aligned}
		1+2\sum\limits_{n=1}^{m}\cos x &= \frac{\sin (m+\frac{1}{2})x}{\sin\frac{1}{2}x} \\ 
		\int_{-\pi} ^\pi (1+2\sum\limits_{n=1}^{m}\cos x) d x &= \int_{-\pi} ^\pi \frac{\sin (m+\frac{1}{2})x}{\sin\frac{1}{2}x d x  } \\
		2\pi &= \int_{-\pi} ^\pi \frac{\sin (m+\frac{1}{2})x}{\sin\frac{1}{2}x d x  }
	\end{aligned}	
	\]
	得 \[\int_{-\pi} ^\pi D_m(x) d x =1\]
	$ D_m(x)  $ 是归一化的偶函数,现在只需要证明当 $x \to 0$ 时,函数在$m \to \infty$ 时, 取极大值。
	\[D_m(0) = \frac{1}{2\pi} \lim_{x\to 0}\frac{\sin (m+\frac{1}{2})x}{ \sin\frac{1}{2}x } = \frac{1}{2\pi} (-2+1) \]
	\[\lim_{m \to \infty} D_m(0) = \lim_{m \to \infty} \frac{1}{2\pi} (2m+1) =\infty \]
	一个在包含点 $x=0$的区域里归一化, 在点$x=0$处趋于$\infty$ 的函数,正是 $\delta$ 函数。因此
	\[\lim_{m \to \infty} D_m(x) = \delta(x) \]
 \end{proof}
 

\begin{example}
		试用$\delta$函数证明对于连续函数,傅里叶级数收敛
\end{example}
\begin{proof}
	傅里叶级数为
	\[\dfrac{a_0}{2} +\sum\limits_{n=1}^{\infty}  \left(  a_n \cos~ nx +  b_n \sin~ n x  \right) \]
	部分和为
	\[S_m(x) = \dfrac{a_0}{2} +\sum\limits_{n=1}^{m}  \left(  a_n \cos~ n x +  b_n \sin~ n x  \right) \]
	代入系数,整理得
	\[ \begin{aligned} S_n(x) & = \frac{1}{2\pi}\int_{-\pi}^\pi f(t)dt + \frac{1}{\pi}\sum\limits_{n=1}^{m} \int_{-\pi}^\pi f(t) \cos~ (n(x -t)) dt  \\
		&= \frac{1}{2\pi}\int_{-\pi}^\pi f(t)dt + \frac{1}{\pi} \int_{-\pi}^\pi f(t) \sum\limits_{n=1}^{m}\cos~n (x -t) dt \\
		&= \frac{1}{2\pi}\int_{-\pi}^\pi f(t) \left[ 1+ 2\sum\limits_{n=1}^{m}\cos~n (x -t) \right] dt
	\end{aligned}\]
	代入三角公式
	\[1+2\sum\limits_{n=1}^{m}\cos x = \frac{\sin (m+\frac{1}{2})x}{\sin\frac{1}{2}x}\]
	得
	\[ \begin{aligned} S_m(x) & =\int_{-\pi}^\pi f(t) \frac{\sin (m+\frac{1}{2})(x -t)}{2\pi \sin\frac{1}{2}(x -t) } dt \\
	&=\int_{-\pi}^\pi f(t) D_m(x-t) \mathrm{d} t	
	\end{aligned}\]
	考虑 $m \to \infty$, 
	\[ \begin{aligned} \lim_{m \to \infty} S_m(x) &= \int_{-\pi}^\pi f(t) \delta (x-t) \mathrm{d} t  \\
		&= f(x)
	 \end{aligned}\]
	说明在$f(x)$连续的所有点,傅里叶级数收敛。 对于不连续的一类断点或极值点(略)。\\
	\textcolor{red}{证毕!}
\end{proof}

\begin{Exercises}
	\item 利用$\Gamma$函数的递推公式,计算 
		$$\Gamma(\frac{1}{2}), \quad  \Gamma(-\frac{1}{2}), \quad \Gamma(-\frac{3}{2}), \quad (\frac{5}{2})!  \quad (-\frac{5}{2})! $$ 
	\item 利用$\Gamma$函数证明
		\[0! =1 \] 
	\item 利用$\Gamma$函数,计算积分
		\[\int_{0}^{\infty} x^2 e^{-\frac{1}{2}x^2} dx \]
	\item 试证明
	\begin{equation*}
		J_{1/2} (x) =\sqrt{\frac{2}{\pi x}} \sin x,  \qquad  J_{-1/2} (x) =\sqrt{\frac{2}{\pi x}} \cos x
	\end{equation*} 
		\item 试证明
		\[ J_{-n}(x) = (-1)^n J_{n}(x), \quad [x J_{-1}(x)]' = -x  J_{0}(x) \]
		\item 试证明
		\[\frac{d}{d x}\left[x^{-n} J_{n}(x)\right] = -x^{-n} J_{n+1}(x)\] 
		\item 求函数的值
		\[ \Gamma(0),\qquad  J_{\frac{3}{2}}(0)\]
		\item 试证明正交性公式
	    \[\int_{0}^{+\infty} J_n(kr)J_n(k'r) r dr = \frac{\delta(k-k')}{k}, \qquad (n\geq-1; k, k'>0)\]
		\item 试证明
		\begin{equation*}
			\Gamma(1/2)=\sqrt{\pi}, \qquad J_{-1/2}(x) = \sqrt{\frac{2}{\pi x}} \cos x  
		\end{equation*}
		\item 试证明零点递推公式  \[ J'_n(\mu_m ^n)= - J_{n+1}(\mu_m ^n)\]
		\item 设函数$f(r)$的贝塞尔展开式为
		\begin{equation*}
			f(r)=\sum_{m=1}^\infty c_m J_n(\frac{\mu_m ^n}{r_0} r)
		\end{equation*}	
		试证明其展开系数为: 
		\begin{equation*}
			c_m=\frac{2} {r^2_0 J_{n+1} ^2 (\mu_m ^n)} \int_0 ^{r_0} f(r) J_n(\frac{\mu_m ^n}{r_0} r) r dr 
		\end{equation*}	
		\item 求解圆域波动方程
		$$\begin{cases}
			\displaystyle u_{tt} = \frac{\partial^2 u }{\partial r^2 } +\frac{1}{r } \frac{\partial u }{\partial r }, ~~~~ (0<r<1, 0<\theta<2\pi, t>0) \\
			u(r,\theta)|_{r=1}=0 	\\
			u(r,\theta,t)|_{t=0} =\cos^2\theta
		\end{cases} $$
		\item 求解圆域热传导方程
		$$\begin{cases}
			\displaystyle u_t= \frac{1}{r^2 } \frac{\partial ^2 u }{\partial \theta ^2
			}, ~~~~ (0<r<R, 0<\theta<2\pi, t>0)\\
			u(r,\theta)|_{r=R}=0 	\\
			u(r,\theta,t)|_{t=0} = \sin 3\theta 	
		\end{cases} $$
		\item 已知半径为$R_0$的球体表面电势为$u_0(\theta)$,求表面电荷的密度分布函数。
		\item 试证明
	 \[
	  \int_{-\infty}^{+\infty} 1 \cdot e^{2\pi i kx} d x = \delta(k)
	 \]
		\item 试证明\[\delta(p_x-p'_x) = \frac{1}{2\pi\hbar} \int_{-\infty}^{+\infty} e^{\frac{i}{\hbar}(p_x-p'_x) x}  dx\]
		\item 解方程	
		\begin{equation*}
			\left[
			\frac{1}{\sin \theta  } \frac{\partial }{\partial \theta } (\sin \theta \frac{\partial }{\partial \theta } )
			+\frac{1}{ \sin^2 \theta  } \frac{\partial^2}{\partial\varphi ^2}\right]  u (\theta, \varphi) +\lambda u (\theta, \varphi) =0 
		\end{equation*}
		\item 试用级数法求解连带拉盖方程
		\begin{equation*}
			x L''  + (m+1 -x) L' +n L =0
		\end{equation*}	
		\item 试写出$f(\varphi) = \cos ^2 \varphi$  在函数系$\{ \Phi _m(\varphi)\}$的展开式。
		\item 求函数$f(x) = 3x^3+x^2+1$的勒让德展开式, 并计算积分  $$ \int_{-1}^{+1} (x^3 -x +1) P_{n}(x) dx, \quad \int_{-1}^{1} (x^2+x) P_l(x) dx, \quad \int_{-1}^{1} x^k P_l(x) dx \quad(k<l), \quad  	\int_{-1}^{1} x^l P_l(x) dx $$ 
		\item 试证明勒让德多项式递推公式
		\begin{equation*}
			(n+1) P_{n+1}(x)-(2 n+1) x P_{n}(x)+n P_{n-1}(x)=0
		\end{equation*}
		\item 试用勒让德多项式递推公式证明
		\begin{equation*}
			\frac{2n+1}{2}\int_{-1}^{1}  P ^2 _{n} dx = 1 
	\end{equation*}	
		\item 试证明 	
	\[P _{n}(-x) =(-1)^n P _{n}(x),\quad P _{n}(-1) =(-1)^n, \quad P'_{n}(-1) = \frac{n(n+1)}{2}(-1)^{n+1} \] 
		\item 求下列连带勒让德多项式的具体表达式
	$$P^1 _{1}(x),\quad P^1 _{2}(x),\quad P^2 _{2}(x),\quad P^1 _{3}(x),\quad P^2 _{3}(x),\quad  P^3 _{3}(x)$$
		\item 求函数$f(x) = 3x^3+x^2+1$的连带勒让德多项式展开式 \\
		\item 计算积分
		\begin{equation*}
		\int_{-1}^{+1} (x^3 -x +1) P^l _{n}(x) dx 
		\end{equation*}
		\item 试证明
		\[P^m_l(-x) = (-1)^{l+m} P^m_l(x), \quad P^{-m}_l(x) = (-1)^m \frac{(l-m)!}{(l+m)!} P^{m}_l (x) \]
		\item 试证明
		\[ Y^*_{lm}(\theta, \varphi) = (-1)^m Y_{l, -m}(\theta, \varphi), \quad Y_{\ell}^0(\theta, \phi)=\sqrt{\frac{2 \ell+1}{4 \pi}} P_{\ell}(\cos \theta)\]
		\item  证明递推关系式$$
		(2 \ell+1)\left(1-x^2\right)^{1 / 2} P_{\ell}^m=P_{\ell+1}^{m+1}-P_{\ell-1}^{m+1}
		$$
		\item 令 $$
		\hat{L}_{\pm}=e^{i \phi}\left(\pm\frac{\partial}{\partial \theta}+i \frac{\cos \theta}{\sin \theta} \frac{\partial}{\partial \phi}\right)
		$$
		试证明 $$
		\hat{L}_{\pm} Y_{\ell}^m=\alpha_{\ell}^m Y_{\ell}^{m \pm 1}
		$$
		\item  求类拉盖方程
		$$
       \frac{\mathrm{d}^2 U}{\mathrm{~d} \xi^2}+\frac{2}{\xi} \frac{\mathrm{d} U}{\mathrm{~d} \xi}+\left(\frac{\beta}{\xi}-\frac{l(l+1)}{\xi^2}\right) U=0
       $$ 
\item 写出 $L_1 (x)$ 和 $L_1 ^0 (x)$ 之间的关系式 
\item 计算积分:
	\begin{equation*}
		\int_{0}^{\infty}   e^{-x} ( L_1 (x) )^2 dx, \qquad  \int_{0}^{\infty}   e^{-x} ( L_2 (x) )^2 dx, \qquad \int_{0}^{\infty}  e^{-x} L _{n}(x)L _{m}(x)  dx  
	\end{equation*}	
\item 求函数$ f(x) = 3x^3+1 $ 的拉盖多项式展开式, 并计算积分
		\[\int_{0}^{+\infty} e^{-x}f(x)L_n(x) dx \]
\item 试证明
		$$
		x L_n^{\prime}(x)=n L_n(x)-n L_{n-1}(x), \quad  L^m_n (x) = (-1)^m \frac{\mathrm{d}^m}{\mathrm{d}x^m} L_{n+m}(x)
		$$
\end{Exercises}
~~\\


