\chapter{偏微分方程}
物理量(u)通常是位置 (x)和时间(t)的函数, 因此是个多元函数$u(x,t)$。描述它变化规律的方程通常含有未知函数$u(x,t)$及其偏分(导),称为\emph{偏微分方程}。

描述物理模型的微分方程中,以二阶常系数线性偏微分方程最为常见,一般式为:
\begin{equation}
	au_{xx}+2bu_{xt}+cu_{tt}+du_x+eu_t+fu=g(x,t)
\end{equation}
它的特征方程为
\begin{equation*}
	ax_{tt}-2bx_{t}+c=0
\end{equation*}
根据$\Delta=b^2-ac$的不同,二阶常系数线性偏微分方程分为三大类:
\begin{itemize}
	\item 双曲型: $\Delta>0$
	\item 抛物型: $\Delta=0$
	\item 椭圆型: $\Delta<0$
\end{itemize}
双曲型的代表是波动方程,抛物型的代表是热传导方程,椭圆型的代表是拉普拉斯和泊松方程。它们的具体形式如下:
\begin{itemize}
	\item 波动方程: \begin{equation}u_{tt} = a^2 \nabla ^2 u \end{equation}
	\item 热传导方程: \begin{equation}u_t = a^2 \nabla ^2 u \end{equation}
	\item 拉普拉斯方程: \begin{equation}\nabla ^2 u = 0 \end{equation}	
\end{itemize} 
式中,$\nabla $是\emph{矢量微分算子}, $\nabla^2$是拉普拉斯算子。

基于矢量微分算子,对标量函数可定义\emph{梯度}$\nabla u$,描述其空间分布的非均匀性。对矢量函数可定义\emph{散度}$\nabla \cdot \mathbf{A} $和\emph{旋度} $\nabla \times \mathbf{A} $。
散度描述“源”与“漏”,旋度描述“旋转”情况,如图[\ref{fig:div}]所示。
	\begin{figure}[h]
		\centering
		\includegraphics[width=0.5\textwidth]{figs/div1.png}
		\caption{散度与旋度}
		\label{fig:div}
	\end{figure}

拉普拉斯(Laplace)算子是标量函数梯度场的散度,
	\begin{equation}
		\nabla ^2 = \nabla \cdot \nabla
	\end{equation}
描述非均匀性的源由。
	
矢量微分算子$\nabla$ 和拉普拉斯算子$\nabla^2$在各种偏微分方程中普遍出现,我们在具体解方程前,先明确它们在各种坐标系中的具体表示。

\section{常用算子}
\subsection{微分算子和拉普拉斯算子} ~\\

微分算子在不同坐标系下的具体表示为
\begin{itemize}
	\item 直角坐标 \begin{equation}\nabla=\frac{\partial}{\partial x} \vec{e}_x+\frac{\partial}{\partial y} \vec{e}_y+\frac{\partial}{\partial z} \vec{e}_z
	\end{equation}
	\item 极坐标系 \begin{equation}
	  \nabla =\vec{e}_{r} \frac{\partial}{\partial r}+\frac{1}{r} \vec{e}_{\theta} \frac{\partial}{\partial \theta}
	\end{equation}
	\item 柱面坐标系\begin{equation}
	  \nabla =\vec{e}_{\rho} \frac{\partial}{\partial \rho}+\frac{1}{\rho} \vec{e}_{\theta} \frac{\partial}{\partial \theta} + \vec{e}_{z} \frac{\partial}{\partial z}
	\end{equation}
	\item 球坐标系
	\begin{equation}
		\nabla =\vec{e}_{r} \frac{\partial}{\partial r}+\frac{1}{r} \vec{e}_{\theta} \frac{\partial}{\partial \theta}+\frac{1}{r \sin \theta} \vec{e}_{\varphi} \frac{\partial}{\partial \varphi}
	\end{equation}
	\end{itemize}
拉普拉斯算子在不同坐标系下的形式 
\begin{itemize}
	\item 直角坐标 \begin{equation}  \nabla ^{2}  = \frac{\partial ^2}{\partial x^2} +\frac{\partial^2 }{\partial y^2}+\frac{\partial^2 }{\partial z^2}\end{equation}
	\item 极坐标系  \begin{equation}\nabla ^{2} =\frac{\partial ^2 }{\partial r^2 } +\frac{1}{r } \frac{\partial }{\partial r } +
	\frac{1}{r^2 } \frac{\partial ^2 }{\partial \theta ^2}  \end{equation}
	\item 柱面坐标系\begin{equation}
	\nabla^2=\frac{\partial^2}{\partial \rho^2}+\frac{1}{\rho} \frac{\partial}{\partial \rho}+\frac{1}{\rho^2} \frac{\partial^2}{\partial \phi^2}+\frac{\partial^2}{\partial z^2}
\end{equation}
	\item 球坐标系
	\begin{equation} \nabla ^{2} =\frac{1}{r^2} \frac{\partial }{\partial r} (r^2\frac{\partial }{\partial r} )+
	\frac{1}{r^2 \sin \theta  } \frac{\partial }{\partial \theta } (\sin \theta \frac{\partial }{\partial \theta } )
	+\frac{1}{r^2 \sin^2 \theta  } \frac{\partial^2}{\partial\varphi ^2}\end{equation}
\end{itemize}
\begin{example}
	试证明直角坐标系拉普拉斯算子为
	$$\displaystyle  \nabla ^{2}  = \frac{\partial ^2}{\partial x^2} +\frac{\partial^2 }{\partial y^2}+\frac{\partial^2 }{\partial z^2}$$
\end{example}

\begin{proof}
	把直角坐标系矢量微分算子写成矩阵形式
	$$
\nabla=\frac{\partial}{\partial x} \vec{e}_x+\frac{\partial}{\partial y} \vec{e}_y+\frac{\partial}{\partial z} \vec{e}_z
 = \begin{pmatrix}
	\dfrac{\partial}{\partial x}\\
	\dfrac{\partial}{\partial y}\\
	\dfrac{\partial}{\partial z}
   \end{pmatrix}$$
   根据拉普拉斯算子定义,有
   $$
   \begin{aligned}
	 \nabla ^2 = \nabla \cdot \nabla  
	 = \begin{pmatrix}
		\dfrac{\partial}{\partial x}&
		\dfrac{\partial}{\partial y} &
		\dfrac{\partial}{\partial z}
	   \end{pmatrix} \begin{pmatrix}
		\dfrac{\partial}{\partial x}\\
		\dfrac{\partial}{\partial y}\\
		\dfrac{\partial}{\partial z}
	   \end{pmatrix} 
	= \frac{\partial ^2}{\partial x^2} +\frac{\partial^2 }{\partial y^2}+\frac{\partial^2 }{\partial z^2}
   \end{aligned}$$
\end{proof}
平面直角坐标系中拉普拉斯算子取二维形式   
$$\displaystyle  \nabla ^{2}  = \frac{\partial ^2}{\partial x^2} +\frac{\partial^2 }{\partial y^2}$$
~~\\ 

\begin{example}
 试证明极坐标系中拉普拉斯算子为 
 $$\nabla ^{2} =\frac{\partial ^2 }{\partial r^2 } +\frac{1}{r } \frac{\partial }{\partial r } +
	\frac{1}{r^2 } \frac{\partial ^2 }{\partial \theta ^2} $$
	\begin{figure}[h]
		\centering
		\includegraphics[width=0.45\textwidth]{figs/polar.png}
		\caption{平面直角坐标与极坐标关系图}
		\label{fig:polar}
	\end{figure}
\end{example}

\begin{proof}
    如图[\ref{fig:polar}]所示,平面直角坐标与极坐标存在换算关系,
	\begin{equation}\label{eq:polar1}
		\begin{cases}
		x= r\cos \theta  \\
		y= r\sin \theta 
		\end{cases} 
		\end{equation}
	分别就$x,y$对$r,\theta$求导, 得
	$$ \dfrac{\partial x}{\partial r} = \cos\theta , \quad \dfrac{\partial y}{\partial r} = \sin \theta $$
	$$ \dfrac{\partial x}{\partial \theta} = -r\sin \theta , \quad \dfrac{\partial y}{\partial \theta}  = r\cos \theta $$
	写成矩阵形式
	\begin{equation}\label{eq:laplace3}
		\left[\begin{array}{ccc}
		\dfrac{\partial x}{\partial r} & \dfrac{\partial y}{\partial r} \\ 
		\dfrac{\partial x}{\partial \theta} & \dfrac{\partial y}{\partial \theta}  
	\end{array}\right]
	= \left[\begin{array}{ccc}
		\cos\theta  & \sin \theta \\ 
		-r\sin \theta  & r\cos \theta 
	\end{array}\right]
\end{equation}
	把函数$u(x,y)$对 $r, \theta $链式求导,有:
	$$\begin{aligned}
		\dfrac{\partial {u}}{\partial {r}}&=\dfrac{\partial {u}}{\partial x} \dfrac{\partial {x}}{\partial {r}}+\dfrac{\partial u}{\partial {y}} \dfrac{\partial {y}}{\partial {r}} \\ 
		\dfrac{\partial {u}}{\partial \theta}&=\dfrac{\partial {u}}{\partial x} \dfrac{\partial{x}}{\partial \theta}+\dfrac{\partial u}{\partial {y}} \dfrac{\partial {y}}{\partial \theta} 
	\end{aligned} $$
	写成矩阵形式 
	$$\left[\begin{array}{ccc}
		\dfrac{\partial u}{\partial r} \\ 
		\dfrac{\partial u}{\partial \theta} 
	\end{array}\right]
	=
	\left[\begin{array}{ccc}
		\dfrac{\partial x}{\partial r} & \dfrac{\partial y}{\partial r} \\ 
		\dfrac{\partial x}{\partial \theta} & \dfrac{\partial y}{\partial \theta}  
	\end{array}\right]
	\left[\begin{array}{ccc}
		\dfrac{\partial u}{\partial x} \\ 
		\dfrac{\partial u}{\partial y} 
	\end{array}\right]$$
    把[\ref{eq:laplace3}]代入上式,得
	$$\left[\begin{array}{ccc}
		\dfrac{\partial u}{\partial r} \\ 
		\dfrac{\partial u}{\partial \theta} 
	\end{array}\right]
	=
	\left[\begin{array}{ccc}
		\cos\theta  & \sin \theta \\ 
		-r\sin \theta  & r\cos \theta 
	\end{array}\right]
	\left[\begin{array}{ccc}
		\dfrac{\partial u}{\partial x} \\ 
		\dfrac{\partial u}{\partial y} 
	\end{array}\right]$$
右边的变换阵写成两个可逆矩阵相乘的形式
	$$\left[\begin{array}{ccc}
		\dfrac{\partial u}{\partial r} \\ 
		\dfrac{\partial u}{\partial \theta} 
	\end{array}\right]
	=
	\left[\begin{array}{ccc}
		1 & 0  \\ 
		0 & r   
	\end{array}\right]
	\left[\begin{array}{ccc}
		\cos\theta  & \sin \theta  \\ 
		-\sin \theta & \cos\theta 
	\end{array}\right]
	\left[\begin{array}{ccc}
		\dfrac{\partial u}{\partial x} \\ 
		\dfrac{\partial u}{\partial y} 
	\end{array}\right]$$
	等式两边依次同乘逆矩阵,得
	$$ \left[\begin{array}{ccc}
			\cos \theta   & -\sin \theta \\
			\sin \theta  &  \cos \theta 
		\end{array}\right]
		\left[\begin{array}{ccc}
			\dfrac{\partial u}{\partial r} \\
			\dfrac{1}{r}\dfrac{\partial u}{\partial \theta} 
		\end{array}\right]
		=
		\left[\begin{array}{ccc}
			\dfrac{\partial u}{\partial x} \\
			\dfrac{\partial u}{\partial y} 
		\end{array}\right]
		$$
		变换矩由极坐标系的基矢构成
		$$\vec{e}_r = \left[\begin{array}{ccc}
			\cos \theta  \\
			\sin \theta 
		\end{array}\right], \quad  
		\vec{e}_\theta = \left[\begin{array}{ccc}
			-\sin \theta \\
			\cos \theta 
		\end{array}\right]
		$$
		有
		$$ 
		\left[\begin{array}{ccc}
			\dfrac{\partial u}{\partial x} \\
			\dfrac{\partial u}{\partial y} 
		\end{array}\right]
		=
		\left[\begin{array}{ccc}
			{\vec{e}_r}&  {\vec{e}_\theta} 
		\end{array}\right]
		\left[\begin{array}{ccc}
			\dfrac{\partial u}{\partial r} \\
			\dfrac{1}{r}\dfrac{\partial u}{\partial \theta} 
		\end{array}\right]
		$$
		等式左边是矢量微分算子的矩阵形式, 因此
	\begin{equation*}
		\nabla u
		=
		\left[\begin{array}{ccc}
			{\vec{e}_r}&  {\vec{e}_\theta} 
		\end{array}\right]
		\left[\begin{array}{ccc}
			\dfrac{\partial u}{\partial r} \\
			\dfrac{1}{r}\dfrac{\partial u}{\partial \theta} 
		\end{array}\right]
		\end{equation*}
		矩阵相乘,得极坐标系下的矢量微分算子
		$$
		\nabla  =\vec{e}_{r} \frac{\partial}{\partial r}+\frac{1}{r} \vec{e}_{\theta} \frac{\partial}{\partial \theta}
		$$ 
	极坐标系的拉普拉斯算符
	\begin{equation*}
		\begin{split}
			\nabla ^2&=\nabla \cdot \nabla \\
			&=(\vec{e}_{r} \frac{\partial}{\partial r}+\frac{1}{r} \vec{e}_{\theta} \frac{\partial}{\partial \theta})  \cdot (\vec{e}_{r} \frac{\partial}{\partial r}+\frac{1}{r} \vec{e}_{\theta} \frac{\partial}{\partial \theta}) \\
			&= \vec{e}_{r} \frac{\partial}{\partial r} [\vec{e}_{r} \frac{\partial}{\partial r}] + \vec{e}_{r} \frac{\partial}{\partial r} [\frac{1}{r} \vec{e}_{\theta} \frac{\partial}{\partial \theta}] + \frac{1}{r}\vec{e}_{\theta} \frac{\partial}{\partial \theta} [\vec{e}_{r} \frac{\partial}{\partial r}]  + \frac{1}{r} \vec{e}_{\theta} \frac{\partial}{\partial \theta} [\frac{1}{r} \vec{e}_{\theta} \frac{\partial}{\partial \theta}] \\
			&= \vec{e}_{r} [\frac{\partial}{\partial r} \vec{e}_{r} \frac{\partial}{\partial r} +  \vec{e}_{r}\frac{\partial}{\partial r} \frac{\partial}{\partial r}] 
			+ \vec{e}_{r} [\frac{\partial}{\partial r} \frac{1}{r} \vec{e}_{\theta} \frac{\partial}{\partial \theta} + \frac{1}{r} \frac{\partial}{\partial r} \vec{e}_{\theta} \frac{\partial}{\partial \theta} + \frac{1}{r} \vec{e}_{\theta}\frac{\partial}{\partial r}  \frac{\partial}{\partial \theta} ] \\
			~ & \hspace{1em}+ \frac{1}{r}\vec{e}_{\theta} [\frac{\partial}{\partial \theta} \vec{e}_{r} \frac{\partial}{\partial r} + \vec{e}_{r} \frac{\partial}{\partial \theta} \frac{\partial}{\partial r}] + \frac{1}{r} \vec{e}_{\theta} [\frac{\partial}{\partial \theta} \frac{1}{r} \vec{e}_{\theta} \frac{\partial}{\partial \theta} + \frac{1}{r} \frac{\partial}{\partial \theta} \vec{e}_{\theta} \frac{\partial}{\partial \theta} + \frac{1}{r} \vec{e}_{\theta}\frac{\partial}{\partial \theta} \frac{\partial}{\partial \theta}] \\
		\end{split}
		\end{equation*}	
	利用极坐标系基矢的微分关系
	$$\left[\begin{array}{ccc}
		\dfrac{\partial \vec{e}_{r}}{\partial r} &   \dfrac{\partial \vec{e}_{\theta}}{\partial r} \\ 
		\dfrac{\partial \vec{e}_{r}}{\partial \theta} &  \dfrac{\partial \vec{e}_{\theta}}{\partial \theta} 
	\end{array}\right]
	= \left[\begin{array}{ccc}
		0 &  0 \\ 
		\vec{e}_{\theta} & -\vec{e}_{r} \\ 
	\end{array}\right]
	$$
	得
	\begin{equation*}
		\begin{split}
			\nabla ^2
			&= \vec{e}_{r} [0 +  \vec{e}_{r}\frac{\partial^2}{\partial r^2}] 
			+ \vec{e}_{r} [-\frac{1}{r^2} \vec{e}_{\theta} \frac{\partial}{\partial \theta} + 0 + \frac{1}{r} \vec{e}_{\theta}\frac{\partial^2}{\partial r \partial \theta}  ] \\
			~ & \hspace{1em}+ \frac{1}{r}\vec{e}_{\theta} [ \vec{e}_{\theta} \frac{\partial}{\partial r} + \vec{e}_{r} \frac{\partial ^2}{\partial \theta \partial r}] + \frac{1}{r} \vec{e}_{\theta} [0 -\frac{1}{r} \vec{e}_{r} \frac{\partial}{\partial \theta} + \frac{1}{r} \vec{e}_{\theta}\frac{\partial ^2}{\partial \theta ^2}] 
		\end{split}
		\end{equation*}	
	利用极坐标系基矢的正交归一性
	\begin{equation*}
		\vec{e}_{i} \cdot \vec{e}_{j} = \delta_{ij}, \quad (i,j = r,\theta )
	\end{equation*}
	得
	$$
	\nabla ^2 = \frac{\partial ^2 }{\partial r^2 } +\frac{1}{r } \frac{\partial }{\partial r } +
				\frac{1}{r^2 } \frac{\partial ^2 }{\partial \theta ^2 } 
	$$ 
   \end{proof}
	
	~~\\ 
	${\color{red}\star}~$ 下面证明极坐标系基矢的正交归一性和微分关系\\
	$\bullet$ 
	(1)正交性
	$$\begin{aligned}
		\vec{e}_r \cdot \vec{e}_\theta  &= \left[\begin{array}{ccc}
			\cos \theta  &
			\sin \theta 
		\end{array}\right] \left[\begin{array}{ccc}
			-\sin \theta \\
			\cos \theta 
		\end{array}\right]\\ 
		&= - \cos \theta \sin \theta + \sin \theta \cos \theta \\
		&=0
	\end{aligned}
	$$ 
	$\bullet$ 
	(2)归一性
	$$\begin{aligned}
		\vec{e}_r \cdot \vec{e}_r  &= \left[\begin{array}{ccc}
			\cos \theta  &
			\sin \theta 
		\end{array}\right] \left[\begin{array}{ccc}
			\cos \theta \\
			\sin \theta 
		\end{array}\right]\\ 
		&= \cos^2 \theta + \sin ^2 \theta =1
	\end{aligned}
	$$ 
	同理,有 
	$$\begin{aligned}
		\vec{e}_\theta \cdot \vec{e}_\theta  &= \left[\begin{array}{ccc}
			-\sin \theta  &
			\cos \theta
		\end{array}\right] \left[\begin{array}{ccc}
			-\sin \theta \\
			\cos \theta 
		\end{array}\right]\\ 
		&= \sin ^2 \theta+\cos^2 \theta   =1
	\end{aligned}$$ 
	$\bullet$
	(3)微分关系 \\
	 $\vec{e}_{r}, \vec{e}_{\theta}$不显含$r$, 因此对$r$的导数为零
	 $$ \dfrac{\partial \vec{e}_{r}}{\partial r} =0 \qquad \dfrac{\partial \vec{e}_{\theta}}{\partial r} =0  $$
	第三项
	$$
	\begin{aligned}
		\dfrac{\partial \vec{e}_{r}}{\partial \theta} = \dfrac{\partial }{\partial \theta} \left[\begin{array}{ccc}
			\cos \theta  \\
			\sin \theta 
		\end{array}\right]  
		= \left[\begin{array}{ccc}
			-\sin \theta  \\
			\cos\theta 
		\end{array}\right] 
		= \vec{e}_{\theta} 
	\end{aligned} 
	$$ 
	第四项
	$$
	\dfrac{\partial \vec{e}_{\theta}}{\partial \theta}  = \dfrac{\partial }{\partial \theta} \left[\begin{array}{ccc}
		-\sin \theta  \\
		\cos\theta 
	\end{array}\right] = - \left[\begin{array}{ccc}
		\cos \theta  \\
		\sin \theta 
	\end{array}\right]= -\vec{e}_{r}
	$$ 
	写成矩阵形式
	$$\left[\begin{array}{ccc}
		\dfrac{\partial \vec{e}_{r}}{\partial r} &   \dfrac{\partial \vec{e}_{\theta}}{\partial r} \\ 
		\dfrac{\partial \vec{e}_{r}}{\partial \theta} &  \dfrac{\partial \vec{e}_{\theta}}{\partial \theta} 
	\end{array}\right]
	= \left[\begin{array}{ccc}
		0 &  0 \\ 
		\vec{e}_{\theta} & -\vec{e}_{r} \\ 
	\end{array}\right]
	$$
	\textcolor{red}{证毕!}
  

 \begin{example}
 试根据极坐标系拉普拉斯算符写出柱面坐标系下的拉普拉斯算符
\end{example}
\emph{解:}
 直角坐标与柱面坐标有如下换算关系,
 \begin{equation}\label{eq:cylindrical}
\begin{cases}
	x= \rho\cos \varphi  \\
	y= \rho\sin \varphi  \\
	z= z
\end{cases} 
\end{equation}
与平面直角坐标/柱面坐标换算关系式[\ref{eq:polar1}]相比较,发现在x-y平面部分,它们在数学上完全相同。因此, 柱面坐标系矢量微分算子的$x-y$平面部分为
$$
\vec{e}_{\rho} \frac{\partial}{\partial \rho}+\frac{1}{\rho} \vec{e}_{\theta} \frac{\partial}{\partial \theta}
$$ 
z轴方向应与直角坐标系的相同
$$\vec{e}_{z} \frac{\partial}{\partial z} $$
由正交性,得柱面坐标系矢量微分算子
$$\nabla = \vec{e}_{\rho} \frac{\partial}{\partial \rho}+\frac{1}{\rho} \vec{e}_{\theta} \frac{\partial}{\partial \theta} + \vec{e}_{z} \frac{\partial}{\partial z} $$
同理,得柱面坐标系拉普拉斯算符 
  $$\nabla^2=\frac{\partial^2}{\partial \rho^2}+\frac{1}{\rho} \frac{\partial}{\partial \rho}+\frac{1}{\rho^2} \frac{\partial^2}{\partial \varphi^2}+\frac{\partial^2}{\partial z^2}$$
\begin{example}
	试证明球坐标系拉普拉斯算符为
	$$ \displaystyle \nabla ^{2} =\frac{1}{r^2} \frac{\partial }{\partial r} (r^2\frac{\partial }{\partial r} )+
	\frac{1}{r^2 \sin \theta  } \frac{\partial }{\partial \theta } (\sin \theta \frac{\partial }{\partial \theta } )
	+\frac{1}{r^2 \sin^2 \theta  } \frac{\partial^2}{\partial\varphi ^2} $$  
   \end{example}	
   \begin{figure}[h]
	\centering
	\includegraphics[width=0.5\textwidth]{figs/2022-12-22-22-31-47.png}
	\caption{直角坐标系与球坐标系关系图}
	%\label{fig:}
\end{figure}
   \begin{proof}
    直角坐标与球坐标有如下换算关系
	\begin{equation}
		\begin{cases}
			x= r\sin \theta \cos \varphi \\
			y= r\sin \theta \sin \varphi \\
			z= r\cos \theta
		\end{cases} 	
	\end{equation}
	分别就$x,y,z$对$r,\theta, \varphi$求导,得
	\begin{equation}\label{eq:laplace5}
	\left[\begin{array}{ccc}
		\dfrac{\partial x}{\partial r} & \dfrac{\partial y}{\partial r} & \dfrac{\partial z}{\partial r}\\ 
		\dfrac{\partial x}{\partial \theta} & \dfrac{\partial y}{\partial \theta} & \dfrac{\partial z}{\partial \theta}  \\ 
		\dfrac{\partial x}{\partial \varphi} & \dfrac{\partial y}{\partial \varphi} & \dfrac{\partial z}{\partial \varphi} 
	\end{array}\right]
	= \left[\begin{array}{ccc}
		\sin \theta \cos \varphi & \sin \theta \sin \varphi & \cos \theta\\ 
		r\cos \theta \cos \varphi & r\cos \theta \sin \varphi  & -r\sin \theta  \\ 
		-r\sin \theta \sin \varphi & r\sin \theta \cos \varphi & 0 
	\end{array}\right]
    \end{equation}
	对函数$u(x,y,z)$,做关于 $r, \theta, \varphi $的链式求导,有:
	$$\begin{aligned}
		\dfrac{\partial {u}}{\partial {r}}&=\dfrac{\partial {u}}{\partial x} \dfrac{\partial {x}}{\partial {r}}+\dfrac{\partial u}{\partial {y}} \dfrac{\partial {y}}{\partial {r}}+\dfrac{\partial u}{\partial z} \dfrac{\partial z}{\partial {r}} \\ 
		\dfrac{\partial {u}}{\partial \theta}&=\dfrac{\partial {u}}{\partial x} \dfrac{\partial{x}}{\partial \theta}+\dfrac{\partial u}{\partial {y}} \dfrac{\partial {y}}{\partial \theta}+\dfrac{\partial{u}}{\partial z} \dfrac{\partial z}{\partial \theta} \\ 
		\dfrac{\partial {u}}{\partial \varphi}&=\dfrac{\partial u}{\partial x} \dfrac{\partial{x}}{\partial \varphi}+\dfrac{\partial u}{\partial y} \dfrac{\partial y}{\partial \varphi}+\dfrac{\partial u}{\partial z} \dfrac{\partial z}{\partial \varphi}
	\end{aligned} $$
	写成矩阵相乘 
	$$\left[\begin{array}{ccc}
		\dfrac{\partial u}{\partial r} \\ \vspace{0.3em}
		\dfrac{\partial u}{\partial \theta} \\ \vspace{0.3em}
		\dfrac{\partial u}{\partial \varphi}
	\end{array}\right]
	=
	\left[\begin{array}{ccc}
		\dfrac{\partial x}{\partial r} & \dfrac{\partial y}{\partial r} & \dfrac{\partial z}{\partial r}\\ 
		\dfrac{\partial x}{\partial \theta} & \dfrac{\partial y}{\partial \theta} & \dfrac{\partial z}{\partial \theta}  \\ 
		\dfrac{\partial x}{\partial \varphi} & \dfrac{\partial y}{\partial \varphi} & \dfrac{\partial z}{\partial \varphi} 
	\end{array}\right]
	\left[\begin{array}{ccc}
		\dfrac{\partial u}{\partial x} \\ \vspace{0.3em}
		\dfrac{\partial u}{\partial y} \\ \vspace{0.3em}
		\dfrac{\partial u}{\partial z}
	\end{array}\right]$$
把[\ref{eq:laplace5}]代入上式,得
	$$\left[\begin{array}{ccc}
		\dfrac{\partial u}{\partial r} \\ \vspace{0.3em}
		\dfrac{\partial u}{\partial \theta} \\ \vspace{0.3em}
		\dfrac{\partial u}{\partial \varphi}
	\end{array}\right]
	=
	\left[\begin{array}{ccc}
		\sin \theta \cos \varphi & \sin \theta \sin \varphi & \cos \theta \\ \vspace{0.3em}
		r \cos \theta \cos \varphi & r \cos \theta \sin \varphi & -r \sin \theta \\ \vspace{0.3em}
		-r \sin \theta \sin \varphi & r \sin \theta \cos \varphi & 0
	\end{array}\right]
	\left[\begin{array}{ccc}
		\dfrac{\partial u}{\partial x} \\ \vspace{0.3em}
		\dfrac{\partial u}{\partial y} \\ \vspace{0.3em}
		\dfrac{\partial u}{\partial z}
	\end{array}\right]$$
把变换阵写成两矩阵相乘的形式
	$$\left[\begin{array}{ccc}
		\dfrac{\partial u}{\partial r} \\ \vspace{0.3em}
		\dfrac{\partial u}{\partial \theta} \\ \vspace{0.3em}
		\dfrac{\partial u}{\partial \varphi}
	\end{array}\right]
	=
	\left[\begin{array}{ccc}
		1 & 0 & 0 \\ \vspace{0.3em}
		0 & r  & 0 \\ \vspace{0.3em}
		0 & 0 & r \sin \theta 
	\end{array}\right]
	\left[\begin{array}{ccc}
		\sin \theta \cos \varphi & \sin \theta \sin \varphi & \cos \theta \\ \vspace{0.3em}
		\cos \theta \cos \varphi & \cos \theta \sin \varphi & - \sin \theta \\ \vspace{0.3em}
		-\sin \varphi &  \cos \varphi & 0
	\end{array}\right]
	\left[\begin{array}{ccc}
		\dfrac{\partial u}{\partial x} \\ \vspace{0.3em}
		\dfrac{\partial u}{\partial y} \\ \vspace{0.3em}
		\dfrac{\partial u}{\partial z}
	\end{array}\right]$$
	依次乘以逆矩阵,并交换方向,得
	$$\left[\begin{array}{ccc}
			\dfrac{\partial u}{\partial x} \\
			\dfrac{\partial u}{\partial y} \\
			\dfrac{\partial u}{\partial z}
		\end{array}\right]
		=
		\left[\begin{array}{ccc}
			\sin \theta \cos \varphi & \cos \theta \cos \varphi & -\sin \varphi \\
			\sin \theta \sin \varphi &  \cos \theta \sin \varphi &  \cos \varphi \\
			\cos \theta & -\sin \theta & 0
		\end{array}\right]
		\left[\begin{array}{ccc}
			\dfrac{\partial u}{\partial r} \\
			\dfrac{1}{r}\dfrac{\partial u}{\partial \theta} \\
			\dfrac{1}{r \sin \theta}\dfrac{\partial u}{\partial \varphi}
		\end{array}\right]
		$$
		变换矩由球坐标系的基矢构成
		$$\vec{e}_r = \left[\begin{array}{ccc}
			\sin \theta \cos \varphi \\
			\sin \theta \sin \varphi \\
			\cos \theta
		\end{array}\right], \quad  
		\vec{e}_\theta = \left[\begin{array}{ccc}
			\cos \theta \cos \varphi \\
			\cos \theta \sin \varphi \\
			-\sin \theta
		\end{array}\right], \quad  
		\vec{e}_\varphi = \left[\begin{array}{ccc}
			-\sin \varphi \\
			\cos \varphi \\
			0
		\end{array}\right]
		$$
		因此,有
		$$ \nabla u =
		\left[\begin{array}{ccc}
			\dfrac{\partial u}{\partial x} \\
			\dfrac{\partial u}{\partial y} \\
			\dfrac{\partial u}{\partial z}
		\end{array}\right]
		=
		\left[\begin{array}{ccc}
			{\vec{e}_r}&  {\vec{e}_\theta} & {\vec{e}_\varphi}
		\end{array}\right]
		\left[\begin{array}{ccc}
			\dfrac{\partial u}{\partial r} \\
			\dfrac{1}{r}\dfrac{\partial u}{\partial \theta} \\
			\dfrac{1}{r \sin \theta}\dfrac{\partial u}{\partial \varphi}
		\end{array}\right]
		$$
	矢量微分算子在球坐标系的表示为
	\begin{equation*}
		\nabla=\vec{e}_{r} \frac{\partial}{\partial r}+\frac{1}{r} \vec{e}_{\theta} \frac{\partial}{\partial \theta}+\frac{1}{r \sin \theta} \vec{e}_{\varphi} \frac{\partial}{\partial \varphi}
		\end{equation*}
	拉普拉斯算子在球坐标系的表示为
	\begin{equation*}
		\begin{split}
			\nabla ^2&=\nabla \cdot \nabla \\
			&=(\vec{e}_{r} \frac{\partial}{\partial r}+\frac{1}{r} \vec{e}_{\theta} \frac{\partial}{\partial \theta}+\frac{1}{r \sin \theta} \vec{e}_{\varphi} \frac{\partial}{\partial \varphi})  \cdot (\vec{e}_{r} \frac{\partial}{\partial r}+\frac{1}{r} \vec{e}_{\theta} \frac{\partial}{\partial \theta}+\frac{1}{r \sin \theta} \vec{e}_{\varphi} \frac{\partial}{\partial \varphi}) \\
			&=\frac{\partial ^2}{\partial r^2} + ( \frac{1}{r} \frac{\partial}{\partial r} + \frac{1}{r^2} \frac{\partial^2} {\partial \theta ^2}  ) + (\frac{1}{r} \frac{\partial}{\partial r}  + \frac{\cos \theta}{r^2 \sin \theta} \frac{\partial} {\partial \theta }  + \frac{1}{r^2 \sin^2 \theta  } \frac{\partial^2}{\partial\varphi ^2} )\\
			&=\frac{1}{r^2} \frac{\partial }{\partial r} (r^2\frac{\partial }{\partial r} )+
			\frac{1}{r^2 \sin \theta  } \frac{\partial }{\partial \theta } (\sin \theta \frac{\partial }{\partial \theta } )
			+\frac{1}{r^2 \sin^2 \theta  } \frac{\partial^2}{\partial\varphi ^2}
		\end{split}
		\end{equation*}	
	计算中,利用基矢的微分关系
	$$\left[\begin{array}{ccc}
		\dfrac{\partial \vec{e}_{r}}{\partial r} & \dfrac{\partial \vec{e}_{\theta}}{\partial r} & \dfrac{\partial \vec{e}_{\varphi}}{\partial r}\\ 
		\dfrac{\partial \vec{e}_{r}}{\partial \theta} & \dfrac{\partial \vec{e}_{\theta}}{\partial \theta} & \dfrac{\partial \vec{e}_{\varphi}}{\partial \theta}  \\ 
		\dfrac{\partial \vec{e}_{r}}{\partial \varphi} & \dfrac{\partial \vec{e}_{\theta}}{\partial \varphi} & \dfrac{\partial \vec{e}_{\varphi}}{\partial \varphi} 
	\end{array}\right]
	= \left[\begin{array}{ccc}
		0 & 0 & 0 \\ 
		\vec{e}_{\theta} & -\vec{e}_{r}  & 0\\ 
		\sin \theta \vec{e}_{\varphi} & \cos \theta \vec{e}_{\varphi} & -\left[\begin{array}{c} \cos \varphi \\ \sin \varphi \\ 0\end{array}\right] 
	\end{array}\right]
	$$
	以及基矢的正交归一性
	\begin{equation*}
		\vec{e}_{i} \cdot \vec{e}_{j} = \delta_{ij}, \quad (i,j = r,\theta, \varphi )
	\end{equation*}
\end{proof}

\subsection{角动量、动量和位置算子} ~\\
为什么矢量微分算子和标拉普拉斯算子会出现在各种数理方程中呢?为了弄清楚这个问题,考察球坐标拉普拉斯算子 :
	\begin{equation*}
		\nabla ^{2} =\frac{1}{r^2} \left[\frac{\partial }{\partial r} (r^2 \frac{\partial }{\partial r} )+
		\frac{1}{ \sin \theta  } \frac{\partial }{\partial \theta } (\sin \theta \frac{\partial }{\partial \theta } )
		+\frac{1}{\sin^2 \theta  } \frac{\partial^2}{\partial\varphi ^2}\right]
	\end{equation*}	
	定义角向(方)算子
	\begin{equation*}
		\hat{L}^2 = - \left[ \frac{1}{\sin \theta  } \frac{\partial }{\partial \theta } (\sin \theta \frac{\partial }{\partial \theta } )
		+\frac{1}{ \sin^2 \theta  } \frac{\partial^2}{\partial\varphi ^2} \right]
	\end{equation*}	
	定义切向(方)算子
	\begin{equation*}
		\hat{R}^2 =-\frac{\partial }{\partial r} (r^2 \frac{\partial }{\partial r} ) 
	\end{equation*}	
  球坐标拉普拉斯算子化为 :
	\begin{equation*}
		\nabla ^{2} = - \frac{1}{r^2} \left[\hat{R}^2 + \hat{L}^2 \right]
	\end{equation*}	
	定义角动量方算子:($\hbar$是约化普郞克常数)
  \begin{equation*}
	L^2 \hbar^2 = - \left[ \frac{1}{ \sin \theta  } \frac{\partial }{\partial \theta } (\sin \theta \frac{\partial }{\partial \theta } )
	+\frac{1}{ \sin^2 \theta  } \frac{\partial^2}{\partial\varphi ^2} \right]\hbar^2
\end{equation*}	
    角动量是动量的切向分量
	\begin{equation*}
		\hat{p}_ \perp  ^2 =  \frac{\hat{L}^2}{r^2}\hbar^2
	\end{equation*}	
	动量的径向分量
	\[ \hat{p}_r ^2 = \frac{\hat{R}^2}{r^2}  \hbar^2\]
	动量(方)算子: 
	\[
	\begin{aligned}
		\hat{\vec{p}}^2 &= \hat{p}_r ^2 + \hat{p}_ \perp  ^2 \\ 
		&= \hbar^2 \frac{1}{r^2} (\hat{R}^2+\hat{L}^2) \\
		&= - \hbar^2 \nabla ^{2}  
	\end{aligned}
	\]
	写成点积形式
	\[\hat{\vec{p}}\cdot\hat{\vec{p}} = ( -i\hbar \nabla) \cdot ( -i\hbar \nabla)\]
	动量算子与矢量微分算子有如下关系
	\begin{equation}
		\hat{\vec{p}} =-i\hbar \nabla
	\end{equation}
	把变换关系
	$$\begin{cases}
		x= r\sin \theta \cos \varphi \\
		y= r\sin \theta \sin \varphi \\
		z=r\cos \theta
	\end{cases} $$
	~~\\
	写成矩阵
	$$
	\left[\begin{array}{ccc}
		x \\
		y \\
		z
	\end{array}\right]
	=
	\left[\begin{array}{ccc}
		r\sin \theta \cos \varphi  \\
		r\sin \theta \sin \varphi  \\
		r\cos \theta
	\end{array}\right]	
	=
	r \vec{e}_r$$ 
	球坐标系下,位置算子:
	\begin{equation}
		\qquad \hat{\vec{r}}=r \vec{e}_r 
	\end{equation}
	角动量算子(通常也记作$\hat{L}$,比角向算子$\hat{L}$多一个因子$\hbar$):
	\begin{equation*}
		\begin{aligned}
		\hat{L} &= \hat{\vec{r}}\times \hat{\vec{p}} \\
		&= r \vec{e}_r \times \left[-i\hbar  ( \vec{e}_{r} \frac{\partial}{\partial r}+\frac{1}{r} \vec{e}_{\theta} \frac{\partial}{\partial \theta}+\frac{1}{r \sin \theta} \vec{e}_{\varphi} \frac{\partial}{\partial \varphi})\right]	\\ 
		&= -i \hbar   ( \vec{e}_{\varphi} \frac{\partial}{\partial \theta} - \frac{1}{\sin \theta}  \vec{e}_{\theta} \frac{\partial}{\partial \varphi} )
		\end{aligned}
	\end{equation*}
	角动量(方)算子
	\begin{equation*}
		\begin{split}
		\hat{L}	^2 &=   - \hbar ^2  ( \vec{e}_{\varphi} \frac{\partial}{\partial \theta} - \frac{1}{\sin \theta}  \vec{e}_{\theta} \frac{\partial}{\partial \varphi} )  \cdot   ( \vec{e}_{\varphi} \frac{\partial}{\partial \theta} - \frac{1}{\sin \theta}  \vec{e}_{\theta} \frac{\partial}{\partial \varphi} ) \\
		&= - \hbar ^2 (\frac{1}{\sin \theta  } \frac{\partial }{\partial \theta } (\sin \theta \frac{\partial }{\partial \theta } )
		+\frac{1}{\sin^2 \theta  } \frac{\partial^2}{\partial\varphi ^2} ) 	
		\end{split}
	\end{equation*}
	\textcolor{red}{结束!} 

~\\
\begin{hint}
	矢量微分算子与动量算子存在对应关系,拉普拉斯算子与动量方(动能)存在对应关系。因此,它们普遍出现在各大数理方程中。
\end{hint}
\begin{example}
已知角动量的Z分量算子
  \begin{equation*}
	\hat{L}_z= -i \hbar \frac{\partial }{\partial\varphi }
\end{equation*}
求解其固有值问题,并对固有函数进行归一化。
\end{example}
\emph{解:}
固有值方程为
  \[\hat{L}_z \Phi(\varphi) = l_z \Phi(\varphi) \]
代入$\hat{L}_z$表达式,得
\[-i \hbar \frac{\partial }{\partial\varphi } \Phi(\varphi) = l_z \Phi(\varphi) \]
整理,得
\[ \frac{1}{\Phi(\varphi)}\frac{\partial }{\partial\varphi } \Phi(\varphi) =  i\dfrac{l_z}{\hbar} \]
解得固有函数
\[ \Phi(\varphi) = A e^{i l_z \varphi / \hbar} \]
利用周期性边界条件$ \Phi(0) = \Phi(2\pi) $, 得 
\[ A e^{i l_z 0 /\hbar} =  A e^{i l_z 2\pi / \hbar}  \]
解得$$e^{i l_z 2\pi / \hbar} =1 $$
欧拉公式
\[ \cos(2\pi l_z  / \hbar) + i \sin(2\pi l_z  / \hbar) =1 \]
得
\[\cos(2\pi l_z  / \hbar) =1\]
因此
\[ 2\pi l_z  / \hbar = 2m \pi \quad (m= {\color{red}0}, \pm 1, \pm 2, \cdots  ) \]
得固有值
\[l_z = m \hbar\]
代回固有函数
\[ \Phi_m (\varphi) = A e^{i m \varphi } \]
归一化
\[
\begin{aligned}
	1 &= \int_{0}^{2\pi} |\Phi_m (\varphi)|^2 d \varphi = A^2 \int_{0}^{2\pi} \Phi_m^* (\varphi) \Phi_m (\varphi) d \varphi \\
	&= A^2 \int_{0}^{2\pi} e^{-i m \varphi } e^{i m \varphi } d \varphi \\
	&= 2\pi A^2
\end{aligned} 
 \]
解得
\[ A = \frac{1}{\sqrt{2\pi} }\]
因此,固有值问题得解 \\
\begin{itemize}
	\item 固有值 
	\[l_z = m \hbar \quad (m=0, \pm 1, \pm 2, \cdots  )  \]
	\item 固有函数 
	\[ \Phi_m (\varphi) = \frac{1}{\sqrt{2\pi} } e^{i m \varphi } \]
\end{itemize}


~~\\ 
 ~~\\ 


\section{波动方程}
自然界普遍存在振动现象,振动在媒介中传播形成波。机械振动的传递构成机械波,电磁场振动的传递构成电磁波(包括光波),温度变化的传递构成温度波(见液态氦),晶体点阵振动的传递构成点阵波(见点阵动力学),自旋磁矩的扰动在铁磁体内传播时形成自旋波(见固体物理学),实际上任何一个宏观或微观物理量受到扰动时都会在平衡位置附近来回振荡,振荡的能量在空间或媒介中传递时形成波。

所有的波动应具有一般性,服从统一的物理规律,描述这个规律的数理方程,称为波动方程。

\subsection{方程的建立}
\begin{example}
考虑一条均匀柔软的轻质细长弦线(长为$l$),两端固定,当它受到扰动后会在平衡位置附近来回做微小的振荡。试建立位移函数 $u(x,t)$所满足的方程。 
\end{example}
\begin{figure}[htbp]
	\centering
	\includegraphics[width=0.55\textwidth]{figs/2022-10-08-21-14-04.png}
	\caption{轻质细长弦线作微小振动示意图}
\end{figure}
\emph{解:}
选取任意微元$ds$为研究对象, 设左右侧临近微元对它的拉力分别$T_1, T_2$,对它们做正交分解\\
水平方向: $$T_2\cos \alpha _2 = T_1\cos \alpha _1 = T_0$$ 
竖直方向:$$F=T_2\sin \alpha _2-T_1\sin \alpha _1 = T_0(\tan \alpha _2-\tan \alpha _1) $$  
对竖直方向运用牛顿第二定律
$$\begin{aligned}
	F =ma &=(\rho ds)~ u_{tt} \\
	T_0 (\tan \alpha _2 -\tan \alpha _1 ) &= \rho ds~ u_{tt} 
\end{aligned}$$
斜率是位移函数的导数
$$ T_0(u_x(x+dx,t)-u_x(x,t)) =\rho ds~ u_{tt}$$
平衡位置微小运动,角度 $ \alpha  $很小,有 
$$ dx = ds \cos \alpha \approx ds $$
代回并整理,得
$$\begin{aligned}
	u_{tt} = \frac{T_0}{\rho} \frac{u_x(x+dx,t)-u_x(x,t)}{dx} =\frac{T_0}{\rho} u_{xx}  
\end{aligned}$$
令 $a^2 =\dfrac{T_0}{\rho}$,
得波动方程
\begin{equation} 
	\boxed{u_{tt} = a^2 u_{xx}}  
\end{equation} 
设初始位置为$\psi (x)$,初始速度为$\phi (x)$,有 \\ 
(1) 初始条件 
	$$ u(x,t)|_{t=0}= \psi (x),\quad  u_t(x,t)|_{t=0}= \phi (x) $$
由两端固定得 \\
(2) 边界条件
	$$ u(x,t)|_{x=0}= 0 ,\quad  u(x,t)|_{x=l}= 0 $$
得波动方程的定解问题
$$\left\{
\begin{aligned}
	&u_{tt} = a^2 u_{xx} \\
	&u(x,t)|_{x=0}= 0 ,\quad  u(x,t)|_{x=l}= 0\\
	&u(x,t)|_{t=0}= \psi (x),\quad  u_t(x,t)|_{t=0}= \phi (x)
\end{aligned}\right.
$$
\textcolor{red}{结束!}


~~\\ 
\begin{example}
	电磁振荡形成电磁波,试从麦克斯韦方程导出电磁波的波动方程
\end{example}
\emph{解:}
			在介质中,定义电位移矢量$\mathbf{D}$ 和磁场强度 $\mathbf{H}$
			\[ \mathbf{D}=\epsilon_0 \mathbf{E} + \mathbf{P} = \epsilon_0 \epsilon_r \mathbf{E} \, (*),  \qquad \mathbf{H}=\frac{1}{\mu_0} \mathbf{B} -\mathbf{M}= \frac{1}{\mu_0\mu_r}\mathbf{B} \, (*)\]
			麦克斯韦方程组表示为
			\[ \begin{aligned}
					\text{I}~~~ \hspace*{2em}&\nabla \cdot \mathbf{D} =\rho _f  \\  
					\text{II}~~ \hspace*{2em}&\nabla \cdot \mathbf{B} = 0  \\  
					\text{III}~ \hspace*{2em}&\nabla \times  \mathbf{E} = -\cfrac{\partial \mathbf{B}}{\partial t }  \\  
					\text{IV~}~ \hspace*{1.7em}&\nabla \times  \mathbf{H} = \mathbf{J}_f +  \cfrac{\partial \mathbf{D}}{\partial t } 
				\end{aligned} \]
			考虑真空,有,  
			\[ \rho _f =0, \qquad  \mathbf{J}_f =0 , \qquad  \mathbf{B} = \mu_0 \mathbf{H}, \qquad  \mathbf{D} = \epsilon_0 \mathbf{E} \]
			代入麦克斯韦方程-IV
			 \[ \nabla \times  \mathbf{H} = \mathbf{J}_f +  \cfrac{\partial \mathbf{D}}{\partial t } \]
			得: \[ \nabla \times \mathbf{B} = \mu_0\epsilon_0 \cfrac{\partial \mathbf{E}} {\partial t } \]
			对麦克斯韦方程-III~做旋度计算
			\[    
			\begin{aligned}
			  \nabla \times (\nabla \times  \mathbf{E}) = - \nabla \times \cfrac{\partial \mathbf{B}}{\partial t } = -  \cfrac{\partial (\nabla \times\mathbf{B})}{\partial t } = - \mu_0\epsilon_0  \cfrac{\partial ^2 \mathbf{E}} {\partial t^2 }
			\end{aligned} \]  
			把麦克斯韦方程-I~代入矢量公式
			\[
			\begin{aligned}
				\nabla \times (\nabla \times  \mathbf{E}) =  \nabla (\nabla \cdot  \mathbf{E})- \nabla^2 \mathbf{E} = - \nabla^2 \mathbf{E} 
			\end{aligned} \]
			联立两式,得
			\[
			\nabla^2 \mathbf{E}= \mu_0\epsilon_0 \cfrac{\partial ^2 \mathbf{E}} {\partial t^2 }\]
			令 $$ c =\frac{1}{\sqrt{\mu_0\epsilon_0}} $$
		 	得波动方程
			\[\mathbf{E}_{tt} =c^2\nabla^2 \mathbf{E}\]
			同理,得磁场波动方程
  			\[\mathbf{B}_{tt} =c^2\nabla^2 \mathbf{B}\]
			波传播的速度为常数$c$,这正是光速。即,在真空中光与电磁具有相同的传播速度。 
			可见,波的传播有时媒介并不是必须的,比如光波、引力波和物质波是不需要媒介的。
			不失一般式,它们可以统一写成
\begin{equation*} 
	u_{tt} = a^2 u_{xx}  
\end{equation*} 
~~\\ 
\begin{hint}
	相同的数学方程,可以从不同的物理模型导出,反映出不同物理现象具有相通性,当我们不考虑方程中各系数的具体物理意义,而只关注如何求解时,称此方程为泛定方程。 波动方程就是一个泛定方程,求解波动方程有助于我们更好地理解自然界各种各样的振荡行为导致的波,比如地震波、声波、电磁波等的共性。
\end{hint}

\subsection{方程的求解} ~\\
对于一个能基本正确描述物理模型的偏微分方程定解问题,它的解通常应该是存在的,并且是唯一的。这是因为经典物理认为,在给定条件下一个物理体系通常具有唯一确定的状态。比如对于确定环境中的物体,若给定初始位置和初始速度,则其随后任意时刻的运动状态是确定的。也就是说:描述经典运动规律的方程的解是存在的,且是唯一的。同时,方程的解应具有很好的稳定性。因为边界条件和初始条件的测定总是有一定误差的,这些微小的误差是不会导致方程的解发生重大改变,因此解是稳定的。

定解问题的存在性、唯一性和稳定性统称定解问题的适定性。如果方程的解具有存在性、唯一性、稳定性,则称这个定解问题是适定的。如果微小的误差导致方程的解发生了重大改变,我们趋于认为这个定解问题不能正确地描述对应物理模型,这个定解问题的提法是不合适的,是与实际不符的,因此求解它是没有实际的物理意义的。

确定定解问题的适定性是方程求解的一个方面,它可确保我们要解的方程是一个数理上合理的方程。另一方面就是获得方程的解函数,并通过考察解函数的光滑性、渐近性行为等,了解极限条件下物体的运动规律。

\begin{example}
	试采用分离变量法求解一维波动方程定解问题
	$$\begin{cases}
		u_{tt}=a^2u_{xx}\\
		u(x,t)|_{t=0}= \varphi (x) ,~~~ u_t(x,t)|_{t=0}= \phi (x) \\
		u(x,t)|_{x=0}= 0, ~~~  u(x,t)|_{x=l}= 0 
	\end{cases}$$ 
\end{example}
\emph{解:}
	设解函数可以写成连乘形式 $$\displaystyle  u(x,t)=T(t)X(x) $$
	求导
	\[ u_{tt} = T''(t)X(x), \qquad  u_{xx} = T(t)X''(x)\]
	代回方程, 得:
	\begin{equation*}
		 T^{''}(t)X(x) =a^2 T(t)X^{''}(x) 
	\end{equation*}
	整理,得
	$$ \dfrac{T^{''}}{a^2 T}=\dfrac{X^{''} }{X} $$
	方程两端对$t$求导
	$$ \frac{\mathrm{d}}{\mathrm{d}t}\dfrac{T^{''}}{a^2 T}=\frac{\mathrm{d}}{\mathrm{d}t}\dfrac{X^{''} }{X} = 0 $$
	说明方程应等于一个常数,记为$-\lambda$, 并称为分离变量常数
	$$ \dfrac{T^{''}}{a^2 T}=\dfrac{X^{''} }{X} =-\lambda $$
	原偏微分方程转化成两个常微分方程\\
方程(I):
$$ \begin{cases}
	X ^{\prime\prime} +\lambda X=0, \quad 0<x<l \\
	X(0)=0, \quad X(l)=0
	\end{cases}  
$$ 
方程(II):
$$ \begin{cases}
	T^{\prime\prime} +\lambda a^2 T=0,  0<t \\
	T(0)X(x)= \varphi (x) ,~~~ T'(0)X(x)= \phi (x)
	\end{cases} 
$$ 
$\star $定解条件的获得 \\
(1) 把$\displaystyle  u(x,t)=T(t)X(x) $ 代入边界条件
$$u(x,t)|_{x=0}= 0, ~~~  u(x,t)|_{x=l}= 0 $$
得 $$X(0)= 0, ~~~  X(l)= 0 $$
(2) 把$\displaystyle  u(x,t)=T(t)X(x) $ 代入初值条件
$$
u(x,t)|_{t=0}= \varphi (x) ,~~~ u_t(x,t)|_{t=0}= \phi (x)   
$$ 
得
$$
T(0)X(x)= \varphi (x) ,~~~ T'(0)X(x) = \phi (x) 
$$ 
分离变量成功的条件:
\begin{itemize}
	\item 原方程必须是齐次的 \\
	如果非齐次,比如含自由项$f(x)$, 或 $f(t)$, 或 $f(x,t)$, 则
	$$T^{''}(t)X(x) =a^2 T(t)X^{''}(x) + f $$分离变量不能继续。 
	\item 边界条件必须是齐次的 \\
	如果非齐次,比如$u(x,t)|_{x=0}= g(t)$,则新的边界条件为 
	$$X(0)= g(t)/T(t)$$方程(I)的边界条件含时间项,依然是双变量问题。方程(II)本来就是双变量问题。因此分离变量不成功
\end{itemize}

~~\\ 
方程(I)是单变量常微分方程,可写成
$$ \frac{\mathrm{d^2}}{\mathrm{d}x^2} X = -\lambda X  $$ 
有微分算子
$$ \hat{L} = \frac{\mathrm{d^2}}{\mathrm{d}x^2} $$
方程变为
$$ \hat{L} X = -\lambda X  $$ 
这是微分算子$\hat{L}$的固有值(本征值)问题,求解可得固体值$ \lambda _n $ 和固有函数  $X_n(x)$ \\
写出特征方程
$$ \mu^2 +\lambda =0  $$ 
特征方程有根
$$\begin{cases}
		\mu~_1 =+\sqrt{-\lambda}\\
		\mu~_2 =-\sqrt{-\lambda}
	\end{cases}$$
~~\\	
注意$\lambda$是任意常数,现分情况讨论:
~~\\
(1) 当$\lambda < 0$时,有两相异实根\\
通解:$$ X(x)=A e^{\sqrt{-\lambda}x} + B e^{-\sqrt{-\lambda}x}  $$ 
代入定解条件$X(0)=0, \quad X(l)=0$, 得
$$ \begin{cases}
		 A + B = 0 \\
		 A e^{\sqrt{-\lambda}l} + B e^{-\sqrt{-\lambda} l} =0
	\end{cases} $$ 
矩阵形式为
	$$ \left[
		\begin{array}{lll}
			1 & 1\\
			e ^{\sqrt{-\lambda} l} & e~^{-\sqrt{-\lambda} l}
		\end{array}
		\right]
		\left[
		\begin{array}{ll}
			A\\
			B
		\end{array}
		\right]
		=\left[
		\begin{array}{ll}
			0\\
			0
		\end{array}
		\right]  $$
有解条件为系数行列式为零
		$$ \begin{vmatrix}
				1 & 1 \\
				e^{\sqrt{-\lambda} l} & e ^{-\sqrt{-\lambda} l}
			\end{vmatrix}
			= 0  $$ 
很明显,这个行列式不等于0, 因此只有零解 (A=0,~B=0),没有非平庸解。
~~\\
(2) 当$\lambda = 0$时,有两相同实根, $$\mu_1 = \mu_2 =0 $$
通解:	$$ X(x) = Ax + B $$ 
分别取$x=0, x=l$, 得定解方程组:
$$\left\{
	\begin{array}{lll}
		B=0\\
		Al+B=0
	\end{array} \right. $$
也只有零解(A=0,~B=0)
~~\\
(3) 当$\lambda > 0$时,有两虚根 $$ \mu~_1 = i \sqrt{\lambda}, \quad \mu _2 = - i \sqrt{\lambda}$$	 
通解:$$ X(x)=A\cos \sqrt{\lambda}x+ B\sin \sqrt{\lambda}x  $$  
分别取$x=0, x=l$, 得定解方程组:
$$ \left[
	\begin{array}{lll}
		1&0\\
		\cos( {\sqrt{\lambda}~l}) &\sin ({\sqrt{\lambda}~l})
	\end{array}
	\right]
\left[
	\begin{array}{ll}
		A\\
		B
	\end{array}
	\right]=
\left[
	\begin{array}{ll}
		0\\
		0
	\end{array}
	\right] $$
由系数行列式为零,得
	$$ \sin ({\sqrt{\lambda}~l})=0  $$
有
	$$ \sqrt{\lambda} l = n \pi,  $$ 
得固有值:$$ \boxed{\lambda~_n = \dfrac{n^2 \pi~^2 }{l~^2 }} $$ \\ 
把固有值 $ \lambda~_n= \dfrac{n^2 \pi~^2 }{l~^2 } $ 代回定解方程组, 得
$$ A= 0,\qquad B=1$$ 
代入通解
$$ X(x)=A\cos \sqrt{\lambda}x+ B\sin \sqrt{\lambda}x  $$
得固有函数:$$ \boxed{X_n(x)=\sin \dfrac{n\pi~}{l} x}   $$ 
若取$n =0$,固有函数$X_0(x)=0$,是零解没有物理意义,因此有
\[ n =1, 2, 3, \dots \]
令 $\dfrac{n\pi~}{l} =\omega_n $,有 \\
\begin{itemize}
	\item 固有值:$\lambda_n = \omega_n^2, \quad (n =1, 2, 3, \dots)$
	\item 固有函数: $X_n(x)= \sin \omega_n x$
\end{itemize} 

~~\\ 
解方程II:$$  T~^{\prime\prime} +\lambda {a~^2 T}=0   $$ 
把固有值 $\lambda_n$ 代入, 得:
$$ T_n ^{\prime\prime} +\omega ^2_n a~^2 ~T_n=0  $$  
改写为:
$$ T_n ^{\prime\prime} + (\omega_n a)^2 T_n=0  $$ 
这是振动模型, 解为
$$ T_n(t) = C _n \cos \omega_n a t+ D _n \sin \omega _n a t   $$ 

~~\\ 
原方程的基本解:
$$ \begin{aligned}
	u_n(x,t) &= T_n(t)X_n(x)\\
		&=(a_n\cos \omega_nat+ b_n\sin \omega _nat ) \sin \omega_n x\\
		&=(a_n\cos\dfrac{ n\pi a}{l}t+ b_n\sin \dfrac{ n\pi a}{l}t) \sin \dfrac{ n\pi }{l}x
\end{aligned}  $$
物理上,把本征函数系\{$\sin \dfrac{ n\pi }{l}x$\}中$n=1$的波称为基波(基态,能量最低的态),$n\geq 2$的波称为谐波(激发态),音乐上则称为基音和泛音。
\begin{figure}[h]
	\centering
	\includegraphics[width=0.45\textwidth]{figs/wave.png}
	\caption{前三个解函数的波型}
	\label{fig:wave}
\end{figure}

~~\\ 
画出前三个基本解函数的波形,如图[\ref{fig:wave}]所示。显然,波长与弦长之间存在关系
\[ l = \frac{\lambda}{2} n\]
这是驻波条件,因此所有基本解都是驻波,物理上称这种基本解为驻波解。注意到原方程是齐次的,根据叠加原理,基本解的线性叠加也是方程的解,称为方程的一般解或叠加解:
$$ \begin{array}{llll}
			u(x,t) &=&\sum\limits_{n=1}^{\infty } u_n(x,t)\\
			&=& \sum\limits_{n=1}^{\infty }  (a_n\cos\dfrac{ n\pi a}{l}t+ b_n\sin \dfrac{ n\pi a}{l}t) \sin \dfrac{ n\pi }{l}x
		\end{array}   $$ 
~~\\ 
\begin{hint}
由于一个本征方程的所有本征函数构成正交完备系,它们的线性叠加包含了所有可能的解。当然,也包括满足任意初始条件的解。
\end{hint}

~~\\
因此,在叠加解中取$t=0$, 结合初始位置条件 $u(x,0)|= \varphi (x)$, 有
$$u(x,0)|= \varphi (x) =\sum\limits_{n=1}^{\infty }  (a_n\cos\dfrac{ n\pi a}{l}0+ b_n\sin \dfrac{ n\pi a}{l}0) \sin \dfrac{ n\pi }{l}x = \sum_{n=1}^{\infty } a_n \sin \dfrac{ n\pi }{l}x $$ 
整理得
$$\varphi (x) = \sum_{n=1}^{\infty } a_n \sin \dfrac{ n\pi }{l}x $$ 
这是一个函数的三角级数展开式,展开系数$a_n$可通过傳里叶级数公式求得。

注意到本题函数$\varphi(x)$的定义域为$x \in$($0,l$),而标准的傳里叶级数展开函数的周期为($-l,l$)。把$\varphi(x)$做周期性延拓。
$$ \begin{aligned}
	a_n &= \frac{1}{l}\int\limits_{-l }^{l}  \varphi (x) \sin \dfrac{ n\pi }{l}x dx \\
	&= \frac{1}{l}\left[\int\limits_{-l }^{0}  \varphi (x) \sin \dfrac{ n\pi }{l}x dx+ \int\limits_{0 }^{l}  \varphi (x) \sin \dfrac{ n\pi }{l}x dx \right]  
\end{aligned}$$
延拓后,$\varphi(x)$在($-l,0$)域与在($0,l$)域的积分相同,得展开系数
\[a_n =  \frac{2}{l}\int\limits_{0 }^{l}  \varphi (x) \sin \dfrac{ n\pi }{l}x dx  \]
有时,称这种级数为半幅傅里叶级数。

展开系数$b_n$可由初始速度条件 $u_t(x,t)|_{t=0}= \phi (x)$确定, 这是导数条件,因此,先把叠加解对时间求导
$$ u_t(x,t) = \sum\limits_{n=1}^{\infty }  (- a_n \dfrac{ n\pi a}{l} \sin\dfrac{ n\pi a}{l}t+ b_n \dfrac{ n\pi a}{l}  \cos \dfrac{ n\pi a}{l}t) \sin \dfrac{ n\pi }{l}x $$
取$t=0$, 有结合初始速度条件,有
$$ u_t(x,0)= \phi (x) = \sum_{n=1}^{\infty } (b_n \frac{ n\pi a}{l}) \sin \frac{ n\pi }{l}x $$
由半幅傳里叶级数公式,得
$$ \begin{aligned}
	b_n \frac{ n\pi a}{l} 
	&=  \frac{2}{l}\int\limits_{0 }^{l}  \phi (x) \sin \dfrac{ n\pi }{l}x dx   \\
	b_n &=  \frac{2} { n\pi a}  \int\limits_{0 }^{l}   \phi (x) \sin \dfrac{ n\pi }{l}x dx
\end{aligned}$$
得解函数
$$ \displaystyle \boxed{\begin{aligned}
	u(x,t) &= \sum\limits_{n=1}^{\infty }  (a_n\cos\dfrac{ n\pi a}{l}t+ b_n\sin \dfrac{ n\pi a}{l}t) \sin \dfrac{ n\pi }{l}x \\ 
	\\
	a_n &=\dfrac{2}{l}\int\limits_{0 }^{l}  \varphi (x) \sin \dfrac{ n\pi }{l}x dx   \\
	b_n &=\dfrac{2} { n\pi a}  \int\limits_{0 }^{l}   \phi (x) \sin \dfrac{ n\pi }{l}x dx  
\end{aligned}} $$


~~\\ 
\begin{hint}
	波动方程定解问题求解(分离变量法)的基本步骤如下
\begin{itemize}
	\item 波动方程经分离变量转化为固有值问题和演化问题
	\item 边界条件决定固有值和固有函数。
	\item 代入固有值解演化问题
	\item 由固有值问题和演化问题的解写出基本解
	\item 经叠加原理由基本解写出叠加解。
	\item 初始条件结合半幅傳里叶级数确定叠加解的系数,得解函数
\end{itemize}
\end{hint}
~~\\ 
\begin{hint}
由于叠加原理要求方程必须是线性的,因此,能用分离变量法求解的定解问题必须满足三个条件: 
	\begin{itemize}
		\item 方程是线性的
		\item 方程是齐次的 
		\item 边界条件是齐次的
	\end{itemize}
\end{hint}
~~\\ 

\subsection{固有函数的性质}
\begin{proposition} 固有函数$$X_n(x)= \sin \dfrac{n\pi~}{l} x, \quad (n=1,2,3,\cdots)$$ 两两正交 
\end{proposition}
\begin{proof}
	考虑到固有函数是实函数,所谓两两正交,即要证明
	\[\int_0 ^l X^*_m(x) X_n(x) dx = \int_0 ^l X_m(x) X_n(x) dx =0, \quad (n\ne m)\]
	固有函数必满足固有方程
	$$ \begin{array}{llll}
			&X_n ^{''}+\lambda_n X_n=0 \qquad (1) \\
			&X_m ^{''}+\lambda_m X_m=0 \qquad (2)
		\end{array}  $$  
	用$X_m$乘(1)式,$X_n$乘(2)式
		$$\begin{array}{llll}
				&X_m X_n ^{''}+\lambda_n X_m X_n=0 \qquad (3)\\
				&X_nX_m ^{''}+\lambda_m X_n X_m=0 \qquad (4)
			\end{array}$$ 
	(3)-(4):
			$$\begin{array}{llll}
					& (\lambda_n - \lambda_m) X_n X_m= X_n X_m ^{''}-X_m X_n ^{''} 
				\end{array}$$  
	两边同时积分:
			$$ 
			\begin{aligned}
					(\lambda_n - \lambda_m )& \int\limits_{0 } ^{l}  X_n X_m dx  =  \int\limits_{0} ^{l}  [ X_n X_m ^{\prime\prime} - X_m X_n ^{\prime\prime} ] dx \\  
					& = [ X_n X_m ^{'} - X_m X_n ^{'} ] _0 ^{l} - \int\limits_{0} ^{l} [ X_n ^{'} X_m ^{'} - X_m ^{'} X_n ^{'} ] dx  
				\end{aligned}$$
	由驻波特性,等式右边的第一项为零,第二项显然为零。因此等式左边等于零
	$$ (\lambda_n-\lambda_m) \int\limits_{0 }^{l}  X_n X_m dx=0   $$ 
	$X_n, X_m$属于不同固有值,即 $\lambda_n \ne \lambda_m$, 有:
	$$ \int\limits_{0 }^{l}  X_n X_m dx=0 , \qquad (n\ne m)  $$  
\textcolor{red}{证毕!}
\end{proof}

~~\\ 
\begin{proposition}固有函数$$X_n(x)= \sin \dfrac{n\pi~}{l} x, \quad (n=1,2,3,\cdots)$$ 平方可积  
\end{proposition}
\begin{proof}
当$n= m$时,有   
$$ \begin{aligned}
			\int\limits_{0 }^{l}  X_n X_m dx&=\int\limits_{0 }^{l}  X_n X_n dx \\ &= \int\limits_{0 }^{l}   \sin ^2  \frac{n\pi~}{l} x dx \\
			&= \frac{1}{2}\int\limits_{0 }^{l}   (1-\cos   \frac{2n\pi~}{l} x) dx \\
			&= \left.\frac{1}{2}[x]\right|_0^l =  \dfrac{l}{2}  \\ 
\end{aligned}  $$ 
\textcolor{red}{证毕!} 
\end{proof}

~~\\ 
把正交性和平方可积写在一起,有
\begin{equation*}
		\int\limits_{0 }^{l}  X_n X_m dx =
		\begin{cases}
		 0, \qquad (n \not= m) \\ 
		 \\ 
		 \dfrac{l}{2} , \qquad (n =m) 
		\end{cases} 
\end{equation*}
写成正交归一形式
\begin{equation*}
	\dfrac{2}{l}\int\limits_{0 }^{l}  X_n X_m dx =
	\begin{cases}
	 0, \qquad (n \not =m)  \\ 
	 \\
	1, \qquad (n=m)
	\end{cases}  
\end{equation*}
写成$\delta$函数形式
\begin{equation*}
	\dfrac{2}{l}\int\limits_{0 }^{l}  X_n X_m dx = \delta _{nm}
\end{equation*}
\begin{example}
	利用波动方程的定解公式,求解如下定解问题
$$\displaystyle  \begin{cases}
			u_{tt} =u_{xx},\quad (0<x<1, t>0)\\
			u(0,t) =u(1,t)=0 \\
			u(x,0) =\sin \pi x, \quad u_t (x,0)=0 
		\end{cases}$$  
\end{example}
\emph{解:}
这是零边界条件的波动方程($a^2 =1$), 有\\
固有值:$$ \lambda_n =\dfrac{n^2 \pi ^2}{l ^2 }= n^2 \pi^2 $$  
固有函数:$$  X_n = \sin \dfrac{n\pi}{l} x = \sin n \pi x $$ 
叠加解:
$$\begin{aligned}
		u(x,t) =& \sum\limits_{n=1}^{\infty }  (a_n\cos\dfrac{ n\pi at}{l}+ b_n\sin \dfrac{ n\pi at}{l}) \sin \dfrac{ n\pi x}{l}\\
		= &\sum\limits_{n=1}^{\infty }  (a_n\cos n\pi t+ b_n\sin n\pi t ) \sin n\pi x \\
	\end{aligned}$$
由初始位置条件确定 $a_n$  
	$$\begin{aligned}
			a_n&=  \dfrac{2}{l} \int\limits_{0 }^{l}  \varphi (x) \sin \dfrac{ n\pi x}{l} dx \\
			&= 2 \int\limits_{0 }^{1}  \sin(\pi x) \sin n\pi x dx    
	\end{aligned}$$ 
由固有函数正交性可知,当$n \not = 1$时,上式积分为零,即$a_n=0,\quad n \not = 1$。现求 $a_1$
	$$\begin{aligned}
		a_1
		&= 2 \int\limits_{0 }^{1}  \sin\pi x \sin \pi x dx \\
		&=2\times \frac{1}{2} =1   
\end{aligned}$$ 
由初始速度条件确定 $b_n$  
$$\begin{aligned}
			b_n&= \dfrac{2} { n\pi a} \int\limits_{0 }^{l}  \phi  (x) \sin \dfrac{ n\pi x}{l} dx  \\
			&= \dfrac{2} { n\pi} \int\limits_{0 }^{1}  0  \sin n\pi x dx \\
			&= 0
		\end{aligned}$$ 
代回,得解函数
$$\begin{aligned}
	u(x,t) &= \sum\limits_{n=1}^{\infty }  (a_n\cos n\pi t+ b_n\sin n\pi t ) \sin n\pi x \\
	&= \cos(\pi t) \sin(\pi x) 
    \end{aligned}$$ 
解函数随时间的振荡如图[\ref{fig:wave2}] 所示。
    \begin{figure}[h]
	\centering
	\includegraphics[width=0.6\textwidth]{figs/wave2.png}
	\caption{解函数振幅随时间呈cos振荡}
	\label{fig:wave2}
\end{figure}

~~\\ 
~~\\ 

\section{热传导方程}
自然界普遍存在能量传导和物质扩散及对流现象。比如传热过程,它存在三种基本方式:热辐射、热传导和热对流。热辐射服从普康克黑体辐射定律,它在量子力学建立的初期被发现。此处,我们以热传导现象为例, 建立一般性的热传导数理方程并求解。

\subsection{方程的建立} ~\\
热量总是从高温区域向温度区域传导,服从的是傅里叶热传导实验定律:
\begin{equation*}
	\vec{q}=-k\nabla u
\end{equation*}
式中,$\vec{q}$是热流强度, 定义为单位时间通过单位横截面积的热量; $k$ 是材料的导热系数 ;$u(x,y,z,t)$ 是温度函数。\\
\begin{example}
 试建立温度函数$u(x,y,z,t)$所满足的微分方程。
\end{example} 
\begin{figure}[h]
	\centering
	\includegraphics[width=0.3\textwidth]{figs/2022-12-11-12-06-43.png}
	\caption{热传导微元}
	%\label{}
\end{figure}
\emph{解:}
	取傅里叶热传导定律的分量形式 \\
	$$q_{x_i}=-k_x \dfrac{\partial }{\partial x_{i}} u$$ 
选取介质任意微元为研究对象, 考虑单位时间在$~x~$方向的净流入\\
\[\begin{aligned}
	-(q_x|_{x+dx} &-q_x|_x)dydz \\
	&=-\dfrac{\partial q_x }{\partial x}  dxdydz\\
	&= \dfrac{\partial }{\partial x} (k_x \dfrac{\partial u}{\partial x})dxdydz \\
\end{aligned}\]
体系元热量净流入:$$
	\left[\frac{\partial }{\partial x} (k_x\frac{\partial u}{\partial x}) + \frac{\partial }{\partial y} (k_y\frac{\partial u}{\partial y}) + \frac{\partial }{\partial z} (k_z\frac{\partial u}{\partial z}) \right]dxdydz 
$$ 
热能导致介质温度发生变化(能量守恒定律), 设$\rho$ 是介质的质量密度, $c$是介质的比热,$\partial u$为$\partial t$时间内微元的温度变化量, 有\\
	\begin{equation*}
		c \rho \frac{\partial u}{\partial t}dxdydz=\left[\frac{\partial }{\partial x} (k_x\frac{\partial u}{\partial x}) + \frac{\partial }{\partial y} (k_y\frac{\partial u}{\partial y}) + \frac{\partial }{\partial z} (k_z\frac{\partial u}{\partial z}) \right]dxdydz 
	\end{equation*}
对于各向同性介质, 有$ k_x =k_y=k_z =k  $ 
\begin{equation*}
	\begin{aligned}
	c \rho \frac{\partial u }{\partial t}&=k\left[\frac{\partial }{\partial x} (\frac{\partial u}{\partial x}) + \frac{\partial }{\partial y} (\frac{\partial u}{\partial y}) + \frac{\partial }{\partial z} (\frac{\partial u}{\partial z}) \right]	 \\
\frac{\partial u }{\partial t}&=\frac{k}{c\rho}\left[\frac{\partial }{\partial x} (\frac{\partial u}{\partial x}) + \frac{\partial }{\partial y} (\frac{\partial u}{\partial y}) + \frac{\partial }{\partial z} (\frac{\partial u}{\partial z}) \right] \\
u_t&=a^2 [u_{xx}   +u_{yy}  +u_{zz}] 
\end{aligned}
\end{equation*}
式中,使用了记号 $$ a^2 = \dfrac{k}{c\rho}$$
整理,得热传导方程
\begin{equation}\label{eq:heat}
	\boxed{u_t = a^2 \nabla ^2 u } 
\end{equation}
对于一维导线,有
$$ u_t= a^2 u_{xx}$$ 
如果存在热源$F(x,t)$,可令 $f=\dfrac{F}{c\rho}$, 有	$$ u_t= a^2 u_{xx}+f$$ 
如果时间足够长,系统达到平衡态,温度不再变化 ($u_t =0$),得 \\ 
$$ u_{xx}=0$$ 
考虑三维形式,有\\
无源$Laplace$ 方程: $$  \nabla ^2 u =0$$ 
有源$Poisson$ 方程: $$  \nabla ^2 u =-f$$    


~~\\ 
\begin{hint}
	能量传输在微观上是通过粒子的相互碰撞和扩散进行的。因此传导方程与扩散方程在本质上是同源的。扩散方程描述物质从高浓度区域向低浓度区域扩散的过程(物质密度梯度驱动)。传导方程描述各种势场梯度驱动的能量传导过程, 随着传导的进行,梯度变小并最终为零,物质密度和势场不再变化($u_t$)而达到平衡。平衡场的规律由拉普拉斯方程描述。
\end{hint}

\subsection{方程的求解}
\begin{example}
	对于有限长导线,求解一维热传导方程的定解问题 \\
	$$\displaystyle \begin{cases}
		u_{t}=a^2u_{xx} ,~~~~ (0<x<l, t>0)\\
		 u(x,t)|_{x=0}= 0, ~~~  u(x,t)|_{x=l}= 0  \\
		u(x,t)|_{t=0}= \psi (x)
	\end{cases}$$ 
\end{example}
\emph{解:}
	这是齐次的二阶线性微分方程, 由于边界条件也是齐次的,可用分离变量法进行求解,
	令 $$\displaystyle  u(x,t)=T(t)X(x) $$ 代回方程, 得 
	\begin{equation*}
		T^{'}(t)X(x) =a^2 T(t)X~^{''}(x) 
	\end{equation*}
	整理
	\begin{equation*}
		\frac{T^{'}}{a^2 T}=\frac{X^{''} }{X} 
	\end{equation*}
	令其等于一个分离变量常数
	\begin{equation*}
		\frac{T^{'}}{a^2 T}=\frac{X^{''} }{X} =-\lambda 
	\end{equation*}
	方程转化为两个常微分方程 \\
	方程(I):\\
	$$\displaystyle  \begin{cases}
		X~^{''} +\lambda X=0  ~~,~~ 0<x<l\\
		X(0)=0 ~,~X(l)=0
	\end{cases}$$ 	
	方程(II):\\
	$$\displaystyle  \begin{cases}
		T~^{'} +\lambda {a~^2 T}=0 \\
		......
	\end{cases}$$ 
	~~\\ 
	方程(I)是零边界条件的固有值问题,与波动方程的固有值问题完全相同, 有\\
	固有值:$$\displaystyle  \lambda~_n=\frac{n^2\pi~^2}{l~^2}, \qquad (n=1,2,3,\cdots)$$  
	固有函数:$$\displaystyle  X~_n= \sin \frac{n\pi~}{l} x$$
	~~\\ 	
	把固有值$\lambda_n$代入方程(II), 得:
	$$\displaystyle  T_n^{'} +\lambda~_n a~^2 ~T_n=0 $$ 
	改写成 $$\displaystyle  T_n^{'} + r_n T_n=0 $$  
	这是衰减模型,有解
	\begin{equation*}
		T_n= exp(-r_nt)
	\end{equation*}
	代回$r_n$,解为:$$ T_n=  \exp(-(\dfrac{n\pi a}{l})^2 t)$$ \\  
	原方程的基本解
	\begin{equation*}
		u_n(x,t)= T_n(t)X_n(x)
	\end{equation*}
	叠加解
	$$\begin{array}{llll}
		u(x,t) &= \sum\limits_{n=1}^{\infty }B_n u_n(x,t) \\
		&= \sum\limits_{n=1}^{\infty }B_n T_n(t)X_n(x)\\
		&= \sum\limits_{n=1}^{\infty }B_n  \exp(-(\dfrac{n\pi a}{l})^2 t) \sin \dfrac{n\pi~}{l} x \\
	\end{array}$$ 
	在叠加解中取$t=0$, 
	\begin{equation*}
		u(x,0) = \sum\limits_{n=1}^{\infty }B_n  \exp(-(\dfrac{n\pi a}{l})^2 0) \sin \dfrac{n\pi~}{l} x
	\end{equation*}
	结合初值条件\\ 
	$$ \displaystyle u(x,0)= \psi(x)$$ 
	得
	$$\psi (x)=\sum\limits_{n=1}^{\infty } B_n \sin \dfrac{ n\pi }{l} x$$ 
	由半幅傳里叶变换公式,得系数:\\  
	$$ \displaystyle B_n=  \dfrac{2}{l}\int\limits_{0 }^{l}  \psi (x) \sin \dfrac{ n\pi }{l} x dx $$
	方程得解
	$$
	\boxed{\begin{aligned}
			&u(x,t) = \sum\limits_{n=1}^{\infty }B_n  \exp(-(\dfrac{n\pi a}{l})^2 t) \sin \dfrac{n\pi~}{l} x \\
			~\\
			&B_n =  \dfrac{2}{l}\int\limits_{0 }^{l}  \psi (x) \sin \dfrac{ n\pi }{l} x dx  
		\end{aligned}} 
	$$ 
\begin{example}
对于一维有限长导线,求解如下热传导方程 
$$\displaystyle \begin{cases}
	u_{t} =u_{xx} ,~~ 0<x<L, ~~t>0\\
	u(0,t) =0, ~~u(L,t)=0 \\
	u(x,0) =x(L-x)
\end{cases}$$ 
\end{example}
\emph{解:}
	与标准方程相比较, 有$ a=1$。零边界条件决定固有值问题的解\\
	固有值:$$\displaystyle  \lambda~_n=\dfrac{n^2\pi~^2}{l~^2 }= (\dfrac{n\pi }{L}) ^2$$  
	固有函数:$$\displaystyle  X~_n=\sin \dfrac{n\pi~}{l} x=\sin \dfrac{n\pi~}{L} x $$
    方程的叠加解为
	$$\displaystyle \begin{aligned}
		u(x,t)&=\sum\limits_{n=1}^{\infty } B_n  \exp(-(\dfrac{n\pi a}{l})^2 t) \sin \dfrac{n\pi~}{l} x\\
		&= \sum\limits_{n=1}^{\infty } B_n  \exp(-(\dfrac{n\pi}{L})^2 t) \sin \dfrac{n\pi~}{L} x 
	\end{aligned}$$ 
	系数为
	$$\displaystyle \begin{aligned}
		B_n&= \dfrac{2}{l}\int_{0 }^{l}  \psi (x) \sin \dfrac{ n\pi }{l} x dx  \\
		&= \dfrac{2}{L}\int_{0 }^{L}  x(L-x) \sin \dfrac{ n\pi }{L} x dx  \\
		&=\dfrac{2}{L} \times 2 (\dfrac{L}{n\pi})^3  [1-\cos n \pi ] \\
		&= 4 \dfrac{L^2}{(n\pi)^3}[1-(-1)^n ]  \\
		&= \dfrac{8L^2}{(2k+1)^3\pi^3}
	\end{aligned}$$  
	代回叠加解, 得解函数
	\begin{equation*}
		u(x,t)= \dfrac{8L^2}{\pi^3} \sum\limits_{k=0}^{\infty }  \dfrac{1}{(2k+1)^3}  \exp(-(\dfrac{(2k+1)\pi a}{L})^2 t) \sin \dfrac{(2k+1)\pi}{L} x
	\end{equation*}
~~\\ 
\begin{hint}
计算定积分: $$ \displaystyle \int_{0 }^{L}  x(L-x) \sin \dfrac{ n\pi }{L} x dx$$ 
\end{hint}
\emph{解:}
(1) 分部积分法
$$
\begin{aligned}
	\int_{0 }^{L} x(L-x) \sin \dfrac{ n\pi }{L} x dx &= -\dfrac{L}{ n\pi }\int_{0 }^{L} x(L-x) d(\cos \dfrac{ n\pi }{L} x ) \\
	&= -\dfrac{L}{ n\pi } [x(L-x)\cos \dfrac{ n\pi }{L} x ]_{0 }^{L} + \dfrac{L}{ n\pi } \int_{0 }^{L} \cos \dfrac{ n\pi }{L}xd( x(L-x)  ) \\
	&= \dfrac{L}{ n\pi } \int_{0 }^{L} (L-2x)\cos \dfrac{ n\pi }{L}xdx \\
	&=(\dfrac{L}{ n\pi })^2 \int_{0 }^{L} (L-2x) d(\sin \dfrac{ n\pi }{L}x) 
\end{aligned}
$$
$$
\begin{aligned}
	~~~~~~ 
	&= (\dfrac{L}{ n\pi })^2 \left\{[(L-2x)\sin \dfrac{ n\pi }{L} x ]_{0 }^{L} - \int_{0 }^{L} \sin \dfrac{ n\pi }{L}xd( L-2x )\right\} \\
	&= - (\dfrac{L}{ n\pi })^2 (-2) \int_{0 }^{L} \sin \dfrac{ n\pi }{L}xdx \\
	&= -2(\dfrac{L}{ n\pi })^3 [\cos \dfrac{ n\pi }{L} x ]_{0 }^{L}\\ 
	&= 2 (\dfrac{L}{ n\pi })^3  [1-\cos n \pi ] 
\end{aligned}
$$ 
对于复杂一点的函数,需要运用多次分部积分才能完成, 过程比较繁杂, 容易出错。\\
(2)列表法 \\
设$ g(x)$ 是 $ k(x)$ 的$ n$阶导数, 则有 
$$
\int_a ^b f(x)g(x)dx = \int_a ^b f(x)k^{(n)}(x)dx   
$$ 
进行分部积分
$$
\begin{aligned}
	\int_a ^b f(x)g(x)dx &= \int_a ^b f(x)k^{(n)}(x)dx \\ 
	&= \int_a ^b f(x)d(k^{(n-1)}(x)) \\
	&= \left[ fk^{(n-1)}\right] _a ^b - \int_a ^b k^{(n-1)}(x)d(f(x)) \\
	&= \left[ fk^{(n-1)}\right] _a ^b +(-1)^1 \int_a ^b f'(x)k^{(n-1)}(x)dx \\
\end{aligned} 
$$
这是一个递推式,
$$
\begin{aligned}
	  \int_a ^b f(x)g(x)dx &=\left[ fk^{(n-1)} - f'k^{(n-2)}  \right] _a ^b + (-1)^2 \int_a ^b f^{(2)}(x)k^{(n-2)}(x)dx \\
	  &=\left[ fk^{(n-1)} - f'k^{(n-2)} +  f''k^{(n-3)} \right] _a ^b + (-1)^3 \int_a ^b f^{(3)}(x)k^{(n-3)}(x)dx \\
	  &\cdots \\
	  &= \left[ fk^{(n-1)} - f'k^{(n-2)} + \cdots \pm f^{(n-1)}k\right]_a ^b  + (-1)^n \int_a ^b f^{(n)}(x)k(x)dx
  \end{aligned} 
$$
$ \bullet $ 如果存在 $ f^{(n)}(x) = 0 $, 则有 
$$
\begin{aligned}
	\int_a ^b f(x)g(x)dx = \left[ fk^{(n-1)} - f'k^{(n-2)} + \cdots + f^{(n-1)}k\right]_a ^b \\
\end{aligned} 
$$
余下部分可查表
  \renewcommand\arraystretch{1.5}  
\begin{table}[h]\label{tab:lb}
	  \caption{列表法求分部积分}
	  \centering
	  \begin{tabular}
	  {p{0.5cm}c|c}	
	  \toprule
	  & $f(x)$  & $k^{(n)}(x)=g(x)$ \\
	  & $f'(x)$  & $k^{(n-1)}(x)$ \\
	  & $f''(x)$  & $k^{(n-2)}(x)$ \\
	  & $\dots $  & $\dots $ \\
	  & $f^{(n)}(x)$  & $k(x)$ \\
	  \bottomrule
	  \end{tabular}
  \end{table}
~~\\ 
表的第一列为$f(x)$的各阶导数:$f'(x),f''(x), f'''(x), \cdots, f^{(n)}(x)$, 表的第二列为$ g(x)$的各阶原函数:$k^{(n)}(x), k^{(n-1)}(x), k^{(n-2)}(x),\cdots, k(x)$。\\
$ \bullet $ 如果存在 $$ \int_a ^b f^{(n)}(x)k(x)dx = \int_a ^b f(x)k^{(n)}(x)dx $$
且$ n $为奇数, 则有
$$ 
\begin{aligned}
	\int_a ^b f(x)g(x)dx = \frac{1}{2}\left[ fk^{(n-1)} - f'k^{(n-2)} + \cdots + f^{(n-1)}k\right]_a ^b \\
\end{aligned} 
$$
余下部分还是查表。

~~\\
本例要求计算 $$ \int_{0 }^{L} x(L-x) \sin \dfrac{ n\pi }{L} x dx  $$
取 $f(x) = x(L-x)$, 存在 $  f^{(3)} =0  $, 因此可用列表法。  先计算表格: \\
\renewcommand\arraystretch{2}  
\begin{table}[h]
	\caption{列表法求$ \int_{0 }^{L} x(L-x) \sin \dfrac{ n\pi }{L} x dx  $}
	\centering
	\begin{tabular}
	{p{0.5cm}c|c}	
	\toprule
	& $f(x)=x(L-x)$  & $k'''(x)=g(x)= \sin \dfrac{ n\pi }{L} x$ \\
	& $f'(x)=L-2x$  & $k''(x) = -(\dfrac{L}{ n\pi })\cos \dfrac{ n\pi }{L} x$ \\
	& $f''(x)=-2$  & $k'(x) = -(\dfrac{L}{ n\pi })^2 \sin \dfrac{ n\pi }{L} x$ \\
	& $f'''(x)=0$  & $k(x) = (\dfrac{L}{ n\pi })^3 \cos \dfrac{ n\pi }{L} x$ \\
	\bottomrule
	\end{tabular}
\end{table}

~~\\ 
然后计算
$$
\begin{aligned}
	\int_{0 }^{L} x(L-x) \sin \dfrac{ n\pi }{L} x dx &= \left[ fk'' - f'k' + f''k \right]\left\vert _{0 }^{L} \right. \\
\end{aligned}  
$$ 
查表:第一项有 $ f(0)=f(L) =0 $, 第二项有 $ k'(0)=k'(L) =0 $, 它们都为零,只要计算第三项 
$$
\begin{aligned}
	\int_{0 }^{L} x(L-x) \sin \dfrac{ n\pi }{L} x dx &= f''k \left\vert _{0 }^{L} \right. \\
	&= \left.(-2) (\frac{L}{ n\pi })^3 \cos \frac{ n\pi }{L} x \right\vert _{0 }^{L} \\
	&=  2 (\frac{L}{ n\pi })^3 [1 - \cos n\pi]
\end{aligned}  
$$  

\subsection{三类边界条件} ~\\
第一类边界条件: 在一维热传导问题中,边界条件的数学表示为:
\begin{equation}
	u (0,t) =0, \quad u (l,t)=0 
\end{equation}
物理上, 这是恒温边界。比如导线两端都放在一个温度为零的巨大的冰水混合物中, 则边界温度是不随时间变化的。或者说,系统的两端与一个温度恒定(比如为零)的大热源相连接。也称狄利克雷(Dirichlet)边界条件。

第二类边界条件: 导线两端的热通量恒定,比如它们与绝热隔板相连,或存在对称边界或热平衡边界,称为诺依曼边界条件。当热通量为零时,数学表示为:
\begin{equation}
	u_x (0,t) =0, \quad u_x (l,t)=0
\end{equation}
 
第三类混合边界条件: 导线一端是第一类边界另一端是第二类边界,数学表示为:
\begin{equation}
	u (0,t) =0, u_x (l,t)=0 
\end{equation} 
以及
\begin{equation}
	u_x (0,t) =0, u (l,t)=0 
\end{equation} 

当然,实际问题会具有更为复杂的边界条件,比如对流边界。此时,边界处于温度为$T_\infty $的流体中,设物体温度为$T_s$,它与流体间的传热系数为$h$, 则有 
\begin{equation}
	k u_x (0,t) = k u_x(l,t)= h (T_s - T_\infty) 
\end{equation} 
总之边界条件描述了导线两端所处的环境或与环境进行热量交换的方式。因此,虽然边界条件的测量误差不会引起解的重大变化,但边界类型的改变,必然导致方程的解发生重大变化。比如复杂的边界条件会导致热传导问题不可解。
~~\\ 

\begin{example}
	求解第二类边界条件热传导方程
	$$
	\begin{cases}
		u_{t} =a^2u_{xx} ~~,~~ 0<x<l, t>0\\
		u_x (0,t) =0, u_x (l,t)=0 \\
		u(x,0) =\psi(x)
	\end{cases}
	$$ 
\end{example}
\emph{解:}
	方程满足分离变量三条件, 令 $$\displaystyle  u(x,t)=T(t)X(x) $$
分离变量,得两个常微分方程 \\
方程(I):\\
$$\displaystyle  \begin{cases}
	X~^{''} +\lambda X=0  ~~,~~ 0<x<l\\
	X'(0)=0 ~,~X' (l)=0
\end{cases}$$ 	
方程(II):\\
$$\displaystyle  \begin{cases}
	T^{'} +\lambda {a~^2 T}=0 \\
	......
\end{cases}$$ 	
方程(I)是第二类边界条件下的固有值问题,可称为固有值问题II。
基于以前的讨论,只有在$\lambda >0 $ 即特征方程有虚根时,方程才有非零解\\
	\begin{equation} \label{eq:two}
		X=A\cos \sqrt{\lambda} x + B \sin \sqrt{\lambda} x
	\end{equation}
为了应用第二类边界条件确定系数,先对上式求导:
	\begin{equation*}
		X' (x)=\sqrt{\lambda} (-A\sin \sqrt{\lambda} x + B \cos \sqrt{\lambda} x)
	\end{equation*}
分别取$x=0, x=l$,得两个表达式, 结合导数零边界条件,得矩阵方程 
$$\left[
	\begin{array}{lll}
		0&1\\
		-\sin( {\sqrt{\lambda}~l}) &\cos ({\sqrt{\lambda}~l})
	\end{array}
	\right]
	\left[
	\begin{array}{ll}
		A\\
		B
	\end{array}
	\right]
	=\left[
	\begin{array}{ll}
		0\\
		0
	\end{array}
	\right]$$
	由系数行列式为零,得$$ \sin\sqrt{\lambda}~l =0$$
	因此有
	$$\sqrt{\lambda}~l = n\pi \quad $$ 
	解得\\
	固有值:
	\begin{equation*}
		\lambda_n =(\frac{n\pi}{l})^2 
	\end{equation*}
	把固有值代回矩阵方程,得系数 $$ [A_n, B_n] ^T =[1, 0] ^T$$
	把它们代回[\ref{eq:two}]式,得 \\
	固有函数:
	\begin{equation*}
		X_n (x)=\cos \frac{n\pi}{l} x
	\end{equation*}
	当$n=0$时,固有函数$X_0 (x)=1$, 这是非零解,代表着体系温度处处相等时的零频状态。\\
	因此有
	\[ n=0, 1, 2, 3, \cdots \]
	第二类边界条件带来两大影响
  \begin{itemize}
	\item 固有函数变为$X_n (x)=\cos \dfrac{n\pi}{l} x$  
	\item 固有值从$n=0$开始
  \end{itemize}
	
  ~~\\ 
  把固有值$\lambda_n =(\dfrac{n\pi}{l})^2 $代回方程(II),得
	\begin{equation*}
		T_n^{'} +(\dfrac{n\pi}{l})^2 T_n=0 
	\end{equation*}
	这是数学哀减模型,
	有解 
	$$
	T_n (t)=  \exp(-(\frac{n\pi a}{l})^2 t)
	$$ 
	原方程基本解
	$$
	u_n(x,t)= T_n (t) X_n (x)=  \exp(-(\frac{n\pi a}{l})^2 t) \cos \frac{n\pi~}{l} x
	$$ 
	叠加解:
	\begin{equation*}
		u(x,t)=\sum\limits_{n=0}^{\infty } C_n  \exp(-(\frac{n\pi a}{l})^2 t) \cos \frac{n\pi~}{l} x
	\end{equation*}
	取$t=0$,结合初值条件, 有
    $$
	u(x,0) =\psi(x) =  \sum\limits_{n=0}^{\infty } C_n  \exp(-(\frac{n\pi a}{l})^2 0) \cos \frac{n\pi~}{l} x
	$$ 
	得
	$$
	\psi(x) =  \sum\limits_{n=0}^{\infty } C_n \cos \frac{n\pi~}{l} x
	$$ 
	考虑到定义域为($0,l$)给傅里叶级数公式带来的影响,有
	$$
	C_n = \dfrac{2}{l} \int_{0}^{l} \psi(x) \cos \dfrac{n\pi}{l} xdx
	$$
	因此,方程的解为
	$$ \left\{
	\begin{aligned}
				&u(x,t)=\sum\limits_{n=0}^{\infty } C_n  \exp(-(\dfrac{n\pi a}{l})^2 t) \cos \dfrac{n\pi~}{l} x \\
				&C_n = \dfrac{2}{l} \int_{0}^{l} \psi(x) \cos \dfrac{n\pi}{l} xdx \\ 
	\end{aligned} \right.
	$$ 
	把零频项从求和记号里单独取出,有
	$$
	\boxed{\begin{aligned}
			& u(x,t)=\frac{C_0}{2} + \sum\limits_{n=1}^{\infty } C_n \exp(-(\frac{n\pi a}{l})^2 t) \cos \frac{n\pi~}{l} x \\
			& C_0 = \dfrac{2}{l} \int_{0}^{l} \psi(x) dx \\
			& C_n = \dfrac{2}{l} \int_{0}^{l} \psi(x) \cos \dfrac{n\pi}{l} xdx 
		\end{aligned}}  
	$$ 
\begin{example}
	求定常问题
	$$\displaystyle  \begin{cases}
		u_{t} =u_{xx} ~~,~~ 0<x<\pi, t>0\\
		u_x (0,t) =0, u_x (l,t)=0 \\
		u(x,0) =x^2 (\pi-x)^2
	\end{cases}$$ 	
\end{example}
\emph{解:}
	导数边界条件确定的固有值和固有函数为\\
	固有值:$$\displaystyle  \lambda~_n=\dfrac{n^2\pi~^2}{l~^2 }= (\dfrac{n\pi }{\pi}) ^2 = n^2 \qquad (n=0,1,2,3,\cdots)$$  
	固有函数:$$\displaystyle  X~_n=\cos \dfrac{n\pi~}{l} x=\cos nx $$
	叠加解: 
	$$
	\begin{aligned}
			u(x,t)&=  \sum\limits_{n=0}^{\infty } C_n  \exp(-(\dfrac{n\pi a}{l})^2 t) \cos \dfrac{n\pi~}{l} x\\ \\ 
		\end{aligned} 
	$$ 
	求展开系数,由于存在$n=0$项,需要单独计算
	$$
	\begin{aligned}
		C_0 &= \dfrac{2}{l} \int_{0}^{l} \psi(x) \cos \dfrac{n\pi}{l} xdx \\
		&= \dfrac{2}{\pi} \int_{0}^{\pi}  x^2 (\pi-x)^2 dx  \\
		&= \dfrac{2}{\pi}\times\frac{\pi^5}{30} \\
		&=\dfrac{\pi ^4}{15}
	\end{aligned}
	$$ 
	计算其他项
	$$
	\begin{aligned}
		C_n&=\dfrac{2}{l} \int_{0}^{l} \psi(x) \cos \dfrac{n\pi}{l} xdx \\
		&= \dfrac{2}{\pi} \int_{0}^{\pi}   x^2 (\pi-x)^2   \cos nx dx \\
		&= \dfrac{2}{\pi} \times (-\dfrac{12 \pi}{n^4})  [\cos n\pi +1 ] \\
		&= -\dfrac{24}{n^4} [ (-1)^n +1 ] \\
		&= -\dfrac{3}{k^4}\quad (n=2k)
	\end{aligned}
	$$ 
代回, 得解函数
$$\begin{aligned}
		 u(x,t)&=\frac{C_0}{2} + \sum\limits_{n=1}^{\infty } C_n \exp(-(\frac{n\pi a}{l})^2 t) \cos \frac{n\pi~}{l} x \\
         &= \dfrac{\pi ^4 /15 }{2} + \sum\limits_{k=1}^{\infty } (-\dfrac{3}{k^4}) \exp(-(\frac{2k\pi}{l})^2 t) \cos \frac{2k\pi~}{l} x \\
		 &= \dfrac{\pi ^4 }{30} - \sum\limits_{k=1}^{\infty } \dfrac{3}{k^4} \exp(-(\frac{2k\pi}{l})^2 t) \cos \frac{2k\pi~}{l} x
	\end{aligned}
	$$ 
$\star$定积分计算
\renewcommand\arraystretch{1.8}
\begin{table}[h]
	\caption{列表法求分部积分$ \displaystyle \int_{0}^{\pi} (\pi-x)^2 x^2  dx  $}\label{<label>}
	\centering
	\begin{tabular}
	{p{0.5cm}c|c}	
	\toprule
	& $f(x)=(\pi-x)^2$  & $k'''(x) = x^2$ \\
	& $f'(x)=-2(\pi-x)$  & $k''(x) = \dfrac{1}{3} x^3$ \\
	& $f''(x)= 2$  & $k'(x) = \dfrac{1}{12} x^4$ \\
	& $f'''(x)= 0$  & $k(x) = \dfrac{1}{60} x^5$ \\
	\bottomrule
	\end{tabular}
\end{table}
~~\\ 
由于第一项, 第二项都为零,只要计算第三项 
$$
\begin{aligned}
	\int_{0}^{\pi} (\pi-x)^2 x^2  dx &= f''k \left\vert _{0 }^{L} \right. = 2 \dfrac{1}{60} x^5 \left\vert_{0}^{\pi} \right.= \frac{\pi^5}{30}
\end{aligned}  
$$  
~~\\ 
\renewcommand\arraystretch{1.8}
\begin{table}[h]
	\caption{列表法求分部积分$ \displaystyle \int_{0}^{\pi}   x^2 (\pi-x)^2   \cos nx dx  $}\label{<label>}
	\centering
	\begin{tabular}
	{p{0.5cm}c|c}	
	\toprule
	& $f(x)=x^2 (\pi-x)^2$  & $k^{(5)}(x) = \cos nx $ \\
	& $f'(x)=2x(2x - \pi )(x - \pi )$  & $k^{(4)}(x) = \dfrac{1}{n} \sin nx $ \\
	& $f''(x)= 2\pi ^2 -12\pi x+12x^2$  & $k'''(x)= -\dfrac{1}{n^2} \cos nx $ \\
	& $f'''(x)= 24x -12\pi$  & $k''(x) = -\dfrac{1}{n^3} \sin nx $ \\
	& $f^{(4)}(x)= 24$  & $k'(x) = \dfrac{1}{n^4} \cos nx $ \\
	& $f^{(5)}(x)= 0$  & $k(x) = \dfrac{1}{n^5} \sin nx $ \\
	\bottomrule
	\end{tabular}
\end{table}
~~\\ 
同理,有第一、二、三、五项都为零,只要算第四项
$$
\begin{aligned}
	\int_{0}^{\pi}   x^2 (\pi-x)^2   \cos nx dx &= - f^{'''}k' \left\vert_{0}^{\pi} \right. \\
	&= - (24x -12\pi) \dfrac{1}{n^4} \cos nx \left\vert_{0}^{\pi} \right.\\
	&= -\dfrac{12 \pi}{n^4}  [\cos n\pi +1 ]
\end{aligned}  
$$ 
~~\\
同理,可以解得混合边界条件的方程 $X'' + \lambda X =0, \quad 0<x<L$ 的固有值及固有函数, 小结如下\\
$$
\begin{aligned}
	&\left.u\right|_{x=0}=0,& \left.u\right|_{x=L}=0,\quad & \lambda_n=\left(\frac{n \pi}{L}\right)^2, \quad & X_n(x)=B_n \sin \frac{n \pi}{L} x, \quad n=1,2,3, \cdots \\
	&\left.u_x\right|_{x=0}=0,&\left.u_x\right|_{x=L}=0,\quad & \lambda_n=\left(\frac{n \pi}{L}\right)^2, \quad & X_n(x)=A_n \cos \frac{n \pi}{L} x,\quad n=0,1,2, \cdots \\
	&\left.u\right|_{x=0}=0,&\left. u_x\right|_{x=L}=0,\quad & \lambda_n=\left[\frac{(2 n+1) \pi}{2 L}\right]^2, \quad & X_n(x)=B_n \sin \frac{(2 n+1) \pi}{2 L} x, \quad n=0,1,2, \cdots \\
	&\left.u_x\right|_{x=0}=0,& \left. u\right|_{x=L}=0,\quad & \lambda_n=\left[\frac{(2 n+1) \pi}{2 L}\right]^2, \quad & X_n(x)=A_n \cos \frac{(2 n+1) \pi}{2 L} x,\quad n=0,1,2, \cdots
\end{aligned}
$$
注意,除第一个的固体值从$n=1$开始取值外,其余三个都是从$n=0$开始。试结合固有函数说明原因。
~~\\ 
~~\\ 

\section{拉普拉斯方程}

拉普拉斯方程是由法国数学家皮埃尔-首先提出而得名。1799年,西蒙·拉普拉斯(Pierre-Simon Laplace)在证明太阳系是一个稳定系统的过程中建立并求解了一个有关引力势的方程,这就是著名的拉普拉斯方程。拉普拉斯方程的解函数是位势函数,也称调和函数(harmonic function),因此拉普拉斯方程也称为位势方程或调和方程。我们知道, 传导方程描述的一个系统从不平衡到平衡的过程, 拉普拉斯方程描述的是一个平衡系统的状态。因此拉普拉斯方程的提出推翻了一个世纪前牛顿关于天体系统不稳定的假设,使人们对于天体系统和势场的认识都前进了一大步。

势函数常用来描写引力场、电磁场、流场等保守场或有势场,因此,拉普拉斯方程在电磁学、光学、天文学和流体力学等一切与势场相关的领域都有着广泛的应用。

\subsection{方程的建立}

\begin{example}
试建立引力势函数$ u(x,y,z) $ 所满足的方程。
\begin{figure}[htbp]
	\centering
	\includegraphics[width=0.2\textwidth]{figs/2022-12-11-22-25-27.png}
	\caption{质点M的引力势}
	%\label{}
\end{figure}
\end{example}
\emph{解:}
以质点M为原点,建立如图坐标系,空间任意一点($x,y,z$)质量为$m$的试验质点感受到来自$M$的引力为\\
   $$ \overrightarrow{F} =-G\dfrac{Mm}{r^3} \overrightarrow {r}~,~~ r=\sqrt{x^2+y^2+z^2}$$
  因此,M激发的引力场强为 \\
  $$ \overrightarrow{A} =\dfrac{GM}{r^3} \overrightarrow{r} $$
考虑M为有限大小的均匀球体(密度为$\rho$),则封闭球面S内的场强通量为 \\
$$\displaystyle \oint_{S} \overrightarrow{A} \cdot d \overrightarrow{S} = \frac{GM_V}{r^2} 4\pi r^2 = 4\pi GM_V = 4\pi G \int_V \rho d\tau $$
运用高斯定理:\\
  $$ \begin{aligned}
	  \int_V  \nabla \cdot \overrightarrow{A} d\tau = \oint_{S} \overrightarrow{A} \cdot d \overrightarrow{S}
	  = \int_V 4\pi G \rho d\tau
  \end{aligned}  $$
  得 
  \begin{equation*}
	  \nabla \cdot \overrightarrow{A} = 4\pi G \rho
  \end{equation*}
场强是位势函数的梯度
  \begin{equation}\label{eq:force}
	  \overrightarrow{A} =-\nabla u
  \end{equation} 
代入 $$\nabla \cdot \overrightarrow{A} = \nabla \cdot \left(-\nabla u\right)= -\nabla ^2 u$$ 
因此
  \begin{equation}
	  \nabla ^2 u= -4\pi G \rho 
  \end{equation}
这就是著名的泊松方程
~~\\ 
对于无源区域,泊松方程变为拉普拉斯方程
\begin{equation} \label{eq:laplace}
	\boxed{\nabla ^2  u =0} 
\end{equation}
定义拉普拉斯算子
\begin{equation*}
	\triangle  = \nabla ^2 
\end{equation*}
拉普拉斯方程可写成
\begin{equation*}
	\triangle   u =0 
\end{equation*}   

~~\\ 

\begin{example}
试从麦克斯韦方程导出拉普拉斯方程 
\end{example}
\emph{解:}
 麦克斯韦方程组为
			\[ \begin{aligned}
					\text{I}~~~ \hspace*{2em}&\nabla \cdot \mathbf{D} =\rho _f  \\  
					\text{II}~~ \hspace*{2em}&\nabla \cdot \mathbf{B} = 0  \\  
					\text{III}~ \hspace*{2em}&\nabla \times  \mathbf{E} = -\cfrac{\partial \mathbf{B}}{\partial t }  \\  
					\text{IV~}~ \hspace*{2em}&\nabla \times  \mathbf{H} = \mathbf{J}_f +  \cfrac{\partial \mathbf{D}}{\partial t } 
				\end{aligned} \]
首先,考虑电场, 有矢量公式
\begin{equation}\label{eq:vect}
  \begin{aligned}
	  \nabla \times (\nabla \times  \mathbf{E}) &=  \nabla (\nabla \cdot  \mathbf{E})- \nabla^2 \mathbf{E} 
  \end{aligned}
\end{equation}
式中含拉普拉斯算子,现在计算前两项。
对麦克斯韦方程第三式求旋度
  $$
  \begin{aligned}
  \nabla \times(\nabla \times \mathbf{E}) & =-\nabla \times \frac{\partial \mathbf{B}}{\partial t} \\
  &=-\frac{\partial}{\partial t} \nabla \times \mathbf{B} 
  \end{aligned}
  $$
  代入物质方程
$$
\mathbf{B}=\mu \mathbf{H}
$$
有
$$
\begin{aligned}
\nabla \times(\nabla \times \mathbf{E}) 
& =-\mu \frac{\partial}{\partial t} \nabla \times \mathbf{H}\\
&=-\mu \frac{\partial}{\partial t}\left(\mathbf{J}+\frac{\partial \mathbf{D}}{\partial t}\right) \\
& =-\mu\left(\frac{\partial \mathbf{J}}{\partial t}+\frac{\partial^2 \mathbf{D}}{\partial t^2}\right)
\end{aligned}
  $$
  代入物质方程
$$
\mathbf{J}=\sigma \mathbf{E}, \quad \mathbf{D}=\varepsilon \mathbf{E}  
$$
得
\begin{equation}\label{eq:vect1}
\nabla \times(\nabla \times \mathbf{E})=-\mu \sigma \frac{\partial \mathbf{E}}{\partial t}-\mu \varepsilon \frac{\partial^2 \mathbf{E}}{\partial t^2} 
\end{equation}
对麦克斯韦方程第一式求梯度
  $$
  \nabla(\nabla \cdot \mathbf{E})=\frac{1}{\varepsilon} \nabla(\nabla \cdot \mathbf{D})=\frac{1}{\varepsilon} \nabla \rho
  $$
  电荷密度$\rho$不显含$\vec{r}$时, 有
  \begin{equation}\label{eq:vect2}
  \nabla(\nabla \cdot \mathbf{E})=0
  \end{equation}  
把[\ref{eq:vect1}]、[\ref{eq:vect2}]式代回[\ref{eq:vect}]式,得
电场方程
\begin{equation}
	\mu \varepsilon \frac{\partial^2 \mathbf{E}}{\partial t^2}+\mu \sigma \frac{\partial \mathbf{E}}{\partial t}=\nabla^2 \mathbf{E}
	\end{equation}
同理,得磁场方程
\begin{equation}
	\mu \varepsilon \frac{\partial^2 \mathbf{H}}{\partial t^2}+\mu \sigma \frac{\partial \mathbf{H}}{\partial t}=\nabla^2 \mathbf{H}
	\end{equation}
	式中一阶项描述了由电导率$ \sigma  $ 刻画的阻尼。阻尼为零时,有
	$$
	\frac{\partial^2 \mathbf{E}}{\partial t^2}=\frac{1}{\mu \varepsilon} \nabla^2 \mathbf{E}, \quad \frac{\partial^2 \mathbf{H}}{\partial t^2}=\frac{1}{\mu \varepsilon} \nabla^2 \mathbf{H}
	$$
	这是矢量波动方程,是相当难以求解的。  
\\
若考虑位势函数$u$,则问题会简单得多
$$
\mathbf{E}=-\nabla u
$$
因此,如果已知函数$u$,求梯度既得电场, 现建立势函数$u$满足的方程
\begin{equation*}
\begin{aligned}
	\nabla^2 u &= \nabla \cdot \nabla u =-\nabla \cdot \mathbf{E} 
	=-\frac{1}{\varepsilon} \nabla \cdot \mathbf{D} 
	=-\frac{\rho}{\varepsilon}
\end{aligned}
\end{equation*}
这正是泊松方程。对于无源场,则满足拉普拉斯方程
$$
\nabla^2 u =0 
$$ 
波动方程与泊松方程(拉普拉斯方程)就这样联系在一起。即:电场矢量满足波动方程,位势函数满足拉普拉斯方程。如果是电荷激发的电磁场,则位势函数满足泊松方程。
\begin{example}
	试证明如果一个复变函数是全纯函数,则它的实部和虚部函数都满足拉普拉斯方程。
\end{example}
\begin{proof}
	设复变函数为
	\[ f(z) =u(x,y) + i v(x,y)\]
	全纯函数必满足柯西-黎曼方程
	\[ u_x = v_y, \qquad  u_y = - v_x\]
	在二维直角坐标拉普拉斯算子中代入柯西-黎曼方程 
	\[ \nabla^2 u(x,y) = \frac{\partial}{\partial x} (\frac{\partial u}{\partial x}) + \frac{\partial}{\partial y} (\frac{\partial u}{\partial y})  = \frac{\partial}{\partial x} (\frac{\partial v}{\partial y}) - \frac{\partial}{\partial y} (\frac{\partial v}{\partial x}) \]
	得拉普拉斯方程
	\[ \nabla^2 u(x,y)  =0\]
	同理,得 
	\[ \nabla^2 v(x,y)  =0\]
\end{proof}

\subsection{方程的求解} ~\\

有源空间的势函数服从泊松方程, 无源空间的势函数服从拉普拉斯方程, 拉普拉斯方程是齐次的泊松方程, 因此,求解拉普拉斯是求解泊松方程的基础。 

很明显, 拉普拉斯方程是个泛定方程。它通常是三维的, 本章考虑二维的拉普拉斯方程的求解。二维区域的形态有很多种,我们考虑两种最简单的
\begin{itemize}
	\item 矩形域
	\item 圆域
\end{itemize}

\begin{example}
	求解矩形区域拉普拉斯方程
	\begin{equation}\label{eq:laplace6}
	\displaystyle \begin{cases}
		u_{xx} +u_{yy} =0 ,~~~~ (0<x<a, 0<y<b)  \quad \cdots \quad  (1)\\
		u(x,0)= f_1 (x) , \quad u(x,b)= f_2 (x) \quad \cdots \quad (2)\\
		u(0,y)= g_1 (y) , \quad u(a,y)= g_2 (y) \quad \cdots \quad (3)
	\end{cases}
   \end{equation}
   \begin{figure}[h]
	\centering
	\includegraphics[width=0.5\textwidth]{figs/four.png}
	\caption{矩形区域拉普拉斯方程定解问题}
	\label{fig:four}
    \end{figure}
\end{example}
	这是第一类边界条件,但不是零边界条件,根据边值问题的线性性质,可分解成两个零边界条件方程。如果令 $x=t$, 方程(1)化为波动方程,方程(2)是边界条件,方程(3)是初始条件。边界条件所确定的固有值问题的解与初始条件是无关的,因此方程(3)可以被置零,得如下方程(A)。同理,若令$y=t$,则可得方程(B)
	$$\displaystyle (A) \begin{cases} 
		u_{xx} +u_{yy} =0 ,~~~~ (0<x<a, 0<y<b) \\
		u(x,0)= f_1 (x) ,  u(x,b)= f_2 (x) \\
		u(0,y)= 0,  u(a,y)= 0 
	\end{cases}$$ 
	$$\displaystyle (B) \begin{cases}
		u_{xx} +u_{yy} =0 ,~~~~ (0<x<a, 0<y<b)\\
		u(x,0)= 0,  u(x,b)= 0 \\
		u(0,y)= g_1 (y) ,  u(a,y)= g_2 (y)
	\end{cases}$$ 
现在方程(A)、(B)都是可解的一类零边界固有值问题。它们的解都是原方程的解。

当然,方程可进一步分解成如下四个定解问题:
	$$\displaystyle  (I) \begin{cases}
		u_{xx} +u_{yy} =0 ,~~~~ (0<x<a, 0<y<b)\\
		u(x,0)= 0,  u(x,b)= 0 \\
		u(0,y)= g_1 (y) ,  u(a,y)= 0
	\end{cases}$$ 
	$$\displaystyle (II)  \begin{cases}
		u_{xx} +u_{yy} =0 ,~~~~ (0<x<a, 0<y<b)\\
		u(x,0)= 0,  u(x,b)= 0 \\
		u(0,y)= 0,  u(a,y)= g_2 (y) 
	\end{cases}$$ 
	$$\displaystyle  (III)  \begin{cases}
		u_{xx} +u_{yy} =0 ,~~~~ (0<x<a, 0<y<b)\\
		u(x,0)= f_1 (x) ,  u(x,b)= 0 \\
		u(0,y)= 0,  u(a,y)= 0 
	\end{cases}$$ 	
	$$\displaystyle  (IV)  \begin{cases}
		u_{xx} +u_{yy} =0 ,~~~~ (0<x<a, 0<y<b)\\
		u(x,0)= 0,  u(x,b)= f_2 (x) \\
		u(0,y)= 0,  u(a,y)= 0 
	\end{cases}$$ 
四个方程在数学上是同类的,只需求解一个。现以方程-(II)为例,演示求解过程。
~~\\ 

\begin{example}
 求解拉普拉斯方程的定解问题
 \begin{equation}\label{eq:laplace2}
 \displaystyle \begin{cases}
	u_{xx} +u_{yy} =0 ,~~~~ (0<x<1, 0<y<1)\\
	u(x,0)= 0, \quad u(x,1)= 0 \\
	u(0,y)=0, \quad u(1,y)= \sin \pi y
\end{cases}  
\end{equation}
\end{example}
\emph{解:}
方程符合分离变量三条件,令 $ u(x,y)=X(x)Y(y)$ ,代回,得 
\begin{equation*}
	X^{''}(x)Y(y) +X(x)Y^{''}=0
\end{equation*}
整理,得
\begin{equation*}
	-\frac{X~^{''}}{X}=\frac{Y~^{''} }{Y}
\end{equation*}	
令上式等于$-\lambda$, 分离出两个常微分方程 \\
方程(1):\\
$$\displaystyle  \begin{cases}
	Y~^{''} +\lambda Y=0  ~~,~~ 0<y<1\\
	Y(0)=0 ~~,~~Y(1)=0 
\end{cases}$$ 	
方程(2):\\
$$\displaystyle  \begin{cases}
	X~^{'} -\lambda X=0  ~~,~~ 0<x<1 \\
	X(0)=0~~,~~X(1)=\sin \pi y 
\end{cases}$$ 
~~\\ 
方程(2)是双变量问题,但方程(1)是原固有值问题-I ($l=1$),有\\
固有值:$$\displaystyle  \lambda_n=\frac{n^2\pi~^2}{l~^2} =n^2\pi~^2, \quad (n=1,2,3,\cdots) $$  
固有函数: $$\displaystyle  Y_n(x)= \sin \frac{n\pi~}{l} y = \sin n \pi y $$
把$\lambda_n$代入方程(2), 得:
$$\displaystyle  X~^{''} - n^2\pi~^2~X=0 $$ 
特征方程有两相异实根($ \pm n\pi $),得通解 
\begin{equation*}
	X_n(x)=C_n \exp(n\pi x )+ D_n \exp(-n\pi x )
\end{equation*}	
结合,得基本解:
$$\displaystyle \begin{aligned}
	u_n(x,y)&= X_n (x)Y_n (x) \\
	&= [C_n \exp(n\pi x )+ D_n \exp(-n\pi x )] \sin (n \pi y)  
\end{aligned}$$ 
有限体系$e$指数形式的解难以定系数,改用三角函数。由双曲函数定义
$$\displaystyle \begin{aligned}
	&\sinh(x) = \dfrac{e^x -e^{-x}}{2} \\
	&\cosh(x) = \dfrac{e^x +e^{-x}}{2} 
\end{aligned}$$
\begin{figure}[htbp]
	\centering
	\includegraphics[width=0.5\textwidth]{figs/sinh.png}
	\caption{双曲函数与指数函数对比图}
	\label{fig:sinh}
\end{figure} 
联立两式,可解得
$$\displaystyle \begin{aligned}
	& e^x = \cosh(x) +\sinh(x) \\
	& e^{-x} =\cosh(x)-\sinh(x) 
\end{aligned}$$ 
图[\ref{fig:sinh}]给出了双曲函数与指数函数图形对比。因此, 基本解可写成
$$\displaystyle \begin{aligned}
	u_n(x,y)&= [a_n \cosh (n\pi x )+ b_n \sinh(n\pi x ) ]\sin (n \pi y)
\end{aligned}$$ 
叠加解为
\begin{equation*}
	u(x, y)   = \sum\limits_{n=1}^{\infty }  [a_n \cosh (n\pi x )+ b_n \sinh (n\pi x ) ] \sin (n \pi y)  
\end{equation*}	
代入定解条件:$ u(0,y)=0$, 得 
$$ \sum\limits_{n=1}^{\infty }  a_n  \sin (n \pi y) =0 $$ 
有$ a_n=0$,代回,有 
$$	u(x,y)    = \sum\limits_{n=1}^{\infty }  b_n \sinh (n\pi x )  \sin (n \pi y)  $$ 
代入定解条件:$u(1,y) = \sin \pi y $, 得
\begin{equation*}
	u(1,y) = \sum\limits_{n=1}^{\infty }  b_n \sinh (n\pi )  \sin (n \pi y)  = \sin \pi y
\end{equation*}
由半幅傅里叶级数公式, 有
\begin{equation*}
	b_n \sinh (n\pi ) = \frac{2}{1} \int_0^1 \sin \pi y \sin (n \pi y)dy
\end{equation*}
由固有函数正交性可知,只有在$n=1$时,右边的积分才不为零, 因此有
$$ b_1\sinh\pi =2 \int_0^1 \sin \pi y \sin \pi y dy$$	
解得 
$$ b_1 =\frac{1}{\sinh\pi}, ~~ b_n=0,~(n>1)$$
代回,得原方程解函数
\begin{equation*}
	u(x,y) = \dfrac{\sinh \pi x}{\sinh \pi}  \sin ( \pi y) 
\end{equation*}	
解得的势函数如图[\ref{fig:heat1}]所示。
\begin{figure}[htbp]
	\centering
	\includegraphics[width=0.5\textwidth]{figs/heat1.png}
	\caption{势函数$u(x,y)$的图形}
	\label{fig:heat1}
\end{figure}

~~\\ 
同理,可求出其他三个方程的解,经由边界条件的线性性质,最终得矩形区域拉普拉斯方程[\ref{eq:laplace6}]的解函数
\begin{equation*}
	u(x,y) = u_I (x,y)  +   u_{II}(x,y)  + u_{III}(x,y)  + u_{IV}(x,y)  
\end{equation*}	
~~\\ 

\begin{example}
	求圆域拉普拉斯方程的定解问题 
	$$  \displaystyle  \left \{ 
	\begin{array}{cc}
		\nabla ^2 u(r,\theta) =0, ~~~~ 0 < r< r_0 \\
		u(r_0,\theta )=f(\theta ) ,~~ 0<\theta <2\pi 
	\end{array}
	\right. $$
\end{example}
\emph{解:}
代入极坐标拉普拉斯算子,并添加周期性边界条件, 有
 $$  \displaystyle  \left \{ 
	\begin{array}{c}
		\left[\dfrac{\partial^2  }{\partial r^2 } +\dfrac{1}{r } \dfrac{\partial  }{\partial r } +
			\dfrac{1}{r^2 } \dfrac{\partial ^2  }{\partial \theta ^2}
		\right] u  =0, \qquad ~~ 0 < r< r_0 \\
		u(r,\theta )=u(r,\theta+2\pi ), ~~u'(r,\theta )=u'(r,\theta+2\pi ) \\
		u(r_0,\theta )=f(\theta ) ,~~~~~~~~~~~~ 0<\theta <2\pi \qquad \qquad \qquad
	\end{array}
	\right. $$
	方程是齐次线性微分方程,考虑分离变量,令 $$\displaystyle  u(r,\theta)=R(r) \Theta(\theta)$$ 代回原方程 
    $$\displaystyle  R''\Theta +\dfrac{1}{r^2} R\Theta '' +\dfrac{1}{r}R'\Theta=0 $$ 
	整理,并令等于 $ \lambda  $ 
	$$\displaystyle  \dfrac{r^2R''+rR'}{R}=-\dfrac{\Theta '' }{\Theta} =\lambda $$ 
	分离成两个常微分方程 \\
	方程-I、 $$\displaystyle 	\Theta '' + \lambda \Theta =0 $$   
	方程-II、$$\displaystyle  r^2 R'' +r R' -\lambda R =0 $$  	
	~~\\ 
	方程-I在$ \lambda >0 $时有虚根,得通解 
	$$\displaystyle  \Theta(\theta)=A\cos \sqrt{\lambda } \theta+B\sin \sqrt{\lambda }\theta $$ 
	分别取$ \theta =0, 2\pi $, 得 
	$$\displaystyle  \Theta(0)=A\cos \sqrt{\lambda } (0)+B\sin \sqrt{\lambda }(0) = A $$ 
	$$\displaystyle  \Theta(2\pi)=A\cos 2\pi\sqrt{\lambda }+B\sin 2\pi\sqrt{\lambda }$$ 
	由 $ \Theta(0) = \Theta(2\pi) $, 得
	$$ A(\cos 2\pi\sqrt{\lambda }-1)+B\sin 2\pi\sqrt{\lambda } =0 \quad \cdots (1)$$
	对通解求导,得
  $$\displaystyle  \Theta'(\theta)=-A\sin \sqrt{\lambda } \theta+B\cos \sqrt{\lambda }\theta $$ 
分别取$ \theta =0, 2\pi $, 得 
	$$\displaystyle  \Theta'(0)=-A\sin \sqrt{\lambda } (0)+B\cos \sqrt{\lambda }(0) = B $$ 
	$$\displaystyle  \Theta'(2\pi)=-A\sin 2\pi\sqrt{\lambda }+B\cos 2\pi\sqrt{\lambda }$$ 
	由 $ \Theta'(0) = \Theta'(2\pi) $, 得
	$$ A(-\sin 2\pi\sqrt{\lambda })+B(\cos 2\pi\sqrt{\lambda }-1) =0 \quad \cdots (2)$$
	联立(1)(2),得方程组
	$$ \left[
	\begin{array}{lll}
		\cos (\sqrt {\lambda} 2\pi )-1  & \sin (\sqrt {\lambda} 2\pi )\\
		-\sin (\sqrt {\lambda} 2\pi ) & \cos (\sqrt {\lambda} 2\pi )-1
	\end{array} \right] 
	\left[
	\begin{array}{lll}
		A\\
		B
	\end{array} \right] 
	=
	\left[
	\begin{array}{lll}
		0\\
		0
	\end{array} \right]
	$$
	由系数行列式为零,得
	$$(\cos (\sqrt {\lambda} 2\pi )-1 ) ^2 + \sin ^2 (\sqrt {\lambda} 2\pi ) =0$$
	因此 
$$\cos (\sqrt {\lambda} 2\pi)=1, \implies \sqrt {\lambda} 2\pi = n(2\pi)$$  
	解得\\   	
	固有值:$$\lambda _n =n^2 ~,~~ (n=0,1,2,...)$$   
	固有函数:$$\displaystyle  \Theta_n (\theta)=A_n\cos n \theta +B_n \sin n \theta $$
~~\\ 
	解方程II:
	$$\displaystyle  r^2 R'' +r R' -\lambda R =0 $$  
	把$\lambda_n =n^2 $代入, 得 
	$$\displaystyle  r^2 R'' +r R' -n^2R =0 $$  
	这是欧拉方程\\ 
	解欧拉方程, 令 $ r=exp(t) $ ,有 $t=\ln r$, 求导 
	$$ \displaystyle \frac{dR}{dr} =\frac{dR}{dt} \frac{dt}{dr} =\frac{1}{r} \frac{dR}{dt} $$  
	$$ \displaystyle \frac{d^2R}{dr^2} =-\frac{1}{r^2}\frac{dR}{dt} + \frac{1}{r} \frac{d}{dr} (\frac{dR}{dt} )= \frac{1}{r^2} (\frac{d^2R}{dt^2}-\frac{dR}{dt} )$$ 
	代回欧拉方程,得
	$$ \displaystyle   \dfrac{d^2R}{dt^2} -n^2 R =0 $$ 
	特征方程有两相异实根,得通解: 
	$$ R_n=C_n\exp(nt)+D_n \exp(-nt) $$
	把 $t=\ln r$ 代回,得
	$$R_n(r)=C_n r^n +D_nr^{-n}$$ 
	第二项在$r\to 0$时发散,应删除,得
	$$R_n(r)= C_n r^n,  ~~ (n=0,1,2,......) $$
	两者结合,得原方程的基本解
	$$\begin{array}{llll}
		u_n(r,\theta) &= R_n(r) \Theta_n (\theta) \\
		&=  r^n(a_n \cos n\theta +b_n \sin n \theta )   
	\end{array}$$ 
	叠加解 
	$$\begin{array}{llll}
		u(r, \theta) &=& \dfrac{a_0}{2}  +\sum\limits_{n=1}^{\infty }r^n (a_n  \cos n\theta +b_n \sin n \theta ) 
	\end{array}$$ 
	在上式中取$r =r_0 $, 结合定解条件:$ u(r_0,\theta)=f (\theta) $, 有 
	$$ f (\theta) = \dfrac{a_0}{2}  +\sum\limits_{n=1}^{\infty } (a_n r_0^n\cos n\theta +b_n r_0^n \sin n \theta ) $$
	由傅里叶公式
	$$\begin{aligned}
		a_n r_0^n & = \frac{2}{l}\int\limits_{0}^{l} f(x) \cos n x dx \\ 
		a_n r_0^n & = \frac{2}{2\pi  } \int\limits_{0}^{2\pi} f(\theta) \cos n \theta d\theta \\
		a_n & = \frac{1}{r_0^n \pi  } \int\limits_{0}^{2\pi} f(\theta) \cos n \theta d\theta 
	\end{aligned}$$
	同理,
	$$  \displaystyle  b_n = \dfrac{1}{r_0 ^n \pi }  \int\limits_{0}^{2\pi} f(\theta) \sin n \theta d\theta $$ 
	因此,原方程的解为
	$$\boxed{\begin{aligned}
			u(r, \theta) &= \dfrac{a_0}{2}  +\sum\limits_{n=1}^{\infty }r^n (a_n  \cos n\theta +b_n \sin n \theta )\\ 
			a_n & = \frac{1}{r_0^n \pi  } \int\limits_{0}^{2\pi} f(\theta) \cos n \theta d\theta \\
			b_n &= \dfrac{1}{r_0 ^n \pi }  \int\limits_{0}^{2\pi} f(\theta) \sin n \theta d\theta
		\end{aligned}} $$
	结束! 

~~\\ 

\begin{example}
	基于公式求解如下圆域边值问题
	$$  \displaystyle  \left\{ 
	\begin{array}{cc}
		\displaystyle {	\dfrac{\partial^2 u }{\partial r^2 } +\dfrac{1}{r } \dfrac{\partial u }{\partial r } +
		\dfrac{1}{r^2 } \dfrac{\partial ^2 u }{\partial \theta ^2
		} } =0, ~~ 0<r<R\\
		u(R,\theta )=A\cos(\theta),~~~~~~~~~ 0<\theta <2\pi 
	\end{array}
	\right. $$
\end{example}
\emph{解:}
	方程的解为
	$$\begin{aligned}
		u(r, \theta) &= \dfrac{a_0}{2}  +\sum\limits_{n=1}^{\infty }r^n (a_n  \cos n\theta +b_n \sin n \theta )
	\end{aligned}$$
	计算系数 
	$$\begin{aligned}
	a_n & = \frac{1}{r_0^n \pi  } \int\limits_{0}^{2\pi} f(\theta) \cos n \theta d\theta \\
	& = \frac{1}{R^n \pi  } \int\limits_{0}^{2\pi} A\cos(\theta) \cos n \theta d\theta \\
	a_1& = \dfrac{A}{R\pi }  \int\limits_{0}^{2\pi} A\cos (\theta) \cos (\theta) d\theta  \\
	& = \frac{A}{R \pi } \frac{2\pi}{2} \\
	& = \dfrac{A}{R}, \quad (a_n =0, \quad n\ne 1)
 \end{aligned} $$
 $$\begin{aligned}
	b_n & = \frac{1}{r_0^n \pi  } \int\limits_{0}^{2\pi} f(\theta) \sin n \theta d\theta \\
	& = \frac{1}{R^n \pi  } \int\limits_{0}^{2\pi} A\cos(\theta) \sin n \theta d\theta \\
	&= 0
 \end{aligned} $$
 把系数代回,得方程的解
 $$ u(r, \theta) = \dfrac{A}{R} r \cos \theta $$
 解函数的图像如图[\ref{fig:lap0.png}]所示,
   \begin{figure}[htbp]
	\centering
	\includegraphics[width=0.5\textwidth]{figs/lap0.png}
	\caption{解函数的图像}
	\label{fig:lap0.png}
  \end{figure}
 结束!

~~\\ 
~~\\
\begin{Exercises} 
\item 已知球坐标基矢 $$\vec{e}_r = \left[\begin{array}{ccc}
		\sin \theta \cos \varphi \\
		\sin \theta \sin \varphi \\
		\cos \theta
	\end{array}\right], \quad  
	\vec{e}_\theta = \left[\begin{array}{ccc}
		\cos \theta \cos \varphi \\
		\cos \theta \sin \varphi \\
		-\sin \theta
	\end{array}\right], \quad  
	\vec{e}_\varphi = \left[\begin{array}{ccc}
		-\sin \varphi \\
		\cos \varphi \\
		0
	\end{array}\right]
	$$ 
	试证明 
	\begin{equation*}
		\vec{e}_{i} \cdot \vec{e}_{j} = \delta_{ij}, \quad \dfrac{\partial \vec{e}_{\varphi}}{\partial \varphi} =  -\left[\begin{array}{c} \cos \varphi \\ \sin \varphi \\ 0\end{array}\right]
	\end{equation*} 
	\item 已知直角坐标与柱面坐标有如下换算关系,
	$$\begin{cases}
		x= \rho\cos \varphi  \\
		y= \rho\sin \varphi  \\
		z= z
	\end{cases} $$ 
	试证明柱面坐标系下拉普拉斯算符为 
	$$\nabla^2=\frac{\partial^2}{\partial \rho^2}+\frac{1}{\rho} \frac{\partial}{\partial \rho}+\frac{1}{\rho^2} \frac{\partial^2}{\partial \varphi^2}+\frac{\partial^2}{\partial z^2}$$

	\item 写出矢量微分算子在直角坐标系、极坐标系、球坐标系和柱面坐标系下的具体形式\\ 
	\item 求函数
	$$
	u(x, y, z)=\ln \left(x^2+y^2+z^2\right), \quad(x, y, z) \neq(0,0,0)
	$$
	在球坐标系下的拉普拉斯 \\
	\item 试证明
	$$
	\frac{1}{r} \frac{d^2(r R)}{d r^2}=\frac{1}{r^2} \frac{d}{d r}\left(r^2 \frac{d R}{d r}\right)
	$$
	\item 试证明直角坐标的角动量Z分量算子,在球坐标系的具体表示为
	\begin{equation*}
		\hat{L}_z= -i \hbar \frac{\partial }{\partial\varphi }
	\end{equation*}
		\item 求解圆域边值问题\\
		~~\\  
		$\displaystyle  \begin{array}{lllllllll}
		&\begin{cases}
			\dfrac{\partial^2 u }{\partial r^2 } +\dfrac{1}{r } \dfrac{\partial u }{\partial r } +
			\dfrac{1}{r^2 } \dfrac{\partial ^2 u }{\partial \theta ^2  } =0, ~~ 0<r<1~~,~~ 0<\theta<2\pi\\
			u(1,\theta)= A\cos 2 \theta +B \cos 4 \theta \\	
		\end{cases} \\	
		\end{array}$ \\
		~~\\  
		\item 求解圆域边值问题\\
		~~\\  
		$\displaystyle  \begin{array}{lllllllll}
		&\begin{cases}
			\dfrac{\partial^2 u }{\partial r^2 } +\dfrac{1}{r } \dfrac{\partial u }{\partial r } =0, ~~ 0<r<l~~,~~ 0<\theta<2\pi\\
			u(l,\theta)= 1 \\	
		\end{cases} \\	
		\end{array}$ \\
		~~\\ 
		\item 求解矩形域边值问题\\
		~~\\  
		$\begin{array}{lllllllll}
			& \begin{cases}
				u_{xx} +u_{yy} =0 ,~~~~ (0<x, y<1)\\
				u(x,0)= u(0,y)=u(x,1)= 0 \\
				u(1,y)= \sin 2\pi y
			\end{cases}	\qquad 
			\begin{cases}
				u_{xx} +u_{yy} =0 ,~~~~ (0<x, y<1)\\
				u(1,y)= u(0,y)=u(x,0)= 0 \\
				u(x,1)= \sin n\pi x
			\end{cases} \\	
		\end{array}$ 
\item 求固有值问题-III 
	 $$\begin{cases}
			X'' (x)  + \lambda X =0   ~~,~~ 0<x<l\\
			X' (0) =0, X (l) =0
	\end{cases}$$	
	\\	
\item 求固有值问题-IV
	$$\begin{cases}
		X'' (x)  + \lambda X =0   ~~,~~ 0<x<l\\
		X (0) =0, X' (l) =0
	\end{cases} $$	
	\\
\item 求固有值问题-V
	$$\begin{cases}
		X'' (x)  + \lambda X =0   ~~,~~ 0<x<2\pi\\
		X (0) =X (2\pi), \quad X'(0) = X' (2\pi)
	\end{cases} $$	
	\\
\item 求固有值问题-VI
	$$ \begin{cases}
		X ^{\prime\prime} -2 a X ^{\prime} +\lambda X=0, \quad 0<x<l \\
		X(0)=0, \quad X(l)=0
		\end{cases}  $$
\item 列表法求积分
	$$\int\limits_{0}^{\pi} x^2 (\pi-x)^2  \sin nx  dx, \qquad \int_0^{\infty} e^{-2 r / a_0} r^3 d r  $$
\item 求解热传导方程的定解问题
	$$\begin{cases}
		u_{t} =a^2u_{xx} ~~,~~ 0<x<l, t>0\\
		u_x(0,t) =u_x(l,t)=0 \\
		u(x,0) =x(l-x/2)
	\end{cases} \qquad \begin{cases}
		u_{t} =2u_{xx} ~~,~~ 0<x<l, t>0\\
		u(0,t) =u(l,t)=0 \\
		u(x,0) =x (l-x)
	\end{cases} $$
\item 试利用热传导方程定解公式求定解问题 \\
$$
\left\{\begin{array}{l}
u_t=3 u_{x x}, \quad(0 \leqslant x \leqslant 2, t \geqslant 0) \\
\left.u_x(x, t)\right|_{x=0}=0,\left.u_x(x, t)\right|_{x=2}=0 \\
\left.u(x, t)\right|_{t=0}=(x-2)^2
\end{array}\right.
$$
\item 若$\psi _1 (x,t)$ 满足方程
	$$u_{t} - a^2 u_{xx} = f_1(x, t)$$
	$\psi _2 (x,t)$ 满足方程
	$$u_{t} - a^2 u_{xx} = f_2(x, t)$$
	试证明它们的线性叠加
	$$
	\psi (x,t) = a_1\psi _1 (x,t) + a_2\psi _2 (x,t) 
	$$ 是如下方程的解
	$$u_{t} - a^2 u_{xx} = a_1f_1(x, t) + a_2f_2(x, t)$$
\item 试利用分离变量法求解波动方程定解问题 \\
	$$~\qquad \begin{cases}
		u_{tt} =a^2u_{xx} ~~,~~ 0<x<l, t>0\\
		u(0,t) =u(l,t)=0 \\
		u(x,0) =sin \dfrac{\pi x}{l} ,  u_t (x,0)=\sin \dfrac{2\pi x}{l} 
	\end{cases}  $$ 
\item 试利用波动方程定解公式求定解问题 \\
		$$\begin{cases}
			u_{tt} =u_{xx} ~~,~~ -1<x<1, t>0\\
			u(-1,t) =u(1,t)=0  \\
			u(x,0) =\sin 4\pi x ,  u_t (x,0)=x (1-x) 
		\end{cases} \quad \begin{cases}
			u_{tt} =a^2u_{xx} ~~,~~ 0<x<1, t>0\\
			u(0,t) =u(1,t)=0 \\
			u(x,0) =3sin \dfrac{3\pi x}{2} +6\sin\dfrac{5\pi x}{2},  u_t (x,0)=0
		\end{cases} $$ 
\item 求解波动方程非定常问题
$$
\left\{\begin{array}{l}
u_{t t}=a^2 u_{x x},(-l<x<l, t>0) \\
\left.u\right|_{x=-l}=\left.u\right|_{x=l}=0 \\
\left.u\right|_{t=0}= f(x) \\
\left.u_t\right|_{t=0}=0
\end{array}\right.
$$
\item 求解波动方程定常问题
$$
\left\{\begin{array}{l}
u_{t t}=a^2 u_{x x},(0<x<l, t>0) \\
\left.u\right|_{x=0}=\left.u_x\right|_{x=l}=0 \\
\left.u\right|_{t=0}=3 \sin 3 \pi x / 2 l+6 \sin 5 \pi x / 2 l \\
\left.u_t\right|_{t=0}=0
\end{array}\right.
$$
\item 求解热传导方程定常问题
$$
\left\{\begin{array}{l}
u_t=a^2 u_{x x},(t>0,0<x<l) \\
u(0, t)=0, u_x(l, t)=0 \\
u(x, 0)=x^2(l-x)
\end{array}\right.
$$
\item 求解拉普拉斯方程的定常问题
$$
\left\{\begin{array}{l}
u_{x x}+u_{y y}=0, \quad -1<x, y<1 \\
u(-1, y)=0, u_y(1, y)=\sin 2 \pi y \\
u(x, -1)=0, u_x(x, 1)=0
\end{array}\right.
$$
\item 求解拉普拉斯方程的定常问题
$$
\left\{\begin{array}{l}
u_{x x}+u_{y y} +u_{z z}=0, \quad 0<x, y, z<1 \\
u|_{x=0}=u|_{y=0}=u|_{x=1}=u|_{y=1} =0\\
u|_{z=0}=0, \quad u|_{z=1}=\sin(\pi x)\sin(3\pi y)
\end{array}\right.
$$
\end{Exercises}
 

 

	

 